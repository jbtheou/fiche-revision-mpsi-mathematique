% Tout ce qui est mis derrière un « % » n'est pas vu par LaTeX
% On appelle cela des « commentaires ».  Les commentaires permettent de
% commenter son document - comme ce que je suis en train de faire
% actuellement - et de cacher du code - cf. la ligne \pagestyle.

\documentclass[a4paper, titlepage,twoside]{book}

% ************************
% * fichier de préambule *
% ************************
 
% ***** extensions *****
\def\renamesymbol#1#2{
  \expandafter\let\expandafter\newsym\expandafter=\csname#2\endcsname
  \expandafter\global\expandafter\let\csname#1#2\endcsname=\newsym
  \expandafter\global\expandafter\let\csname#2\endcsname=\origsym
}

\usepackage[utf8x]{inputenc}			  % Utilisation du UTF8
\usepackage{textcomp}				  % Accents dans les titres
\usepackage [ french ] {babel}                    % Titres en français
\usepackage [T1] {fontenc} 			  % Correspondance clavier -> document
\usepackage{wasysym}
\renamesymbol{wasysym}{iint}
\renamesymbol{wasysym}{iiint}
\usepackage[Lenny]{fncychap}                      % Beau Chapitre
\usepackage{dsfont}                    	  % Pour afficher N,Z,D,Q,R,C
\usepackage{fancyhdr}                             % Entete et pied de pages
\usepackage [outerbars] {changebar}               % Positionnement barre en marge externe
\usepackage{amssymb}
\usepackage{amsmath}				  % Utilisation de la librairie de Maths
%\usepackage{amsfont}				  % Utilisation des polices de Maths
\usepackage{enumerate}				  % Permet d'utiliser la fonction énumerate
\usepackage{dsfont}				  % Utilisation des polices Dsfont
\usepackage{ae}					  % Rend le PDF plus lisible
\usepackage[pdftex]{graphicx}                 % dernière étant la langue principale
\usepackage{color}
\usepackage{palatino}
\usepackage{paralist}
\usepackage{natbib}
\usepackage[margin=3cm]{geometry}
\usepackage{fancyhdr}
\usepackage[Lenny]{fncychap}
\usepackage{graphicx}
\usepackage{wrapfig}
\usepackage{url}
\usepackage{graphics}

\definecolor{Dark}{gray}{.2}
\definecolor{Medium}{gray}{.6}
\definecolor{Light}{gray}{.8}
\newcommand*{\plogo}{\fbox{$\mathcal{PL}$}}


\newtheorem{de}{Définition}
\newtheorem{theo}{Théorème}
\newtheorem{prop}{Propriété}
\newtheorem{princ}{Principe}
\newtheorem{conv}{Convention}
\newtheorem{loi}{Loi}
\newtheorem{voc}{Vocabulaire}
\newtheorem{enon}{\'Enonc\'e}
\newtheorem{nota}{Nota}



\newlength{\drop}


\newcommand*{\titleGMPHY}{\begingroup% Gentle Madness
\setlength{\drop}{0.1\textheight}
%\vspace*{\baselineskip}
\vfill
  \hbox{%
  \hspace*{0.2\textwidth}%
  \rule{1pt}{\textheight}
  \hspace*{0.05\textwidth}%
  \parbox[b]{0.75\textwidth}{
  \vbox{%
    %\vspace{\drop}
    {\noindent\Huge\bfseries Fiches de Révision\\[0.5\baselineskip]
               MPSI}\\[2\baselineskip]
    {\Large\itshape TOME II - Mathématiques}\\[4\baselineskip]
    {\Large Jean-Baptiste Théou}\par
    \vspace{0.5\textheight}
    {\noindent Creactive Commons - Version 0.1}\\[\baselineskip]
    }% end of vbox
    }% end of parbox
  }% end of hbox

\null
\endgroup}

\newcommand*{\titleGMMATH}{\begingroup% Gentle Madness
\setlength{\drop}{0.1\textheight}
%\vspace*{\baselineskip}
\vfill
  \hbox{%
  \hspace*{0.2\textwidth}%
  \rule{1pt}{\textheight}
  \hspace*{0.05\textwidth}%
  \parbox[b]{0.75\textwidth}{
  \vbox{%
    %\vspace{\drop}
    {\noindent\Huge\bfseries Fiches de Révision\\[0.5\baselineskip]
               MPSI}\\[2\baselineskip]
    {\Large\itshape TOME II - Mathématiques}\\[4\baselineskip]
    {\Large Jean-Baptiste Théou}\par
    \vspace{0.5\textheight}
    {\noindent Creactive Commons}\\[\baselineskip]
    }% end of vbox
    }% end of parbox
  }% end of hbox

\null
\endgroup}


\begin{document}

\pagestyle{empty}
\titleGMPHY
\clearpage
\frontmatter                  % Prologue.
\chapter{Licence}
J'ai décidé d'éditer cet ouvrage sous la licence Créative Commons suivante : CC-by-nc-sa. Pour plus d'information :\\
http://creativecommons.org/licenses/by-nc-sa/2.0/fr/.\\
Ce type de licence vous offre une grande liberté, tout en permettant de protéger mon travail contre une utilisation commercial à mon insu par exemple.\\
Pour plus d'information sur vos droits, consultez le site de Créative Commons
\chapter{Avant-propos}
Il y a un plus d'un an, au milieu de ma SUP MP, j'ai décidé de faire mes fiches de révision à l'aide de Latex, un "traitement de texte" très puissant. Il en résulte les fiches qui suivent. Je pense que travailler sur des fiches de révision, totalement séparé de notre cours, est un énorme plus, et réduit grandement la quantité de travail pour apprendre son cours, ce qui laisse plus de temps pour les exercices. Mon experience en tout cas va dans ce sens, j'ai notablement progressé à l'aide de ces fiches.\\
J'ai décidé de les rassembler sous forme d'un "livre", ou plutôt sous forme d'un recueil. Ce livre à pour principal interet pour moi d'être transportable en cours. C'est cet interet qui m'a poussé à faire ce livre.\\
Dans la philosophie de mes fiches de révision, ce livre est disponible gratuitement et librement sur mon blog. Il est édité sous License Créative Commons. Vous pouvez librement adapter ce libre à vos besoins, les sources Latex sont disponibles sur mon blog. Je pense que pour être en accord avec la philosophie de ces fiches, il serai bien que si vous effectuez des modifications de mon ouvrage, vous rendiez ces modifications disponible à tous. Je laisserai volontiers une place pour vos modifications sur mon blog. Je pense sincèrement que ce serai vraiment profitable au plus grand nombre, et dans la logique de mon travail.\\
J'ai hiérarchisé mon ouvrage de façon chronologique, tout en rassemblant les chapitres portant sur le même sujet sous une même partie. Les parties sont rangées dans l'ordre "d'apparition" en MPSI. J'ai mis en Annexe des petites fiches de méthodologie, qui peuvent s'avérer utiles.\\
Je vous souhaite une bonne lecture, et surtout une bonne réussite.\\
Jean-Baptiste Théou
\chapter{Remerciements}
Je tient à remercier tout particulièrement Yann Guillou, ex Professeur de Physique-Chimie en MPSI au Lycée Lesage, actuellement en poste en Guadeloupe, qui m'a permis de consolider mes connaisances en physique et qui m'a ouvert les yeux sur la réalité de la physique et sur son histoire. Ces "digressions historiques" resterons de bons moments dans mon esprit, pour longtemps. Je remercie aussi Paul Maheu, Professeur de Mathématiques en MPSI au Lycée Lesage, qui m'a permis d'aquérir de solides connaisances en Mathématiques.\\
Sans eux, ce livre ne pourrai exister.\\
Pour finir, je me dois à mon avis d'insérer cette citation dans mon ouvrage, citation que nous a donné Mr Guillou pour nos premiers coups de crayon en Prépa. Elle est à méditer ....
  \begin{flushright}
    \begin{tabular}{@{}p{6cm}@{}}
      {\raggedleft \itshape Je suis convaincu qu'il est plus bénéfique pour un étudiant de retrouver des démonstrations à partir de quelques indications que de les lire et de les relire ....\\
      Qu'il les lisent une fois, qu'il les retrouvent souvent\par}\\

   {\raggedleft \textsc{Srinivâsa Aiyangâr Râmânujan}(1886-1920)\par}

 \end{tabular}

\end{flushright}




\mainmatter           % On passe aux choses serieuses
\part{Fonctions de $\mathbb{R}$ dans $\mathbb{R}$}
\chapter{$\mathbb{R}$}
\section{D\'efinitions}
\section{Structure}
\begin{de}
($\mathbb{R}$,+,$\times$) est un corps totalement ordonnée. On dit qu'il est archimédien.\\
\end{de}
\begin{de}
La relation "$\leq$" est une relation d'ordre. Elle est :
\begin{itemize}
 \item[$\rightarrow$] Reflexive : $$\forall x \in \mathbb{R}~ x \leq x$$
 \item[$\rightarrow$] Anti-symétrique : $$\forall(x,y)\in \mathbb{R}^2 \mbox{ si } : (x \leq y,y\leq x),  \mbox{ alors } x=y$$
 \item[$\rightarrow$] Transitive : $$\forall(x,y,z)\in \mathbb{R}^3 \mbox{ si } : (x \leq y,y\leq z), \mbox{ alors } x\leq z$$
\end{itemize}
\end{de}
\subsection{Majorant - Minorant}
Soit A un ensemble
\subsubsection{Majorant}
\begin{de}
Si M est un majorant de A, avec $M \in A$, alors :
$$M = Max(A)$$
\end{de}
\begin{de}
Si M est le plus petit des majorants de A, alors M est la borne supérieure de A : 
$$M = Sup(A)$$
\end{de}
\begin{prop}
Si $A~ c~ \mathbb{R}$, si Max(A) existe, alors Sup(A) existe et :
$$Sup(A) = Max(A)$$
\end{prop}
\subsubsection{Minorant}
\begin{de}
Si M est un minorant de A, avec $M \in A$, alors :
$$M = Min(A)$$
\end{de}
\begin{de}
Si M est le plus grand des minorant de A, alors M est la borne inférieure de A : 
$$M = Inf(A)$$
\end{de}
\begin{prop}
Si $A~ c~ \mathbb{R}$, si Min(A) existe, alors Inf(A) existe et :
$$Min(A) = Inf(A)$$
\end{prop}
\subsection{Borne supérieure - Borne inférieure}
\begin{prop}
Toute partie de $\mathbb{R}$ non vide et minorée possède une borne inférieure.
\end{prop}
\begin{prop}
Toute partie de $\mathbb{R}$ non vide et majorée possède une borne supérieure.
\end{prop}
\begin{prop}
Toute partie de Z non vide et majorée possède un plus grand éléments. (Max)
\end{prop}
\subsection{Partie bornée de $\mathbb{R}$}
Soit $A$ une partie de E. On note ceci : $A \in P(E)$.\\
A est bornée si et seulement si :$$\exists M \in \mathbb{R}~ tq~ \forall a\in A,~ |a|\leq M$$
\begin{prop}
Propriété d'Archimède : Soient (x,y)$\in \mathbb{R}$ et x>0, alors :
$$\exists p \in Z~ tq~ y < px$$
\end{prop}
\subsection{Partie entière}
\begin{de}
Soit $x \in \mathbb{R}$.\\
Il existe un unique entier p telque p$\leq x$<p+1\\
Cette entier p en la partie entière de $x$. On le note $E(x)$.
\end{de}
\begin{de}
En complément, on défini la partie décimale de $x$, notée $D(x)$ :
$$D(x) = x-E(x)$$
\end{de}
\subsection{Densité}
\begin{de}
Soit A une partie de $\mathbb{R}$\\
A est dense dans $\mathbb{R}$ si, avec $x \neq y$ :
$$\forall(x,y) \in \mathbb{R}^2~ \exists a \in A~ tq~ a\in]x;y[$$
\end{de}
\begin{prop}
Puisque l'espace des fractions rationnels, notée Q, est dense dans $\mathbb{R}$, si $x \in \mathbb{R}$, alors il existe une suite de rationnelle qui converge vers x. 
\end{prop}
\section{Partie de $\mathbb{R}$}
\begin{de}
 Soient (a,b)$\in \mathbb{R}^2$. On appelle segment d'extrémité a,b :
$$[a,b] = \left\lbrace x\in \mathbb{R} / a\leq x \leq b\right\rbrace $$
\end{de}
\begin{de}
 Soit I une partie de $\mathbb{R}$. I est un intervalle si :
$$\forall x\in I, \forall y \in I,~ [x;y]~c~I$$
\end{de}
\subsection{Sous-groupes de ($\mathbb{R}$;+)}
\subsubsection{Critère de reconnaissance des sous-groupes}
\begin{de}
Soit H une partie de $\mathbb{R}$.\\
On dit de H est un sous-groupe de ($\mathbb{R}$;+) si (H;+) est un groupe. 
\end{de}
\begin{prop}
H est un sous-groupe si et seulement si :
\begin{enumerate}[1-]
 \item $H~ c~ \mathbb{R}$ et H non vide
 \item $\forall(x;y)\in H^2, x-y\in H$
\end{enumerate}

\end{prop}

 % Relue
\chapter{Limite d'une fonction}
\section{Définitions}
\begin{de}
Soit f une fonction, I un intervalle. \\
f est majorée sur I si :$$\exists m~ tq~ \forall x \in I~ f(x) < m$$
\end{de}
\begin{de}
On dit que f est croissante sur I si :
$$\forall (x,x') \in I^2~ si~ x<x',~ f(x) \leq f(x')$$
\end{de}
\subsection{Fonction k-lipschitzienne}
\begin{de}
Soit f : I $\rightarrow \mathbb{R}$.\\
f est k-lipschitzienne si : 
$$\forall (x,y) \in I^2~ |f(x)-f(y)|\leq k|x-y|$$
\end{de}
\begin{prop}
Soient f et g deux fonctions k-lipschitziennes sur I, alors f+g est aussi k-lipschitzienne sur I
\end{prop}
\subsection{Limite et continuité}
\subsubsection{Au voisinage d'un réel a}
Soit $a \in R$.
\begin{de}
La propriété P est vraie au voisinage de a si elle est vraie sur l'intersection de I et d'un intervalle ouvert de centre a :
$$(\exists \alpha > 0~ tq~ (P)~ soit~ vraie~ \forall x \in ]a-\alpha;a+\alpha[\cap I)$$
\end{de}
\subsubsection{Au voisinage de $+\infty$}
\begin{de}
(P) est vraie au voisinage de $+\infty$ si elle est vrai sur I$\cap]A;+\infty[$, avec A fixé.
\end{de}
\subsection{Limite}
\begin{de}
Soit (a,b) $\in \mathbb{R}^2$ :
$$(\lim_a f = b) \Leftrightarrow (\forall \varepsilon > 0~ \exists \alpha > 0~ \forall x \in D_f~ |x-a|\leq \alpha~ \Rightarrow~ |f(x)-b|<\varepsilon)$$ 
\end{de}
\begin{prop}
Soit a un réel, a = $+\infty$ ou a = $-\infty$.\\
On suppose que f et g coincident au voisinage de a, alors : 
$$\lim_a f = \lim_a g$$
\end{prop}
\subsection{Continuité}
\begin{de}
Si f est définie en a, la limite éventuelle en a est nécessairement b=f(a)
\end{de}
\begin{prop}
Soit $a \in \mathbb{R}$ :\\
\begin{itemize}
 \item Si a $\in D_f$
\begin{itemize}
 \item Si $\underset{a}\lim f = f(a)$, alors f est continue en a.
 \item Si $\underset{a}\lim f \neq f(a)$, alors c'est impossible.
 \item Si la limite n'existe pas, alors f n'est pas continue en a
\end{itemize}
\item Si a $\notin D_f$
\begin{itemize}
 \item Si $\underset{a}\lim f$ existe (dans $\mathbb{R}$), alors f est prolongable par continuité
 \item Si la limite n'existe pas, rien à dire, sauf que f n'est pas prolongable.
\end{itemize}
\end{itemize}
\end{prop}
\subsubsection{Caractérisation à l'aide de suite}
\begin{prop}
$$(\lim_a f = b) \Leftrightarrow (\forall(x_n) \mbox{ suite convergente de limite a, } f(x_n) \mbox{ converge vers b})$$
\end{prop}
On en déduit que : 
\begin{prop}
Si il existe $(u_n),(v_n)$ deux suite telque : 
\end{prop}
$$\left\{\begin{array}{l}
   \underset{n \mapsto +\infty}\lim (u_n) = a\\
   \underset{n \mapsto +\infty}\lim (v_n) = a\\
  \end{array}\right.$$
et avec b $\neq$ b': 
$$\left\{\begin{array}{l}
   \underset{n \mapsto +\infty}\lim (f(u_n)) = b\\
   \underset{n \mapsto +\infty}\lim (f(v_n)) = b'\\
   \end{array}\right.$$
alors $\underset{a}\lim f$ n'existe pas
\section{Limité ou continuité à gauche et à droite}
\subsection{Segment}
\begin{de}
Soit I C $\mathbb{R}$. I est un intervalle si $\forall (a,b) \in I^2~ \left[a,b \right] C I $\\
\end{de}
\begin{de}
Soit a $\in \mathbb{R}$, et I un intervalle.\\
a est interieur à I si : 
$$\exists \alpha > 0~ tq~ \left]a-\alpha,a+\alpha \right[ C I$$
L'interieur de I, notée $\overset{o}I$, est l'ensemble des points interieurs à I.
\end{de}
\subsection{Limite à droite, limite à gauche}
\subsubsection{Limite à droite}
\begin{de}
Soit f, fonction définie sur un intervalle I, sauf peut etre en a, avec a interieur à I.\\
La limite à droite de f en a est, si elle existe, la limite en a de la restriction de f à I$\cap \left]a,+\infty \right[$
On la note : 
$$\lim_{a^+} f$$
\end{de}
\subsubsection{Limite à gauche}
\begin{de}
Soit f, fonction définie sur un intervalle I, sauf peut etre en a, avec a interieur à I.\\
La limite à gauche, de f en a est, si elle existe, la limite en a de la restriction de f à I$\cap \left]-\infty,a \right[$
On la note : 
$$\lim_{a^-} f$$
\end{de}
\begin{prop}
Si f est défini au voisinage de a : \\
Si $a \in D_f$, la limite en a de f est b si et seulement si :
$$\left\{\begin{array}{l}
   \underset{a^+}\lim f = b\\
   \underset{a^-}\lim f = b\\
   f(a)=b\\
\end{array}\right.$$
Si $a \notin D_f$, la limite en a de f est b si et seulement si :
$$\left\{\begin{array}{l}
   \underset{a^+}\lim f = b\\
   \underset{a^-}\lim f = b\\
\end{array}\right.$$
\end{prop}
\begin{prop}
Si :
$$\lim_a f = b$$
et 
$$\lim_b g = c$$
Alors :
$$\lim_a gof = c$$
\end{prop}
\subsection{Continuité d'un intervalle}
\begin{prop}
Soit a $\in I$ :
$$(\lim_a f \mbox{ existe }) \Leftrightarrow (\mbox{ f est continue en a})$$
\end{prop}
\begin{prop}
Soit I un intervalle : 
$$(\mbox{ f est continue sur I }) \Leftrightarrow ( \forall a \in I, \mbox{ f est continue en a })$$
\end{prop}
\section{Image continue}
\subsection{D'un intervalle}
\subsubsection{Théorème des valeurs intermédiaires}
\begin{prop}
Soit I un intervalle, (a,b)$\in I^2$. Si f est continue sur I, et $y_0 \in \left[f(a),f(b)\right]$, alors :
$$\exists c \in \left[a,b\right]~ tq~ y_0 = f(c) $$
\end{prop}
\begin{prop}
Si f est continue sur I, un intervalle, si $a \in I$, et $b \in I$, et f(a) et f(b) sont de signes contraire, alors : 
$$\exists c \in \left[a,b\right]~ tq~ f(c) = 0$$
\end{prop}
\begin{prop}
Si I est un intervalle, et f continue sur I, alors f(I) est un intervalle.
\end{prop}
\subsection{D'un segment}
\begin{prop}
Soit f fonction continue sur [a,b], avec a et b réel.\\
Alors f est bornée sur [a,b]
\end{prop}
\begin{prop}
Soit f fonction continue sur [a,b], avec a et b réel.\\
Alors le Sup et l'Inf de la fonction sur [a,b] existent.
\end{prop}
\begin{prop}
Soit f continue sur [a,b]. Alors : 
$$\exists (m,M) \in \mathbb{R}^2, ~tq~ f(\left[a,b\right]) = \left[m,M\right] $$
\end{prop}
\section{Continuité uniforme sur un intervalle}
\begin{de}
Soit f fonction définie sur un intervalle I.\\
On dit que f est uniformement continue sur I si: 
$$\forall \varepsilon > 0,~ \exists \alpha >0~ tq~ \forall (x,y) \in I^2~ |x-y| < \alpha~ \Rightarrow |f(x)-f(y)|<\varepsilon$$
\end{de}
\begin{prop}
Si f est uniformement continue sur I, alors f est continue sur I.
\end{prop}
\begin{prop}
Une fonction k-lipschitzienne sur I est uniformement continue sur I.
\end{prop}
\begin{theo}
Théorème de Heine :\\
Toutes fonctions continue sur un segments [a,b] est uniformement continue sur le segment.
\end{theo}
\section{Fonction monotone}
\subsection{Théorème de la "limite monotone"}
\begin{prop}
Si f est croissante sur I, et a $\in \overset{o}I$, alors :
$$\lim_{a^-} f~ et~ \lim_{a^+} f \mbox{ existent, mais peuvent etre différentes}$$
\end{prop}
\subsection{Monotonie et continuité}
\begin{prop}
Soit f fonction définie et croissante sur I\\
$$( \mbox{ f est continue sur I })\Leftrightarrow(\mbox{ f(I) est un intervalle })$$
\end{prop}
\subsection{Théorème de la bijection}
\begin{prop}
Soit I un intervalle.\\
Si f est continue, strictement monotone sur I, alors f est une bijection de I sur l'intervalle f(I), et $f^{-1}$ est continue sur f(I).
\end{prop}

 % Relue
\chapter{Dérivation des fonctions de $\mathbb{R}$ dans $\mathbb{R}$}
\section{Définitions}
\subsection{Définitions}
\begin{de}
Soit f définie sur un voisinage d'un réel a.
f est dérivable en a si :
$$\lim_{x \mapsto a} \dfrac{f(x)-f(a)}{x-a}$$
existe dans $\mathbb{R}$\\
Si f est dérivable, cette limite est le nombre dérivée de f en a.
\end{de}
\begin{prop}
Le nombre dérivé est la pente d'une droite passant par a. Cette droite est appelé tangente à la courbe représentative de f au point d'abscisse a.\\
L'équation de cette tangente est :
$$y = f(a) +f'(a)(x-a)$$
\end{prop}
\subsection{Lien entre tangente et dérivabilité}
Soit $x \in D_f$.
\begin{prop}
 Si on peut écrire f(x) sous la forme :
$$f(x) = f(a)+A(x-a)+\varepsilon(x)(x-a)$$
avec :
$$\lim_{x \rightarrow a} \varepsilon(x) = 0$$
alors f est dérivable en a et f'(a) = A.
\end{prop}
\subsection{Continuité et dérivabilité}
\begin{prop}
Si f est dérivable en a, alors f est continue en a
\end{prop}
\begin{prop}
f est dérivable en a si et seulement si :
$$\left\{\begin{array}{l}
    \mbox{ f est dérivable à droite en a}\\
    \mbox{ f est dérivable à gauche en a}\\
    f'_d(a) = f'_g(a)
  \end{array}\right.$$
\end{prop}

\subsection{Théorème de Rolle}
\begin{theo}
 Si :
$$\left\{\begin{array}{l}
     \mbox{ f est continue sur [a,b]}\\
     \mbox{ f est dérivable sur ]a,b[}\\
     f(a) = f(b)
  \end{array}\right.$$
alors : 
$$\exists c \in ]a,b[~ tq~ f'(c)=0$$
\end{theo}
\subsection{Théorème des accroissement finies}
\begin{theo}
Si :
$$\left\{\begin{array}{l}
     \mbox{ f est continue sur [a,b]}\\
     \mbox{ f est dérivable sur ]a,b[}\\
  \end{array}\right.$$
alors :
$$\exists c \in ]a,b[~ tq~ \dfrac{f(b)-f(a)}{b-a} = f'(c)$$
\end{theo}
\subsection{Inégalité des accroissement finies}
\begin{theo}
Si :
$$\left\{\begin{array}{l}
     \mbox{ f est continue sur [a,b]}\\
     \mbox{ f est dérivable sur ]a,b[}\\
     \mbox{ f' est borné sur ]a,b[}
  \end{array}\right.$$
Soit M un majorant de |f'|, alors :
$$|f(b) - f(a)| \leq M|b-a|$$
\end{theo}
\subsubsection{Conséquence}
\begin{itemize}
 \item[$\rightarrow$] Si f est croissante sur I, alors f'(a) $\geq 0$
 \item[$\rightarrow$] Si f et g sont dérivable sur [a,b], avec : $\forall x \in [a,b]~ f'(x) \leq g'(x)$, alors :
$$f(b)-f(a) \leq g(b)-g(a)$$
\end{itemize}
\subsection{Classe d'une fonction}
Soit f fonction, I un intervalle
\begin{de}
f est de classe $C^n$ sur I si $f^{(n)}$ est définie et continue sur I
\end{de}
\subsubsection{Opération}
\begin{prop}
Soit I un intervalle, soit $n \in N$.\\
La somme, le produit, la composé de fonction $C^n$ sur I, sont des fonctions $C^n$ sur I
\end{prop}
\subsection{Formulaire}
$\forall n \in N$, $\forall x \in \mathbb{R}$ :
$$cos^{(n)}(x) = cos(x+\dfrac{n\pi}{2})$$
$$sin^{(n)}(x) = sin(x+\dfrac{n\pi}{2})$$
\subsection{Formule de Leinbniz}
Soient f et g deux fonctions n fois dérivable sur I :
$$(fg)^{n)} = \sum_{k=0}^n \left( \dfrac{n}{k}\right) f^{(k)}.g^{(n-k)}$$
 % Relue
\chapter{Étude locale d'une fonction}
\section{Étude locale}
\subsection{Dominance - Équivalence - Négligeabilité}
Soit $a \in \mathbb{R} \cup\left\{+\infty,-\infty\right\}$. Soient f et g deux fonctions définie au voisinage de a sauf peut être en a.
\begin{de}
 On dit que f est dominée par g au voisinage de a si $\exists V_a$, voisinage de a telque $\mid \dfrac{f}{g} \mid$ soit majorée de $V_a$ 
$$(f(x) = 0(g(x)))\Leftrightarrow(\exists V_a,\mbox{ voisinage de a}, \exists M \in \mathbb{R} \mbox{ telque } \forall x \in V_a~ \mid f(x) \mid \leq M \mid g(x) \mid)$$
\end{de}
\begin{de}
On dit que f est négligeable devant g au voisinage de a, si, pour a $\in \mathbb{R}$ :
$$\forall \varepsilon > 0~ \exists \alpha > 0~ telque~ \forall x \in~ ]a+\alpha;a-\alpha[~ |f(x)|\leq \varepsilon |g(x)|$$
On le note $f(x) \ll g(x)$ et $f(x)=o(g(x))$. On a : 
$$(f(x) \ll g(x) ) \Leftrightarrow (\lim_{x \rightarrow a} \dfrac{f(x)}{g(x)} = 0)$$
La définition est identique si a est infini
\end{de}
\begin{de}
 On dit que f est équivalent à g, si :
$$f(x)-g(x) \ll g(x)$$
On note $f(x) \sim g(x)$. Et on a : 
$$(f(x)\sim g(x)) \Leftrightarrow (\lim_{x \rightarrow a} \dfrac{f(x)}{g(x)}=1)$$
\end{de}
\subsection{Comparaison successives}
Soit f,g,h trois fonctions défini au voisinage de a, sauf peut être en a.\\
Si :
\begin{itemize}
 \item[$\rightarrow$] Si $f(x)\ll g(x)$, $g(x)\ll h(x)$ alors :
$$f(x)\ll h(x)$$
 \item[$\rightarrow$] Si $f(x)\ll g(x)$, $g(x) \sim h(x) $ alors :
$$f(x) \ll h(x) $$
 \item[$\rightarrow$] Si $f(x) \sim g(x)$, $g(x) \ll h(x)$ alors :
$$f(x) \ll h(x)$$
 \item[$\rightarrow$] Si $f(x) \sim g(x)$, $g(x) \sim h(x)$ alors 
$$f(x) \sim g(x)$$
\end{itemize}
\subsection{Échelle de comparaison}
Au voisinage de 0 :
$$0 \ll .. \ll x^2 \ll x \ll 1 \ll ln(x) \ll \dfrac{1}{x}$$
Au voisinage de $\infty$
$$0 \ll \dfrac{1}{x^2} \ll \dfrac{1}{x} \ll 1 \ll 1 \ll ln(x) \ll \sqrt{x} \ll x^2 \ll e^x$$
\subsection{Règles de Manipulation}
\subsubsection{Somme de deux fonctions}
Si, au voisinage de a :
\[\left.\begin{array}{l}
   f(x) \sim \alpha u(x)\\
   g(x) \sim \beta u(x) \\
   \alpha + \beta \neq 0
  \end{array}\right\}
\mbox{Alors } f(x)+g(x) \sim (\alpha+\beta) u(x)\]
\subsubsection{Produit, rapport, valeur absolu}
\[\left.\begin{array}{l}
   f(x) \sim u(x)\\
   g(x) \sim v(x)\\
  \end{array}\right\}
\mbox{Alors } f(x)\times g(x) \sim u(x)\times v(x)\]
De plus : 
$$\dfrac{f(x)}{g(x)}\sim \dfrac{u(x)}{v(x)}$$
Soit $\alpha$ un réel :
$$(f(x))^{\alpha} \sim (u(x))^{\alpha}$$
$$\mid f(x) \mid \sim \mid u(x)\mid$$
\subsubsection{Changement de variable}
Le changement de variable dans un équivalent est autorisé, mais pas la composé ne l'est pas.
\begin{prop}
 Si $f(x) \sim g(x)$, alors $\lim_{a}f$ et $\lim_{b}f$ ont même nature et si elles existent sont égales
\end{prop}
\subsection{Formule de Taylor avec reste de Young}
\subsubsection{Préliminaire}
\begin{theo}
Si $\varphi$ est une fonction dérivable sur $V_0$, un voisinage de 0, et si
\[\left.\begin{array}{l}
   \varphi(0)=0\\
   \exists n \in N \mbox{ telque si } x \rightarrow 0, \varphi'(x)=O(x^n)
  \end{array}\right\}
\mbox{Alors }\varphi(x)=O(x^{n+1})\]
Si $\varphi$ est dérivable sur $V_0$ et si :
\[\left.\begin{array}{l}
   \varphi(0)=0\\
    \mbox{si } x \rightarrow 0, \varphi'(x)=o(x^n)
  \end{array}\right\}
\mbox{Alors si }x\rightarrow0~ \varphi(x)=o(x^{n+1})\]
\end{theo}
\subsubsection{Formule de Taylor}
\begin{de}
 Si f est de classe $C^n$ sur $V_0$, alors $\forall x \in V_0$:
$$f(x) = f(0)+f'(0)x+...+\dfrac{x^n}{n!}f^{(n)}(0)+o(x^n)$$
Si f est de classe $C^n$ sur un voisinage de a, $V_a$, $a\in \mathbb{R}$, alors $\forall x \in V_a$:
$$f(x) = f(0)+...+\dfrac{(x-a)^n}{n!}f^{(n)}(a)+o((x-a)^n)$$
\end{de}
 % Relue
\chapter{Développements limités}
\section{Notation de Landau}
\begin{de}
 Si, lorsque x $\mapsto 0$, $f(x)\ll g(x)$, on note : 
$$f(x)=o(g(x))$$
Soit n,p entiers :
\begin{itemize}
 \item[$\rightarrow$] $x^n \times o(x^p) = o(x^{n+p})$
 \item[$\rightarrow$] $o(x^n)\times o(x^p) = o(x^{n+p})$
 \item[$\rightarrow$] $o(x^n)+o(x^p) = o(x^{inf(n,p)})$
 \item[$\rightarrow$] Si A est un réel fixé :
$$A \times o(x^n) = o(x^n)$$
\end{itemize}
\end{de}
\section{Définitions}
\begin{de}
Soit f une fonction définie au voisinage de O. \\
On dit que f possède un développement limité d'ordre n si il $\exists(a_0,...a_n) \in \mathbb{R}^n$ telque :
$$f(x)-(a_0+a_1x+...+a_nx^n) \ll x^n$$
donc, au voisinage de 0, $f(x)=a_0+a_1x+...+a_nx^n+o(x^n)$. \\
Il y a unicité du développement limité.\\
On peut faire une combinaisons linéaire de développement limité.
\end{de}
\begin{de}
 On appelle partie principale du développement limité la fonction polynomiale suivant : 
$$x\mapsto a_0+a_1x+...+a_nx^n$$
\end{de}
\section{Équivalence et développement limité}
\begin{de}
Si f possède un développement limité d'ordre n au voisinage de 0, si $\exists k$ telque $a_k\neq 0$, notons p l'indice du $1^{er}$
terme non nuls, alors, au voisinage de 0 :
$$f(x)\sim a_px^p$$
\end{de}
\section{Régularité au voisinage de 0 et développement limité}
\begin{de}
Au voisinage de 0 :
\begin{itemize}
 \item[$\rightarrow$] f est de classe $C^0 \Leftrightarrow \exists$ un développement limité d'ordre O
 \item[$\rightarrow$] f est dérivable $\Leftrightarrow \exists$ un développement limité d'ordre 1
  \[\left.\begin{array}{l}
 \mbox{f est de classe } C^1 \Rightarrow \exists \mbox{ un développement limité d'ordre 1}\\
 \mbox{f est de classe } C^2 \Rightarrow \exists \mbox{ un développement limité d'ordre 2}
  \end{array}\right\}
\mbox{Formule de Taylor-Young}\]
\end{itemize}
\end{de}
\section{Développement limités usuels}
\begin{itemize}
 \item[$\rightarrow$]$(1+x)^{\alpha} = 1 + \alpha x+\dfrac{\alpha (\alpha - 1)}{2!}x^2+o(x^2)$
 \item[$\rightarrow$]$cos(x) = 1 - \dfrac{x^2}{2!}+\dfrac{x^4}{4!}-\dfrac{x^6}{6!}+o(x^6)$
 \item[$\rightarrow$]$sin(x) = 1 - \dfrac{x^3}{3!}+\dfrac{x^5}{5!}-\dfrac{x^7}{7!}+o(x^7)$
 \item[$\rightarrow$]$e^x = 1 +x + \dfrac{x^2}{2!}+\dfrac{x^3}{3!}+\dfrac{x^4}{4!}+o(x^4)$
 \item[$\rightarrow$]$\dfrac{1}{1-x} = 1+x+x^2+...+x^n+o(x^n)$
 \item[$\rightarrow$]$ch(x) = 1 + \dfrac{x^2}{2!}+\dfrac{x^4}{4!}+...+\dfrac{x^{2n}}{2n!}+o(x^{2n+1})$
 \item[$\rightarrow$]$sh(x) = 1 + \dfrac{x^3}{3!}+\dfrac{x^5}{5!}+...+\dfrac{x^{2n+1}}{(2n+1)!}+o(x^{2n+1})$
 \item[$\rightarrow$] $ln(1+x) = x - \dfrac{x^2}{2} + \dfrac{x^3}{3} +o(x^3)$
\end{itemize}
\section{Dérivation et Intégration}
\begin{de}
Pour obtenir le développement limité de f'(x), on dérive terme à terme le développement limité de f(x). \\
Pour obtenir le développement limité de F(x), une primitive de f(x), on intègre terme à terme :\\
Si $f(x)=a_0+...+a_nx^n+o(x^n)$, alors :
$$F(x)=F(0)+a_0x+...+\dfrac{a_n}{n+1}x^{n+1}+o(x^{n+1})$$
\end{de}
\section{Développement limité au voisinage d'un réel a}
\begin{de}
Soit f fonction défini au voisinage de a. On dit que f possède, au voisinage de a, un développement limité d'ordre n si $\exists P \in \mathbb{R}_n[X]$ telque :
$$f(x) = \lambda_0+\lambda_1(x-a)+....+\lambda_n(x-a)^n+o((x-a)^n)$$
De plus : \\
(f est dérivable en a)$\Leftrightarrow$ (f est défini en a, \\et f possède au voisinage de a un développement limité d'ordre 1)
\end{de}
\subsection{Tangente}
\begin{prop}
 Si, au voisinage de a, f(x) = $\lambda_0+\lambda_1(x-a)+o((x-a))$, alors :
$$y = \lambda_0+\lambda_1(x-a)$$
est tangent à la courbe en a. Le terme suivant non nul détermine la position relative de la tangente par rapport à la courbe.
\end{prop}
\section{Développement limité généralisé}
Soit $\alpha \in \mathbb{R}$
\begin{de}
Si au voisinage de 0, on peut écrire :
$$f(x)=\lambda_0x^{\alpha}+...+\lambda_nx^{\alpha+n}+o(x^{\alpha+n})$$
Alors ceci constitue un développement limité généralisé de f en 0. 
\end{de}
\begin{de}
Si $x\mapsto +\infty$, avec f défini au voisinage de $+\infty$. Si on peut écrire : 
$$f(x)=\lambda_0x^{\alpha}+....+\lambda_nx^{\alpha-n}+o(x^{\alpha-n})$$
\end{de}
 % Relue
\part{Les suites}
\chapter{Suite num\'erique - G\'eneralit\'e}
\section{Propriétés}
\subsection{Opérations}
Soit (a),(b),(c) trois suites. On peut effectuer trois types d'opérations sur les suites :
\begin{itemize}
 \item[$\rightarrow$] Une somme : 
$$((a)+(b) = (c)) \Leftrightarrow (\forall n \in N~ a+b =c)$$
 \item[$\rightarrow$] Un produit : 
$$((a).(b) = (c)) \Leftrightarrow (\forall n \in N~ a.b =c)$$
 \item[$\rightarrow$] Un produit par un scalaire : 
$$((a) = \lambda(c)) \Leftrightarrow (\forall n \in N~ a = \lambda c)$$
\end{itemize}
\section{Suites particulière}
\subsection{Suite arithmétiques}
Soit ($u_n$) une suite arithmétiques :
$$\sum_{k=0}^n u_k = (n+1).\dfrac{u_0+u_n}{2}$$
\subsection{Suite géométrique}
Soit ($u_n$) une suite géométrique, de raison q:
$$\sum_{k=0}^n u_k = u_0.\dfrac{1-q^{n+1}}{1-q}$$
\section{Suites vérifiant une relation de récurrence linéaire à coefficiants constants}
Soit ($u_n$) une suite vérifiant la récurrence :
$$u_{n+2} = a.u_{n+1} + b.u_{n}$$
Alors, on obtient l'équation caractéristique, en simplifiant par $r^n$ :
$$r^2 = ar+b$$
Donc :
$$\exists(A,B)\in \Re^2~ tq~ (u_n) = A(r_1)^n+B(r_2)^n$$
 % Relue
\chapter{Convergence des suite numériques réelles}
\section{Suites convergentes}
\begin{de}
Soit $(u_n)_{n \geq 0}$ une suite de nombre réels.\\
On dit que $(u_n)_{n \geq 0}$ converge vers 0 si :
$$\forall \varepsilon > 0, \exists n_0 \in \mathbb{N}~ tq~ \forall n \geq n_0~ |u_n|<\varepsilon$$
\end{de}
\begin{de}
Soit $l \in \mathbb{R}$. La suite $u_n$ converge vers l si :
$$\forall \varepsilon > 0, \exists n_0 \in \mathbb{N}~ tq~ \forall n \geq n_0~ |u_n -l|<\varepsilon$$
\end{de}
Les suites convergentes possède les propriétés suivantes :\\
\begin{itemize}
 \item[$\rightarrow$] Une suite constante est convergente\\
 \item[$\rightarrow$] Une suite géométrique de raison a avec |a|<1 converge vers 0\\
 \item[$\rightarrow$] La suite $(\dfrac{1}{n})_{n\geq 1}$ converge vers 0\\
 \item[$\rightarrow$] Si ($u_n)$ converge vers une limite l, elle est unique.\\
 \item[$\rightarrow$] Un suite ($u_n$) converge vers 0 si et seulement si $(|u_n|)$ converge vers 0\\
 \item[$\rightarrow$] Une suite convergente est bornée\\
 \item[$\rightarrow$] Si ($a_n$) converge vers 0 et $\exists n_0 \in \mathbb{N}~ tq~ \forall n \geq n_0~ |u_n|\leq |a_n|$, alors $(u_n)$ converge vers 0\\
\end{itemize}
\subsection{Caractérisation de la borne supérieur}
On peut caractériser la borne supérieur d'un ensemble non vide et majorée à l'aide d'une suite.\\
Soit A une partie de $\mathbb{R}$
$$(\mbox{ M est la borne supérieur de A }) \Leftrightarrow \left\{\begin{array}{l}
    \mbox{M est un majorant de A}\\
    \exists(a_n)~ tq~ \forall n \in \mathbb{N} a_n\in A~ et~ a_n\mbox{ converge vers M}
  \end{array}\right.$$
\subsection{Caractérisation d'une partie dense}
Soit A une partie de $\mathbb{R}$
$$(\mbox{ A est dense dans } \mathbb{R}) \Leftrightarrow (\forall x \in \mathbb{R}, \exists(u_n) \in \mathbb{R}^n, \forall n ~u_n \in A~ et~ a_n\mbox{ converge vers x})$$
\subsection{Opération sur les suites convergentes}
Nous avons les propriétés suivantes :\\
\begin{itemize}
 \item[$\rightarrow$] La somme de deux suites convergente est convergente, et la limite de la somme est la somme des limites.\\
 \item[$\rightarrow$] Le produit par une constante d'une suite convergente est convergente\\
 \item[$\rightarrow$] Le produit d'une suite bornée par une suite convergente de limite nul est une suite convergente de limite nul.\\
 \item[$\rightarrow$] Le produit d'une suite convergente par une suite convergente de limite nul est une suite convergente de limite nul.\\
 \item[$\rightarrow$] Le produit de deux suite convergentes est une suite convergente\\
 \item[$\rightarrow$] Si $(u_n)$ est une suite convergente de terme tous non nuls, si $l \neq 0$, alors $\dfrac{1}{u_n}$ converge vers $\dfrac{1}{l}$\\
 \item[$\rightarrow$] Si $(u_n)$ est une suite convergente de limite l, alors $(|u_n|)$ converge vers |l|\\
\end{itemize}
\subsection{Lien entre le signe de la limite et le signe des termes de la suite}
\begin{itemize}
 \item[$\rightarrow$] Si l>0, alors $\exists n_0 \in \mathbb{N}$, tq $\forall n \geq n_0$, $u_n > 0$\\
 \item[$\rightarrow$] Si l<0, alors $\exists n_0 \in \mathbb{N}$, tq $\forall n \geq n_0$, $u_n < 0$\\
 \item[$\rightarrow$] Si $\exists n_0 \in \mathbb{N} $ tq $\forall n \geq n_0$, $u_n < 0$, alors $l \leq 0$\\
 \item[$\rightarrow$]Si $\exists n_0 \in \mathbb{N} $ tq $\forall n \geq n_0$, $u_n > 0$, alors $l \geq 0$\\
\end{itemize}
De ces correspondances, on détermine les comparaisons entre deux suites convergentes.
\subsection{Théorème d'encadrement}
Si : 
$$\left\{\begin{array}{l}
    (u_n) \mbox{ converge vers l}\\
    (v_n) \mbox{ converge vers l}\\
    \exists n_0 \in \mathbb{N},~ \forall n\geq n_0,~ u_n\leq x_n \leq v_n\\
  \end{array}\right.$$
Alors $x_n$ converge vers l.
\subsection{Suite extraites}
\begin{prop}
 Si $(u_n)$ converge, alors toutes ses suites extraites converge vers la même limite.
\end{prop}
\begin{prop}
Si :
$$\left\{\begin{array}{l}
    (u_{\varphi_{(n)}})~ et~ (u_{\psi_{(n)}})\mbox{ converge vers la même limite}\\
    \{\varphi_{(n)}~ /~ n\in \mathbb{N}\} \cup   \{\psi_{(n)}~ /~ n\in \mathbb{N}\} = \mathbb{N}\\
  \end{array}\right.$$
Alors la suite $(u_n)$ converge
\end{prop}

\section{Suites divergentes}
\subsection{Caractéristation des suites divergentes}
Si :
$$\left\{\begin{array}{l}
    (u_{\varphi_{(n)}})~ et~ (u_{\psi_{(n)}})\mbox{ converge vers des limites différentes}\\
    \{\varphi_{(n)}~ /~ n\in \mathbb{N}\} \cup   \{\psi_{(n)}~ /~ n\in \mathbb{N}\} = \mathbb{N}\\
  \end{array}\right.$$
Alors la suite $(u_n)$ diverge

\subsection{Suites qui diverge vers $\pm\infty$}
\begin{de}
Soit $(u_n)_{n \geq 0}$ une suite de nombre réels.\\
On dit que $(u_n)_{n \geq 0}$ diverge vers $+\infty$ si :
$$\forall A \in \mathbb{R}^+, \exists n_0 \in \mathbb{N}~ tq~ \forall n \geq n_0~ u_n>A$$
\end{de}
\begin{de}
Soit $(u_n)_{n \geq 0}$ une suite de nombre réels.\\
On dit que $(u_n)_{n \geq 0}$ diverge vers $-\infty$ si :
$$\forall B \in \mathbb{R}^-, \exists n_0 \in \mathbb{N}~ tq~ \forall n \geq n_0~ u_n<B$$
\end{de}
\subsubsection{Propriétés}
\begin{itemize}
 \item[$\rightarrow$] $((u_n)$ diverge vers $-\infty$) $\Leftrightarrow$ (($-u_n)$ diverge vers $+\infty$)
 \item[$\rightarrow$] La somme d'une suite bornée et d'une suite qui diverge vers $+\infty$ diverge vers $+\infty$
 \item[$\rightarrow$] L'inverse d'une suite qui tend vers $+\infty$ converge vers 0
\end{itemize}
\subsection{Théorème de minoration}
\begin{theo}
 Si ($u_n$) et $(v_n)$ sont deux suites telque :
$$\left\{\begin{array}{l}
    (u_n) \mbox{ diverge vers } +\infty \\
    \exists n_0~ tq~ \forall n \geq n_0~ u_n \geq x_n\\
  \end{array}\right.$$
Alors $(x_n)$ diverge vers $+\infty$
\end{theo}
\section{Suite monotone et convergente}
\begin{theo}
Soit $(u_n)$ une suite croissante.\\
Si elle est majorée, alors elle converge. Sinon, elle diverge vers $+\infty$
\end{theo}
\begin{theo}
Soit $(u_n)$ une suite décroissante.\\
Si elle est minorée, alors elle converge. Sinon, elle diverge vers $-\infty$
\end{theo}
\subsection{Suites adjacentes}
\begin{de}
Soient $(u_n)$ et $(v_n)$ deux suites. Elle sont dites adjacentes si : 
\begin{enumerate}
 \item $(u_n)$ croissante
 \item $(v_n)$ décroissante
 \item $(u_n - v_n)$ converge vers 0
\end{enumerate}
\end{de}
\begin{prop}
Si $(u_n)$ et $(v_n)$ sont adjacentes, alors elles convergent vers la même limite, notée l et : 
$$\forall n \in \mathbb{N},~ u_n \leq l \leq v_n $$
\end{prop}
\subsection{Segments emboités}
\begin{de}
On considère une suite de segments. On dit que la suite est emboitée si :
$$\forall n \in \mathbb{N}~ \left[a_{n+1},b_{n+1}\right] C \left[a_n,b_n\right]  $$
\end{de}
\begin{prop}
Nous avons les propriétés suivantes : 
\begin{itemize}
 \item[$\rightarrow$] L'intersection de tous les intervalles d'une suites d'intervalle emboitée est non vide
 \item[$\rightarrow$] Si la longeur de l'intervalle tend vers 0, alors l'intersection est un singleton
\end{itemize}
\end{prop}
\subsection{Théorème de Bolzano-Weierstrass}
\begin{theo}
De toutes suites réelle bornée, on peut extraire une suite convergente.
\end{theo}

 % Relue
\chapter{Suite à valeur complexe}
\section{Convergence}
\begin{de}
 Soit ($z_n$) une suite à valeur complexe. On dit que cette suite converge vers $\lambda$ si et seulement si :
$$\forall \varepsilon > 0, \exists n_0 \in \mathbb{N}~ tq~ \forall n \geq n_0~ |u_n -\lambda|<\varepsilon$$
\end{de}
\section{Partie réeles, partie imaginaire}
\begin{prop}
Si Re($z_n$) converge vers a et Im($z_n$) converge vers b, alors $(z_n)$ converge vers a+ib.
\end{prop}
\section{Suites des modules et suites des arguments}
Soit ($z_n$) la suite défini par :
$$\forall n \in \mathbb{N}~ z_n = \rho_ne^{i\theta_n}$$
\begin{prop}
Si :
$$\left\{\begin{array}{l}
    (\rho_n) \mbox{ converge vers a}\\
    (\theta_n) \mbox{ converge vers b}
  \end{array}\right.$$
alors $(z_n)$ converge vers $ae^{ib}$
\end{prop}
Mais si la suite des arguments ne converge pas, la suite $(z_n)$ peut quand meme converger.
\section{Opération}
\subsection{Somme de deux suites convergente}
\begin{prop}
 Soient $(z_n)$ et $(z'_n)$ deux suites convergentes de limite $\lambda$ et $\lambda'$.\\
On obtient que $(z_n+z'_n)$ converge vers $\lambda+\lambda'$
\end{prop}

 % Relue
\chapter{Etude des suites}
\section{Suite complexe}
Soit $(z_n)$ une suite de complexe
\begin{prop}
$$((z_n)\mbox{ converge vers }\lambda ) \Leftrightarrow (|z_n - \lambda| \mbox{ converge vers 0 })$$
\end{prop}
\begin{prop}
$$((z_n)\mbox{ converge vers }\lambda ) \Leftrightarrow (Re(z_n) \mbox{ converge vers Re(}\lambda)~ et~ Im(z_n) \mbox{ converge vers Im(}\lambda))$$
\end{prop}
\begin{prop}
Si $(z_n)$ converge vers $\lambda$, alors $(|z_n|)$ converge vers |$\lambda$|.\\
Mais aucune information sur le comportement de l'argument.
\end{prop}
\begin{prop}
Si le module de $z_n$ converge vers R et que l'argument de $z_n$ converge vers $\alpha$, alors $(z_n)$ converge vers R$e^{i\alpha}$
\end{prop}
\begin{prop}
Soit $(z_n)$ suite complexe défini par :
$$\forall n \in \mathbb{N}~ z_n = a^n$$
Avec $a \in C$.
Si :
\begin{itemize}
 \item[$\rightarrow$] a=0, $(z_n)$ est une suite constante\\
 \item[$\rightarrow$] a $\in$ ]0,1[, $(z_n)$ converge vers 0\\
 \item[$\rightarrow$] |a| > 1, $(z_n)$ diverge vers $+\infty$\\
 \item[$\rightarrow$] |a| = 1.\\ \begin{itemize}
 \item[$\rightarrow$] Si a $\neq$ 1, alors la suite diverge\\
 \item[$\rightarrow$] Si a = 1, $(z_n)$ est une suite constante\\
\end{itemize}
\end{itemize}
\end{prop}
\section{Suites définies par récurrence}
\begin{de}
Soit f une fonction de $\mathbb{R}$ dans $\mathbb{R}$, définie sur $D_f$, et $(u_n)$ une suite définie telque :
$$u_0 \in \mathbb{R}$$
$$\forall n,~ u_{n+1} = f(u_n)$$
\end{de}
\subsection{Existance de la suite}
\begin{prop}
Soit $(u_n)$ une suite de réel, défini par récurrence à l'aide de la fonction f.
$$(\forall n,~ u_n \mbox{ existe}) \Leftarrow (u_0 \in D_f~ et~ D_f \mbox{ est stable par f})$$
$$(\forall n,~ u_n \mbox{ existe}) \Leftarrow (u_0 \in D_f~ et~ f(D_f) C D_f) $$
\end{prop}
\subsection{Sens de variation}
\begin{prop}
Soit $(u_n)$ une suite de réel, défini par récurrence à l'aide de la fonction f.
$$( (u_n) \mbox{ est croissante }) \Leftrightarrow (\forall n \in \mathbb{N},~ u_{n+1}\geq u_n)$$
$$( (u_n) \mbox{ est croissante }) \Leftrightarrow (\forall n \in \mathbb{N},~ f(u_n)\geq u_n)$$
En pratique, si $\forall x \in I,~ f(x) \geq x$ et si $\forall n \in \mathbb{N},~ u_n\in I$, alors ($u_n$) est croissante. 
\end{prop}
\subsection{Limite éventuelle}
Si $(u_n)$ converge vers l et que f est continue en l, alors l = f(l)
\section{Règle de d'Alembert}
Soit ($u_n$) une suite de réels positifs.\\
Supposons que :
$$\lim_{n \rightarrow \infty} \dfrac{u_{n+1}}{u_n} = a$$
\begin{itemize}
 \item[$\rightarrow$] Si a $\in \left[0,1\right[$, ($u_n$) converge vers 0
 \item[$\rightarrow$] Si a>1, $(u_n)$ diverge vers $+\infty$
\end{itemize}
\section{Comparaison des suites}
\subsection{Définitions}
\begin{de}
Soit $(u_n)$ et $(v_n)$ deux suites à valeur réelle.\\
On dit que $(v_n)$ domine $(u_n)$ si :
$$\exists A \in \mathbb{R}^{+}~ tq~ \forall n \in \mathbb{N}~ |u_n| \leq A.|v_n|$$
On note $u_n$=O($v_n$)
\end{de}
\begin{de}
On dit que $(u_n)$ est négligable devant $(v_n)$ ou que $(v_n)$ est préponderant devant $(u_n)$ si : 
$$\forall \varepsilon > 0~ \exists n_0 \in \mathbb{N}~ tq~ \forall n \geq n_0~ |u_n|\leq \varepsilon|v_n|$$
On note $u_n \ll v_n$ ou $u_n = o(v_n)$
\end{de}
\begin{de}
On dit que $(u_n)$ est équivalent à $(v_n)$ si ($u_n - v_n$) est négligable devant $(v_n)$
\end{de}
\subsection{Comparaison des suites de référence}
\begin{prop}
Soit $(c_n)$ une suite telleque :
$$\lim_{n \rightarrow +\infty}|c_n| = +\infty$$
Si $(a_n)$ converge vers 0, si $(b_n)$ converge vers l, l$\neq 0$, alors :
$$a_n \ll b_n \ll c_n$$
\end{prop}
\subsubsection{Comparaison des suites qui divergent vers $+\infty$}
Soit A > 1, $\alpha > 0$ :
$$ln(n) \ll n^{\alpha} \ll A^n \ll n! \ll n^n$$
\subsubsection{Comparaion des suites qui converge vers 0}
Soit B < 1, $\beta < 0$, alors :
$$0 \ll B^n \ll n^{\beta} \ll \dfrac{1}{ln(n)}$$
\subsubsection{Comparaion des suites convergente de limite non nul}
Soient l,l' deux réels non nuls.\\
Soient $(u_n)$ et ($v_n$) deux suites qui convergent respectivement vers l et l'.\\
Alors :
$$u_n \sim \dfrac{l}{l'}v_n$$
\section{Règles d'utilisation des équivalents et négligabilité}
\begin{prop}
Soient $(u_n)$, $(v_n)$ et $(w_n)$ trois suites.
$$u_n \ll v_n~ et~ v_n \ll w_n \Rightarrow u_n \ll w_n$$
$$u_n \sim v_n~ et~ v_n \sim w_n \Rightarrow u_n \sim w_n$$
$$u_n \sim v_n~ et~ v_n \ll w_n \Rightarrow u_n \ll w_n$$
$$u_n \ll v_n~ et~ v_n \sim w_n \Rightarrow u_n \ll w_n$$
\end{prop}
\begin{prop}
Soient $(u_n)$, $(v_n)$ deux suites et $(a_n),(b_n)$ deux autres suites.\\
Si :
$$(u_n)\sim(v_n)~ et ~ (a_n)\sim(b_n)$$
alors :
$$(u_n)(a_n) \sim (v_n)(b_n)$$
Ceci n'est pas vrai dans le cas de l'addition.
\end{prop}
\begin{prop}
Soient $(u_n)$, $(v_n)$ et $(a_n)$ trois suites et $\lambda,\mu$ deux réels de somme non nuls.\\
Si :
$$(u_n)\sim\lambda(a_n)~ et ~ (v_n)\sim\lambda(a_n)$$
Alors :
$$u_n + v_n \sim (\lambda+\mu)a_n$$
Si la somme des deux réels est nul, nous n'avons aucun résultats.
\end{prop}
\begin{prop}
Si $u_n \sim v_n$, alors $u_n^{\alpha} \sim v_n^{\alpha}$
\end{prop}
\subsubsection{Propriété des équivalents}
\begin{prop}
Soient $u_n$ et $v_n$ deux suites telque $u_n \sim v_n$.
\begin{itemize}
 \item[$\rightarrow$] Si $u_n$ diverge, alors $v_n$ diverge
\item[$\rightarrow$] Si $u_n$ converge, alors $v_n$ converge vers la même limite
 \item[$\rightarrow$] Si la limite de $u_n$ en l'infini est l'infini, alors la limite de $v_n$ en l'infini est aussi l'infini
\end{itemize}

\end{prop}

 % Relue
\part{Arcs Param\'etr\'e}
\chapter{Arcs Paramétrés et Arcs Polaire}
\section{Étude locale d'un arc}
\begin{de}
Soit $\tau$ un arc paramétré défini par (F,I), avec I un intervalle et F une fonction : 
$$F : I \rightarrow \mathbb{R}^2$$
$$t \mapsto M(t)$$
avec : M(t) = $\begin{pmatrix}
  x(t)\\
  y(t)\\
\end{pmatrix}$
\end{de}
\begin{prop}
Supposons que x et y soient de classe $C^n$ sur I, alors on dit que ($\tau$) est de classe $C^n$
\end{prop}
\subsection{Point Régulier} 
Si $\dfrac{d\overrightarrow{M}}{dt}(t_0) \neq \overrightarrow{0}$, alors  $\dfrac{d\overrightarrow{M}}{dt}(t_0)$ est un vecteur directeur de la tangente au support de l'arc en M($t_0$). On dit alors que M($t_0$) est un point régulier de l'arc.
\subsection{Point Singulier} Si $\dfrac{d\overrightarrow{M}}{dt}(t_0) = \overrightarrow{0}$, alors M($t_0$) est un point singulier.\\
Par application de la formule de Taylor, on obtient les deux entiers caractéristiques suivants.
\subsubsection{Premier entier caractéristique de $(\tau)$ en M($t_0$)}
\begin{de}
Notons p, si il existe : 
$$p = Min\{k / \dfrac{d^k\overrightarrow{M}}{dt^k}(t_0) \neq \overrightarrow{O}\}$$
alors $\dfrac{d^p\overrightarrow{M}}{dt^p}(t_0)$ est tangent à l'arc en M($t_0$)
\end{de}
\subsubsection{Deuxième entier caractéristique de ($\tau$) en M($t_0$)}
Notons q, si il existe :
$$q = Min\{k / \dfrac{d^k\overrightarrow{M}}{dt^k}(t_0)~ et~ \dfrac{d^p\overrightarrow{M}}{dt^p}(t_0) \mbox{ soient non colinéaire} \}$$
\subsubsection{Coordonnée de M(t) dans un repère particulier}
Soit R le repère défini par : $(M(t_0),\dfrac{d^p\overrightarrow{M}}{dt^p}(t_0),\dfrac{d^q\overrightarrow{M}}{dt^q}(t_0))$\\
Dans ce repère, on obtient les coordonnées suivantes pour M(t): 
\[\left\{\begin{array}{l}
  X(t) \sim \dfrac{(t-t_0)^p}{p!}\\
  Y(t) \sim \dfrac{(t-t_0)^q}{q!}\\
  \end{array}\right.\]
On obtient donc :
\begin{itemize}
 \item[$\rightarrow$]p paire, q impaire : Point de rebroussement du première ordre
 \item[$\rightarrow$]p paire, q paire : Point de rebroussement du deuxième ordre
 \item[$\rightarrow$]p impaire, q paire : Point ordinaire
 \item[$\rightarrow$]p impaire, q impaire : Point d'inflexion
\end{itemize}
\section{Etude métrique des arc paramétré}
\subsection{Longeur d'un arc}
Soit $(\tau)$ un arc de classe $C^1$ sur I = [a,b], avec a < b.\\ Soit l($\widehat{M(a)M(b)}$) la longeur de l'arc $(\tau)$ reliant M(a) à M(b). On obtient : 
$$l(\widehat{M(a)M(b)}) = \int_a^b ||\dfrac{d\overrightarrow{M}}{dt}(t)||dt$$
\subsection{Abscisse curviligne}
Soit $\tau$ un arc de classe $C^1$ sur un intervalle I.\\
On défini une abscisse curviligne avec :
\begin{enumerate}[1-]
 \item Un point de l'arc, appelé origine.
 \item Une orientation sur l'axe : 
\begin{itemize}
 \item[$\rightarrow$] Le sens des t croissants
 \item[$\rightarrow$] Ou le sens des t décroissants
\end{itemize}
\end{enumerate}
Soit M(t$_0$) l'origine de l'abcisse, soit s(t) l'abscisse du point M(t).
$$s(t) = \varepsilon \int_{t_0}^t ||\dfrac{d\overrightarrow{M}}{dt}(u)||du$$
avec $\varepsilon = \pm 1$ selon l'orientation de l'axe.\\
On obtient aussi : 
$$\dfrac{ds}{dt} = ||\dfrac{d\overrightarrow{M}}{dt}||.\varepsilon$$
s est un paramétrage du support de l'arc. De plus, on obtient : 
$$||\dfrac{d\overrightarrow{M}}{ds}|| = 1$$
\subsubsection{Repère de Frenet en M(t)}
On note $\overrightarrow{T} = \dfrac{d\overrightarrow{M}}{ds}$ le premier vecteur de Frenet en M($t)$. On le calcul en utilisant le faite que : 
$$\dfrac{d\overrightarrow{M}}{ds} = \dfrac{\varepsilon}{||\frac{d\overrightarrow{M}}{dt}||}.\dfrac{d\overrightarrow{M}}{dt}$$
On note $\overrightarrow{N}$ l'unique vecteur vérifiant que (M(t),$\overrightarrow{T},\overrightarrow{N}$) soit un repère orthonormée directe, appelé repère de Frenet en M(t).\\
Soit $\varphi =(\overrightarrow{i},\overrightarrow{T}) [2\pi]$, avec $\overrightarrow{i}$ vecteur horizontal passant par M(t).\\
Alors : 
$$\overrightarrow{T} = \begin{pmatrix}
  cos(\varphi)\\
  sin(\varphi)\\
\end{pmatrix}$$
$$\overrightarrow{N} = \begin{pmatrix}
  -sin(\varphi)\\
  cos(\varphi)\\
\end{pmatrix}$$
\subsection{Courbure d'un arc en un point}
\begin{de}
On défini le rayon de courbure en M(t), notée R, par : 
$$R = \dfrac{ds}{d\varphi}$$
\end{de}
\begin{de}
La courbure en M(t), notée $\gamma$, est défini par : 
$$\gamma = \dfrac{1}{R} = \dfrac{d\varphi}{ds} = \dfrac{\frac{d\varphi}{dt}}{\frac{ds}{dt}}$$
\end{de}
\subsection{Formules de Frenet}
Soit $\overrightarrow{T}\begin{pmatrix}
  cos(\varphi)\\
  sin(\varphi)\\
\end{pmatrix}$ et $\overrightarrow{N}\begin{pmatrix}
  -sin(\varphi)\\
  cos(\varphi)\\
\end{pmatrix}$ les deux vecteurs de la base de Frenet.\\
Sachant que : 
$$\dfrac{d\overrightarrow{T}}{d\varphi} = \overrightarrow{N}$$
On obtient : 
$$\dfrac{d\overrightarrow{T}}{ds} = \gamma \overrightarrow{N}$$
De meme, on obtient que : 
$$\dfrac{d\overrightarrow{N}}{ds} = -\gamma \overrightarrow{T}$$
\subsubsection{Lien entre courbure, vitesse et accélération}
Notons $\overrightarrow{V} = \dfrac{d\overrightarrow{M}}{dt}$ et v = ||$\overrightarrow{V}$||.\\
On obtient : 
$$\overrightarrow{V} = \varepsilon.v.\overrightarrow{T}$$.
Notons $\overrightarrow{a} = \dfrac{d\overrightarrow{V}}{dt}$. Alors : 
$$\overrightarrow{a} = \varepsilon.\dfrac{dv}{dt}.\overrightarrow{T}+v^2.\gamma.\overrightarrow{N}$$
De cette expression, on en déduit que : 
$$\gamma = \varepsilon.\dfrac{Det(\overrightarrow{V},\overrightarrow{a})}{v^3}$$
\section{Plan d'étude d'un arc paramétré}
\begin{enumerate}[1-]
 \item Domaine de définition\\
 \item Réduction du domaine d'étude :\\ 
\begin{itemize} 
\item[{$\rightarrow$}] Periodicité\\
\item[{$\rightarrow$}] Symétrie\\
\item[{$\rightarrow$}] Partié\\
\end{itemize}
\item Dérivablité : Faire un double tableau de variation (un pour x, un pour y)\\
\item Tangentes\\
\begin{itemize} 
\item[{$\rightarrow$}] \underline{En un point régulier} : Le vecteur de coordonée $(x'(t_0),y'(t_0))$ est tangent en $t_0$.\\
\item[{$\rightarrow$}] \underline{En un point singulier} :\\
$$\underset{t \mapsto t_0}\lim \dfrac{y'(t)}{x'(t)}$$
Ou on peut utiliser la méthode des entiers caractéristiques
\item[{$\rightarrow$}] \underline{En un point limite}\\
$$\underset{t \mapsto t_0}\lim \dfrac{y(t)-\underset{t_0}\lim y}{x(t)-\underset{t_0}\lim x}$$
\end{itemize}
\item Branches infinies : Si $\underset{t_0}\lim x= +\infty$ et $\underset{t_0}\lim y= +\infty$\\
\begin{itemize}
 \item[$\rightarrow$] $\underset{t_0}\lim\dfrac{y}{x}$\\
\begin{itemize}
 \item[$\rightarrow$] $\infty$ : \underline{Branche de direction Oy}\\
 \item[$\rightarrow$] $a \in \mathbb{R}$ :\\
\begin{itemize}
 \item[$\rightarrow$] $\underset{t_0}\lim y-ax$ :\\
\begin{itemize}
 \item[$\rightarrow$] $b \in \mathbb{R}$ : \underline{y = ax+b asymptote}\\
 \item[$\rightarrow$] $\infty~ ou~ \emptyset$ : \underline{Branche de direction ax}\\
\end{itemize}
 \item[$\rightarrow$] 0 : \underline{Branche de direction Ox}\\
 \item[$\rightarrow$] $\emptyset$ : \underline{Aucune méthode}\\
\end{itemize}
\end{itemize}
\end{itemize}
\item Concavité :\\
\begin{itemize}
 \item[$\rightarrow$] Etude du signe de $Det(\overrightarrow{v};\overrightarrow{\Gamma})$\\
\item[$\rightarrow$] L'angle $(\overrightarrow{v};\overrightarrow{\Gamma})$ donne la position de la tangente\\
\item[$\rightarrow$] Les points d'inflexion sont les points de changements de concavité\\
\end{itemize}
\item Point double : On résoud le système suivant, d'inconnu (t,t'), avec $t \neq t'$ :
\[\left\{\begin{array}{l}
   x(t) = x(t')\\
   y(t) = y(t') \\
  \end{array}\right.\]
\end{enumerate}
\section{Arcs polaire}
\subsection{Liens polaire-cartésien}
Soit ($\rho$,$\theta$) les coordonnée de M, telque :
$$\overrightarrow{OM} = \rho\overrightarrow{u}(\theta)$$
On obtient les coordonées cartérisien de M avec : 
\[\left\{\begin{array}{l}
   x = \rho cos(\theta)\\
   y = \rho sin(\theta)\\
  \end{array}\right.\]
On a donc : 
$$\rho = \pm \sqrt{x^2 + y^2}$$
\subsection{Equivalence et symétrie}
Soit $M(\rho,\theta) = M(-\rho,\theta + \pi)$ (cette égalité est vérifie pour tout point du plan) :
\begin{itemize}
 \item[$\rightarrow$] $M_1$ symétrique de M par rapport à (Ox) si :
$$M_1(\rho, -\theta + k2\pi / k \in Z)$$
\item[$\rightarrow$] $M_2$ symétrique de M par rapport à (Oy) si :
$$M_2(\rho, \pi -\theta)$$
\item[$\rightarrow$] $M_3$ symétrique de M par rapport à l'origine si :
$$M_3(\rho, \pi + \theta)$$
\end{itemize}
\subsection{Étude des tangentes}
\begin{itemize}
 \item[$\rightarrow$] Si $\rho(\theta)$ est dérivable en $\theta$ : $\dfrac{dM(\theta)}{\theta}$ est tangent en M($\theta$)
$$\dfrac{dM(\theta)}{\theta} = \rho'\overrightarrow{u} + \rho\overrightarrow{v}$$
avec : 
$$\overrightarrow{u}(\theta)\begin{pmatrix}
  \cos(x)  \\
  \sin x  \\
\end{pmatrix}$$
$$\overrightarrow{v}(\theta)\begin{pmatrix}
  \ -sin(x)  \\
  \cos x  \\
\end{pmatrix}$$
\item[$\rightarrow$] Si $\rho(\theta)$ = 0 : $\overrightarrow{u}(\theta)$ est tangent en 0
\end{itemize}
\subsection{Etude d'une branche infini}
Si : $$\lim_{\theta \mapsto \theta_0} \rho(\theta) = \infty$$
Alors la courbe possède une branche infini de direction $\overrightarrow{u}(\theta_0)$.\
On réalise alors une étude dans le repère (0,$\overrightarrow{u}(\theta_0)$,$\overrightarrow{v}(\theta_0))$ : 
$$\lim_{\theta \mapsto \theta_0} Y(\theta)$$
avec $Y(\theta) = \rho sin(\theta - \theta_0)$

\chapter{Les coniques}
\section{Définition}
\begin{de}
 On appelle conique de foyer F, de directrice $\Delta$ et d'excentricité e l'ensemble :
$$\left\lbrace M / e.distance(M,\Delta) = MF \right\rbrace $$
\end{de}
\subsection{Définitions bifocale d'une ellipse}
\begin{de}
Soit F et F' les deux foyers de l'ellipse. On défini cette ellipse par : 
$$(\mbox{ M appartient à l'ellipse }) \Leftrightarrow ( MF + MF' = cte = 2.a)$$
avec a le demi-grand axe.
\end{de}
\subsection{Définitions bifocale d'une hyperbole}
\begin{de}
Soit F et F' les deux foyers de l'hyperbole. On défini cette hyperbole par : 
$$(\mbox{ M appartient à l'hyperbole }) \Leftrightarrow ( |MF - MF'| = cte = 2.a)$$
avec a la valeur absolue de la distance du point d'intersection entre l'axe 0x et l'hyperbole avec l'origine.
\end{de}
\begin{de}
On appelle cercle principale d'une ellipse le cercle de centre O et de rayon a.
\end{de}
\section{Les différentes coniques}
\subsection{L'ellipse}
Une ellipse est définie par l'équation suivante : 
$$\dfrac{x^2}{a^2} + \dfrac{y^2}{b^2} = 1$$
Avec a>b, a est appelé le demi grand axe, et b le demi petit axe. On peut aussi paramétrer un point M de l'ellipse par : 
 \[\left\{\begin{array}{l}
   x(t) = acos(t)\\
   y(t) = bsin(t) \\
  \end{array}\right.\] 
Avec t l'angle entre l'axe Ox et OP, avec P le point correspondant à M sur le cercle principale de l'ellipse. M est obtenir à partir de P à l'aide d'une affinité.
\subsection{L'hyperbole}
Une hyperbole est définie par l'équation suivante : 
$$\dfrac{x^2}{a^2} - \dfrac{y^2}{b^2} = 1$$
Pour vérifier cette formule, on prend y=0, et on doit obtenir deux solutions. De plus, on obtient les asymptotes en annulant 1. On peut aussi paramétrer un point M de l'hyperbole par : 
 \[\left\{\begin{array}{l}
   x(t) = ach(t)\\
   y(t) = bsh(t) \\
  \end{array}\right.\] 
\subsection{Parabole}
L'équation réduite est : 
$$y^2 = 2px$$
\section{Équation polaire dans un repère de centre F}
Soit M($\rho,\theta$), $\Delta$ droite d'équation $x = x_{\Delta}$. L'équation générale est :
$$\rho = \dfrac{ex_{\Delta}}{1+ecos(\theta)}$$
avec e l'excentricité de la conique. Notons c l'abscisse de F.
\begin{itemize}
 \item[$\rightarrow$] Si e < 1: C'est une ellipse. Nous avons donc les résultats suivants \\
\begin{itemize}
 \item[$\rightarrow$] $e = \dfrac{c}{a}$\\
 \item[$\rightarrow$] $c^2 = a^2 - b^2$\\
 \item[$\rightarrow$] $x_{\Delta} = \dfrac{a^2}{c}$\\
\end{itemize}
 \item[$\rightarrow$] Si e = 1 : C'est une parabole.\\
 \item[$\rightarrow$] Si e > 1 : C'est une hyperbole.\\
\begin{itemize}
 \item[$\rightarrow$] $e = \dfrac{c}{a}$\\
 \item[$\rightarrow$] $c^2 = a^2 + b^2$\\
 \item[$\rightarrow$] $x_{\Delta} = \dfrac{a^2}{c}$\\
\end{itemize}
\end{itemize}

\part{Fonctions de $\mathbb{R}^2$ dans $\mathbb{R}$}
\chapter{Fonctions de $\mathbb{R}^2$ dans $\mathbb{R}$}
\section{Norme}
\begin{de}
Soit E un espace vectoriel.\\
n est une norme de E si :
$$n : E \rightarrow \mathbb{R}$$
telque :
\begin{itemize}
 \item[$\rightarrow$] $\forall x \in E$~ n(x) $\geq$ 0
 \item[$\rightarrow$] $\forall \lambda \in \mathbb{R},~ \forall x \in E,~ n(\lambda x) = |\lambda|n(x)$
 \item[$\rightarrow$] $\forall x \in E,~ n(x)=0 \Leftrightarrow x=0$
 \item[$\rightarrow$] $\forall (x,y) \in E^2~ n(x+y) \leq n(x)+n(y)$
\end{itemize}
\end{de}
\begin{prop}
La norme euclidienne, notée $||(x,y)||_2$ est défini par : 
$$\forall (x,y) \in \mathbb{R}^2~ ||(x,y)||_2 = \sqrt{x^2+y^2}$$
\end{prop}
\begin{de}
On défini la norme $||(x,y)||_n$ par : 
$$||(x,y)||_n = (|x|^n + |y|^n)^{1/n}$$
\end{de}
\begin{de}
On défini la norme infini par : 
$$\forall (x,y)\in \mathbb{R}^2~ ||(x,y)||_{\infty} = Max(|x|,|y|)$$
\end{de}
\subsection{Boules}
\begin{de}
Soit n une norme sur E, soit $x_0 \in E$, et $r \in \mathbb{R}^+$.\\
On appelle Boule de centre $x_0$, de rayon r : 
$$B(x_0,r) = \{ x \in E / n(x_0-x) \leq r\}$$ 
\end{de}
\subsection{Norme équivalentes}
\begin{de}
Soient $n_1,n_2$ deux normes sur E.\\
$n_1~ et~ n_2$ sont dites équivalentes si il existe deux réels strictement positifs telque $\forall x \in E $ : 
$$\alpha n_2(x) \leq n_1(x) \leq \beta n_2(x)$$
$$\dfrac{1}{\beta}n_1(x) \leq n_2(x) \leq \dfrac{1}{\alpha} n_1(x)$$
\end{de}
\begin{prop}
Dans un espace de dimension finie, toutes les normes sont équivalentes.
\end{prop}
\subsection{Convergence d'une suite}
Soit $(u_n)$ une suite de vecteur de E : 
\begin{itemize}
 \item[$\rightarrow$] On dit que $(u_n)$ converge vers 0 pour la norme n si la suite des réels ($n(u_n)$) converge vers 0
 \item[$\rightarrow$] Si deux normes sont équivalente, toutes suites convergentes pour l'une est convergente pour l'autre
 \item[$\rightarrow$] Dans un espace de dimension finie, la définition de la convergence ne dépend pas de la norme considéré.
\end{itemize}
\section{Limite d'une fonction de $\mathbb{R}^2$ dans $\mathbb{R}$}
\begin{de}
Soit A une partie de $\mathbb{R}^2$, et n une norme de $\mathbb{R}^2.$\\
Soit f une fonction de A dans $\mathbb{R}$.\\
Soit ($x_0,y_0) \in A$, et l un réel.\\
On dit que : $\underset{(x_0,y_0)}\lim f = l$
$$\forall \varepsilon > 0~ \exists \alpha > 0~ tq~ \forall (x,y) \in B((x_0,y_0),\alpha) : |f(x,y) - l| < \varepsilon$$
\end{de}
\begin{prop}
Soit f et g deux fonction défini sur A : 
\begin{itemize}
 \item[$\rightarrow$] La limite de la somme et la somme des limites.
 \item[$\rightarrow$] La limite du produit et le produit des limites
 \item[$\rightarrow$] Composition : Soit f : $\mathbb{R}^2 \rightarrow \mathbb{R}$ et $\varphi : \mathbb{R} \rightarrow \mathbb{R}$. Si $\underset{(a,b)}\lim f = l$ et si $\underset{l}\lim \varphi = m$, alors : $$\underset{(a,b)}\lim \varphi o f = m$$
\end{itemize}
\end{prop}
\subsection{Théorème d'encadrement}
Soient f,g,h fonctions défini sur A.\\ Si $\forall(x,y) \in A$ : 
$$h(x,y) \leq f(x,y) \leq g(x,y)$$
et si $\underset{(a,b)}\lim g = \underset{(a,b)}\lim h = l$, alors : 
$$\underset{(a,b)}\lim f = l$$
\subsection{Caractérisation de la divergence}
Si : 
\begin{itemize}
 \item[$\rightarrow$] $\underset{t_0}\lim x_1 = a$ 
 \item[$\rightarrow$] $\underset{t_0}\lim y_1 = b$ 
 \item[$\rightarrow$] $\underset{\alpha}\lim x_2 = a$ 
 \item[$\rightarrow$] $\underset{\alpha}\lim y_2 = b$ 
\end{itemize}
et $\underset{t \rightarrow t_0}\lim f(x_1(t),y_1(t)) = l$ et $\underset{u \rightarrow \alpha}\lim f(x_2(y),y_2(u)) = l'$, avec  l $\neq$ l', alors : 
$$\underset{(a,b)}\lim f \mbox{ n'existe pas}$$
\subsection{Continuité}
\begin{de}
Si f est une fonction définie en (a,b) et sur un voisinage de (a,b), avec : 
$$\lim_{(a,b)} f = f(a,b)$$
Alors, on dit que f est continue en (a,b).
\end{de}
\section{Dérivation}
\subsection{Dérivées partielles}
Soit f fonction définie au voisinage de (a,b).
\begin{de}
On appelle première dérivée partielle de f en (a,b) : 
$$\dfrac{\partial f}{\partial x} (a,b) = \lim_{x \rightarrow a} \dfrac{f(x,b) - f(a,b)}{x-a}$$
\end{de}
\begin{de}
On appelle deuxième dérivée partielle de f en (a,b) : 
$$\dfrac{\partial f}{\partial y} (a,b) = \lim_{y \rightarrow b} \dfrac{f(a,y) - f(a,b)}{y-b}$$
\end{de}
\subsection{Dérivée suivant un vecteur}
\begin{de}
Soit $\overrightarrow{u}(\alpha,\beta)$ un vecteur de $\mathbb{R}^2$.
On défini le nombre dérivée de f en (a,b) suivant $\overrightarrow{u}$, notée d$_{\overrightarrow{u}}f(a,b)$, par : 
$$d_{\overrightarrow{u}}f(a,b) = \lim_{t \rightarrow 0} \dfrac{f(a+\alpha.t,b+\beta.t) - f(a,b)}{t}$$
\end{de}
\subsection{Fonction de classe $C^1$}
\begin{de}
Soit A une partie de $\mathbb{R}^2$. On dit que f est de classe $C^1$ sur A si : 
\begin{itemize}
\item[$\rightarrow$] f possède sur A deux dérivées partielles
 \item[$\rightarrow$] Ces deux fonctions sont continue sur A
\end{itemize}
\end{de}
\subsection{Développement limité d'ordre 1}
Si f est de classe $C^1$ sur A, voisinage de (a,b), alors :
$$\forall(x,y) \in A~ f(x,y) = f(a,b) + \dfrac{\partial f}{\partial x}(a,b).(x-a) + \dfrac{\partial f}{\partial y}(a,b).(y-b) + o(||(x,y)-(a,b)||)$$
\begin{prop}
Soit $\overrightarrow{u}(\alpha,\beta)$ et f une fonction de classe $C^1$ au voisinage de (a,b).
$$d_{\overrightarrow{u}}f = \alpha.\dfrac{\partial f}{\partial x}(a,b) + \beta\dfrac{\partial f}{\partial y}(a,b)$$
On en déduit que si f est de classe $C^1$ au voisinage de (a,b), alors $\forall \overrightarrow{u} \in \mathbb{R}^2,~ d_{\overrightarrow{u}}f$ existe et est continue.
\end{prop}
\subsection{Plan tangent}
\begin{de}
On appelle plan tangent à la surface représentative d'une fonction de classe $C^1$ le plan définie par le repère : 
$$(M(a,b,f(a,b),\overrightarrow{t_{\overrightarrow{i}}},\overrightarrow{t_{\overrightarrow{j}}})$$
avec : 
$$\overrightarrow{t_{\overrightarrow{i}}} =\begin{pmatrix}
  1  \\
  0  \\
  \dfrac{\partial f}{\partial x}(a,b)\\
\end{pmatrix} $$
$$\overrightarrow{t_{\overrightarrow{j}}} =\begin{pmatrix}
  0  \\
  1  \\
  \dfrac{\partial f}{\partial y}(a,b)\\
\end{pmatrix} $$
\end{de}
\subsubsection{Vecteur normale au plan tangent}
\begin{de}
On défini un vecteur normale au plan tangent, notée $\overrightarrow{n}$, par :
$$\overrightarrow{n} = \overrightarrow{grad}(f(x,y)-z)$$
\end{de}
\subsection{Dérivées partielles d'ordre 2}
\begin{de}
Soit f définie sur une partie A de $\mathbb{R}^2$.\\
Si $\dfrac{\partial f}{\partial x}$ est défini sur A et possède des dérivées partielles, on les notes : 
$$\dfrac{\partial^2f}{\partial x^2} = \dfrac{\partial}{\partial x}.\dfrac{\partial f}{\partial x}$$
$$\dfrac{\partial^2f}{\partial y \partial x} = \dfrac{\partial}{\partial y}.\dfrac{\partial f}{\partial x}$$
Respectivement pour $\dfrac{\partial f}{\partial y}$, on obtient :
$$\dfrac{\partial^2f}{\partial y^2} = \dfrac{\partial}{\partial y}.\dfrac{\partial f}{\partial y}$$
$$\dfrac{\partial^2f}{\partial x \partial y} = \dfrac{\partial}{\partial x}.\dfrac{\partial f}{\partial y}$$
\end{de}
\begin{theo}
Si $\dfrac{\partial^2f}{\partial x \partial y}$ et $\dfrac{\partial^2f}{\partial y \partial x}$ sont continue sur A, alors : 
$$\dfrac{\partial^2f}{\partial x \partial y} = \dfrac{\partial^2f}{\partial y \partial x}$$
\end{theo}
\subsection{Dérivée des composées}
\subsubsection{Premier type de composées}
Soit f une fonction de classe $C^1$ de A dans $\mathbb{R}$, avec $A c \mathbb{R}$.\\
Soit $\varphi$ une fonction dérivable sur $\mathbb{R}$.\\
Soit g = $\varphi$ o f.
$$g : A \rightarrow \mathbb{R}$$
$$(x,y) \mapsto \varphi(f(x,y))$$
On obtient les dérivées partielles suivantes : 
$$\dfrac{\partial g}{\partial x}(x,y) = \dfrac{\partial f}{\partial x}(x,y).\varphi'(f(x,y))$$
$$\dfrac{\partial g}{\partial y}(x,y) = \dfrac{\partial f}{\partial y}(x,y).\varphi'(f(x,y))$$
\subsubsection{Second type de composées}
Soit x,y deux fonctions de $\mathbb{R}$ dans $\mathbb{R}$, dérivable sur $\mathbb{R}$.\\
Soit f une fonction de $\mathbb{R}^2$ dans $\mathbb{R}$, de classe $C^1$ sur $\mathbb{R}^2$.\\
Soit $\varphi$ la fonction définie par : 
$$\varphi : \mathbb{R} \rightarrow \mathbb{R}$$
$$t \mapsto f(x(t),y(t))$$
On obtient, à l'aide d'un développement limité :
$$\forall t : \varphi'(t) = x'(t).\dfrac{\partial f}{\partial x}(x(t),y(t)) + y'(t).\dfrac{\partial f}{\partial y}(x(t),y(t))$$

\part{Equations differentielles}

\chapter{Équation différentielle}
\section{Fonction exponentielle complexe}
Soit :
$$f : t \mapsto x(t) + iy(t)= e^{rt}$$
Avec r = a+ib. \\On obtient, $\forall t \in \mathbb{R}$:
$$f'(t) = re^{rt}$$
\section{Équation différentielle}
\subsection{Première ordre}
$(\forall x \in \mathbb{R}~ ay'(t) + by(t) = 0) \Leftrightarrow (\exists K \in \mathbb{R},~ tq~ \forall x \in \mathbb{R}~ y=Ke^{-\dfrac{b}{a}x})$
\subsection{Second ordre}
On établie l'équation caractéristique, de la forme :
$$ax^2+bx+c = 0$$
On détermine $\Delta$ et on obtient :
\begin{itemize}
 \item[$\rightarrow$] $\Delta > 0$ : $$\exists(A,B)\in \mathbb{R}^2~ tq~ \forall x \in \mathbb{R}~ y(x) = Ae^{r_1x}+Be^{r_2x}$$
 \item[$\rightarrow$] $\Delta = 0$ : $$\exists(A,B)\in \mathbb{R}^2~ tq~ \forall x \in \mathbb{R}~ y(x) = (Ax+B)e^{r_0x}$$
 \item[$\rightarrow$] $\Delta > 0$ : Solution dans C, avec $r_0 = \pm i\omega$ $$\exists(A,B)\in \mathbb{R}^2~ tq~ \forall x \in \mathbb{R}~ y(x) = (Acos(\omega x)+Bsin(\omega x))e^{r_0x}$$ 
\section{Recherche d'une solution particulière}
\subsection{Second membre constant}
Soit (E) l'équation différentielle suivante :
$$ay''+by'+cy=d$$
Si $C \neq 0$ :\
$$y_0 : x \mapsto \dfrac{d}{c}$$
est une solution particulière.\
Si $C = 0$ :\
$$y_0 : x \mapsto \dfrac{d}{b}x$$
est une solution particulière
\subsection{Second membre polynomiale}
En géneral, on recherche un polynome de meme degrès.\
Soit :
$$y''+3y = 2x+1$$
On pose : 
$$y_0 : x \mapsto ax+b$$
On dérive deux fois $y_0$ et on remplace dans l'équation pour déterminer a et b
\subsection{Second membre exponentielle}
Si le second membre est de la forme :
$$x \mapsto e^{\alpha x}$$
Alors on peut espérer une solution de la forme $\lambda e^{\alpha x}$
\section{Méthode de variation de la constante}
Soit (E) l'équation différentielle suivante : 
$$(E)~ :~ ay'(t)+by(t) = f(x)$$
On résoud l'équation sans second membre, plus on pose $\forall x \in \mathbb{R}$:
$$z(x) = \dfrac{y(x)}{e^{-\frac{b}{a}x}} \Leftrightarrow y(x) = z(x)e^{-\frac{b}{a}x}$$
Puis on injecte cette expression y(x) dans (E) pour déterminer z(x)
\end{itemize}
\section{Principe de superposition}
Soit (E) l'équation différentielle suivante :
$$(E)~ :~ ay''+by'+cy = f_1(x) + f_2(x)$$
On considere :
$$(E_1)~ :~ ay''+by'+cy = 0$$
$$(E_2)~ :~ ay''+by'+cy = f_1(x)$$
$$(E_3)~ :~ ay''+by'+cy = f_2(x)$$
Soit $y_1,y_2$ solutions respective de $(E_2)$ et $(E_3)$. La solution particuliere de (E) est $y_1+y_2$

\chapter{Équations différentielle linéaire}
\section{Généralité}
\begin{de}
On considère (E) l'équation d'inconnue la fonction y n fois dérivable sur une partie A de $\mathbb{R}$ :
$$\forall x \in A,~ a_n(x).y^{(n)}(x)+...+a_0.y(x) = b(x)$$
avec : $a_n,..,a_0,b$ des fonctions définies sur A.\\
On dit que (E) est une équation differentielle linéaire d'ordre n.
\end{de}
\begin{nota}
On note cette équation : 
$$(E) :~ \forall x \in A :~  a_n(x).y^{(n)}+...+a_0.y = b(x)$$
\end{nota}
\begin{prop}
L'ensemble des solutions de (E) est soit:
\begin{itemize}
 \item[$\rightarrow$] Vide
 \item[$\rightarrow$] Un espace affine de direction l'espace vectoriel des solutions de l'équation sans second membre.\\
\end{itemize}
Si les coefficients sont constant.\\
L'ensemble des solutions est un espace de dimension n. 
\end{prop}
\chapter{Équations différentielles linéaire d'ordre 1}
\begin{de}
Soit A $\in \mathbb{R}$.\\
Soit a,b,c trois fonctions définies sur A, et (E):
$$(E) :~ a(x).y' + b(x).y = c(x)$$
$$(E_0) :~ a(x).y' + b(x).y = 0$$
Si :
\begin{itemize}
 \item[$\rightarrow$] Si a et b sont continues sur A
 \item[$\rightarrow$] A est un intervalle, notons le I
 \item[$\rightarrow$] $\forall x \in I~ a(x) \neq 0$
\end{itemize}
Alors : 
$$(E_0) \Leftrightarrow (\exists K \in \mathbb{R}~ tq~ \forall x \in I~ y(x)=K.e^{\int \frac{-b(x)}{a(x)}dx})$$
\end{de}

\part{Int\'egration}
\chapter{Intégration}
\begin{de}
L'intégrale d'une fonction est par définition un nombre.\\
Une primitive est une autre fonction.
\end{de}
\section{Fonctions continues par morceaux}
\subsection{Subdivision}
Soit [a,b] segment de $\mathbb{R}$, avec a<b.
\begin{de}
On appelle subdivision de [a,b], une liste vérifiant :
$$a=x_0<x_1<....<x_n=b$$
On note $\sigma=(x_i)_{a\leq i\leq n}$.\\
Le pas d'une subdivision, qui est la longueur d'intervalle la plus importante, est défini comme :
$$\underset{1 \leq i \leq n}\max |x_i - x_{i-1}|$$
\end{de}
Si les intervalles sont tous de la mêmes longueurs, la subdivision est dite régulière. De plus, si $\sigma$ est celle d'une subdivision régulière de [a,b], alors :
$$\forall k,~ x_k = a+k\dfrac{b-a}{n}$$ 
\subsection{Fonction en escalier sur [a,b]}
Soit f fonction définie sur [a,b].
\begin{de}
 On dit que f est en escalier si il existe une subdivision de [a,b] : $(x_i)_{0 \leq i \leq n}$ telle que :
$$\forall i \in \left\lbrace1,...n\right\rbrace, \mbox{ f est constante sur } ]x_{i-1};x_i[$$
\end{de}
\begin{prop}
On défini les propriétés suivantes :
 \begin{itemize}
 \item[$\rightarrow$] Une fonction en escalier est bornée
 \item[$\rightarrow$] L'ensemble des fonctions en escalier sur [a,b] est un $\mathbb{R}$-espace vectoriel
\end{itemize}
\end{prop}
\subsection{Fonction continue par morceaux}
Soit f défini sur [a,b]
\begin{de}
 f est continue par morceaux sur [a,b] si :
\[\left\{\begin{array}{l}
   \forall i \in \left\lbrace1,2,...,n\right\rbrace \mbox{ f est continue sur }]x_{i-1};x_i[\\
   \forall i \in \left\lbrace1,2,...,n\right\rbrace~ \underset{x_{i}^-}\lim\mbox{ existe }\\
    \forall i \in \left\lbrace0,2,...,n-1\right\rbrace~ \underset{x_{i}^+}\lim\mbox{ existe }\\
  \end{array}\right.
\]
\end{de}
\begin{prop}
 On défini les propriétés suivantes :
\begin{itemize}
 \item[$\rightarrow$] Une fonction continue par morceaux est bornée
 \item[$\rightarrow$] L'ensemble des fonctions continue par morceaux sur [a,b] est un espace vectoriel
 \item[$\rightarrow$] Une fonction en escalier est continue par morceaux
 \item[$\rightarrow$] Une fonction continue sur [a,b] est continue par morceaux
\end{itemize}
\end{prop}

\subsection{Approximation d'une fonction continue par morceaux par des fonctions en escalier}
Soit f continue par morceaux sur [a,b] ( ce qui comprend les fonctions continues sur [a,b])
\begin{de}
$$\exists \varepsilon > 0 \mbox{ Il existe deux fonctions en escalier sur [a,b], }\varphi \mbox{ et } \psi \mbox{ telque } $$
\[\left\{\begin{array}{c}
  \forall x \in [a,b]~ \varphi(x)\leq f(x) \leq \psi(x)\\
  0 \leq \psi(x) - \varphi(x) \leq \varepsilon 
  \end{array}\right.
\]
\end{de}
\section{Intégrale de Riemann}
\subsection{Intégrale d'une fonction en escalier}
\begin{de}
Soit $\varphi$ fonction en escalier sur [a,b].\\
Notons $(x_i)_{0 \leq i \leq n}$ une subdivision adaptée à $\varphi$, et posons : 
$$\forall i \in \left\lbrace 1,..,n\right\rbrace,~ \forall x \in ]x_{i-1};x_i[~ \varphi(x) = \lambda_i$$
L'intégrale sur [a,b] de $\varphi$ est : 
$$\int_{[a,b]}\varphi = \sum_{i=1}^n(x_i - x_{i-1})\lambda_i$$
\end{de}
On note aussi cette intégrale de la façon suivante :\\
Si a < b, alors :
$$\int_{[a,b]}\varphi = \int_a^b\varphi$$
Si b < a : 
$$\int_{[a,b]}\varphi = \int_b^a\varphi$$
Convention : 
$$\int_a^b\varphi =-\int_b^a\varphi$$
$$\int_a^a\varphi = 0$$
\begin{prop}
Si $\varphi$ et $\psi$ sont deux fonctions en escalier sur [a,b].\\
Si $\lambda$ et $\mu$ sont deux réels :
$$\int_{[a,b]}\lambda\varphi + \mu\varphi = \lambda\int_{[a,b]}\varphi + \mu\int_{[a,b]}\psi$$
\end{prop}
\begin{prop}
Soient $\varphi$ et $\psi$ deux fonctions en escalier sur [a,b].\\
Si $\forall x \in [a,b]$, $\varphi(x) \leq \psi(x)$, alors :
$$\int_{[a,b]}\varphi \leq \int_{[a,b]}\psi$$
\end{prop}
De plus, si $\varphi \geq 0$ sur [a,b] alors : 
$$\int_{[a,b]}\varphi \geq 0$$
\section{Intégrale d'une fonction continue par morceaux}
Soit f fonction continue par morceaux sur [a,b]. Notons :
\[\left\{\begin{array}{l}
   E_+ = \left\lbrace \varphi \mbox{ en escalier sur [a,b] } / \varphi \geq g \right\rbrace \\
   E_- = \left\lbrace \varphi \mbox{ en escalier sur [a,b] } / \varphi \leq g \right\rbrace \\
   A_+ = \left\lbrace \int_{[a,b]} \varphi / \varphi \in E_+ \right\rbrace\\
   A_- = \left\lbrace \int_{[a,b]} \varphi / \varphi \in E_- \right\rbrace\\
  \end{array}\right.
\]
\begin{prop}
 Inf($A_+$) et Sup($A_-$) existent et sont égaux
\end{prop}
\begin{de}
 La valeur commune de ces deux réels est l'intégrale de Riemannsur [a,b] de f. On la note :
$$\int_{[a,b]}f$$
\end{de}
\subsection{Somme de Riemann}
\begin{prop}
Soit f fonction continue par morceaux sur [a,b].\\
Notons $\forall n \in N$:
Alors $(u_n)_{n \geq 0}$ et $(v_n)_{n \geq 0}$ converge vers $\int_{[a,b]}f$
\[\left\{\begin{array}{l}
   u_n = \frac{b-a}{n} \sum_{k=1}^n f(a+k\frac{b-a}{n} ) \\
   v_n = \frac{b-a}{n} \sum_{k=0}^{n-1} f(a+k\frac{b-a}{n} ) \\
  \end{array}\right.
\]
\end{prop}
\subsection{Linéarité}
\begin{prop}
Soient $\lambda$,$\mu$ réels, f, g fonctions continues par morceaux.
$$\int_{[a,b]}\lambda\varphi + \mu\varphi = \lambda\int_{[a,b]}\varphi + \mu\int_{[a,b]}\psi$$
\end{prop}
\subsection{Transmition de l'ordre}
\begin{prop}
 Si f et g sont continue par morceaux et $f \leq g$, alors :
$$\int_{[a,b]}f \leq \int_{[a,b]}g$$
\end{prop}
\subsection{Intégrale et valeur absolu}
\begin{prop}
Soit f fonction continue par morceaux sur [a,b], donc : 
$$\int_{[a,b]}|f| \geq \mid\int_{[a,b]}f\mid$$
\end{prop}
\subsection{Relation de Chasles}
\begin{prop}
Soit f continues par morceaux sur [a,b] et $b \in [a,c]$.
$$\int_{[a,c]}f = \int_{[a,b]}f + \int_{[b,c]}f$$
\end{prop}
\subsection{Inégalité de la moyenne}
\begin{prop}
Soit f,g continues par morceaux sur [a,b], g est bornée, avec a < b. Donc M = $\underset{[a,b]}\sup|g|$ existe.
$$|\int_{[a,b]}fg| \leq \underset{[a,b]}\sup|g|\times\int_{[a,b]}|f|$$
\end{prop}
\begin{de}
 $\dfrac{1}{b-a}\int_{[a,b]}g$ est la valeur moyenne de g sur [a,b].
$$\int_{[a,b]}g = \mu(b-a)$$
\end{de}
\section{Intégrale et primitive d'une fonction continue}
On obtient les propriétés suivant : 
\[\left\{\begin{array}{l}
   \mbox{Soit } x \in I.\mbox{ f est continue sur I, donc } \int_a^x f \mbox{ existe} \\
   \mbox{Si } x_0 \mbox{ est à l'intérieur de I, si f est continue sur I, alors } \int_a^x f \mbox { est aussi continue sur I} \\
   \mbox{ Si f est continue sur I, alors } g : x \mapsto \int_a^x f \mbox{ est dérivable sur I et sa dérivé est f.}
  \end{array}\right.
\]
De la dernière propriété, on déduit que g est de classe $C^1$ sur I. En résumé, si f est de classe $C^n$ alors g est de classe $C^{n+1}$
\begin{prop}
Si $\varphi$ est une fonction positive et continue sur [a,b] d'inégalité nulle, alors :
$$\varphi = 0$$
\end{prop}

\subsection{Utilisation des primitives d'une fonction continue}
\begin{de}
 Soit f définie sur un intervalle I. Une primitive de f sur I, c'est une fonction dérivable sur I dont la dérivée est f.
\end{de}
\subsection{Ensemble des primitives d'une fonction continue}
Soit f continue sur un intervalle I, si F est une primitive de f sur I, alors :
$$(\mbox{G est une autre primitive de f sur I}) \Leftrightarrow (\exists K \in \mathbb{R} \mbox{ tq } \forall x \in I~ G(x)=F(x)+K)$$
Il en découle que : 
$$( \mbox{F est une primitive de f sur I }) \Leftrightarrow (\exists K \in \mathbb{R} \mbox{ tq } \forall x \in I~ F(x)=\int_a^x f(t)dt + K)$$
Et que $\forall (a,b)^2 \in I^2$
$$\int_a^b f(t)dt = F(b) - F(a)$$
\subsection{Notation}
\begin{de}
 Si f est continue sur I : $\int f(x)dx$ désigne la valeur de x d'une primitive de f.
\end{de}
\subsection{Technique de calcul d'une intégrale}
\subsubsection{Intégrale par partie}
\begin{de}
 Si f,g sont de classe $C^1$ sur I, $(a,b)\in I^2$, alors :
$$\int_a^b fg' = [fg]_a^b - \int_a^b f'g$$
\end{de}
\subsubsection{Changement de variables}
\begin{de}
 Soit u une bijection de classe $C^1$ de $[\alpha,\beta]$ sur un intervalle [a,b]\\
Soit f continue sur [a,b].
$$\int_a^b f(u)du = \int_{\alpha}^{\beta} f(u(t))u'(t)dt$$
\end{de}
\subsection{Intégrale d'une fonction paire, impaire, periodique}
\subsubsection{Fonction paire}
\begin{prop}
Soit a $\in \mathbb{R}$\\
Si f est continue et paire sur [-a,a], alors :
$$\int_{-a}^a f = 2\int_0^a f$$
\end{prop}
\subsubsection{Fonction impaire}
\begin{prop}
Si f est continue et impaire sur [-a,a], alors :
$$\int_{-a}^a f = 0$$
\end{prop}
\subsubsection{Fonction periodique}
\begin{prop}
Si f est continue et T periodique sur $\mathbb{R}$\\
Soit a $\in \mathbb{R}$
$$\int_a^{a+T} \mbox{ f est indépendant de a}$$
\end{prop}
\section{Inégalité de Cauchy-Schwarz}
\begin{de}
Soient f,g continues sur [a,b] : 
$$|\int_{[a,b]}fg| \leq \sqrt{\int_{[a,b]}f^2\int_{[a,b]}g^2}$$
\end{de}
Si cette inégalité devient une égalité, alors $\exists \lambda_0 \in \mathbb{R}$ telque 
$$g = -\lambda_0f$$
\section{Formule de Taylor avec reste intégrale}
\begin{de}
Soit f fonction de classe $C^n$.\\
$\forall x \in D_f$, au voisinage de a :
$$f(x)=f(a)+....+\dfrac{(x-a)^{n-1}}{(n-1)!}f^{(n-1)}(a) + \int_a^x \dfrac{(x-a)^{n-1}}{(n-1)!}f^{(n)}(t)dt$$
\end{de}
\section{Inégalité de Taylor-Lagrange}
\begin{de}
Soit f de classe $C^{n+1}$ sur I, $a \in I$.\\
Supposons que $f^{(n+1)}$ soit majorée sur I.\\
Notons $M_{n+1} = \underset{I}\sup|f^{(n+1)}|$\\
$\forall x \in I$ :
$$|\int_a^x\dfrac{(x-t)^n}{n!}(t)dt| \leq \dfrac{(x-a)^{n+1}}{(n+1)!}M_{n+1}$$
\end{de}



\part{Nombres complexes}
\chapter{Nombres complexes}
\section{Formules}
\subsection{Généralités}
\begin{itemize}
\item[{$\rightarrow$}] (C,+,x) est un corps
 \item[{$\rightarrow$}] (a+ib)+(a'+ib') = (a+a')+i(b+b')\\
 \item[{$\rightarrow$}] (a+ib).(a'+ib') = (aa'-bb')+i(ab'+a'b)\\
 \item[{$\rightarrow$}] Il y a unicité de la partie réelle et de la partie imaginaire pour un complexe.\\
 \item[{$\rightarrow$}]$\overline{z+z'}=\overline{z}+\overline{z'}$\\
 \item[{$\rightarrow$}]$\overline{z.z'}=\overline{z}.\overline{z'}$\\
 \item[{$\rightarrow$}]Re(z) = $\dfrac{z+\overline{z}}{2}$\\
\item[{$\rightarrow$}]Im(z) = $\dfrac{z-\overline{z}}{2}$\\
\item[{$\rightarrow$}]$z_{\overrightarrow{AB}} = z_b-z_a$ \\
\item[{$\rightarrow$}]On défini le barycentre de la façon suivante : $$\alpha + \beta \neq 0, z_g = \dfrac{\alpha z_A + \beta z_B}{\alpha + \beta}$$
\end{itemize}
\subsection{Forme Trigonométrique et exponentielle}
Soit z = x+iy un complexe.
\begin{itemize}
 \item[{$\rightarrow$}] |z| = $\sqrt{x^2+y^2}$\\
 \item[{$\rightarrow$}] $|z|^2 = z.\overline{z}$\\
 \item[{$\rightarrow$}] |-z| = |z| = $|\overline{z}|$\\
\item[{$\rightarrow$}] $(z \in \mathbb{R}) \Leftrightarrow (\overline{z} = z)$\\
\item[{$\rightarrow$}] $(z \in i\mathbb{R}) \Leftrightarrow (\overline{z} = -z)$\\
 \item[{$\rightarrow$}] |Im(z)| $\leq$ |z|\\
 \item[{$\rightarrow$}] |Re(z)| $\leq$ |z|\\
 \item[{$\rightarrow$}] $|z.z'| = |z|.|z'|$ \\
 \item[{$\rightarrow$}] $|z+z'| \leq |z|+|z'|$\\
\item[{$\rightarrow$}] $e^{i(\theta + \theta'}) = cos(\theta+\theta')+isin(\theta+\theta')$\\
\item[{$\rightarrow$}] $\dfrac{e^{i\theta}+e^{-i\theta}}{2} = cos(\theta)$\\
\item[{$\rightarrow$}] $\dfrac{e^{i\theta}-e^{-i\theta}}{2} = sin(\theta)$\\
\item[{$\rightarrow$}] cos(iy) = ch(y)\\
\item[{$\rightarrow$}] sin(iy) = ish(y)\\
\item[{$\rightarrow$}] Formule de Moivre : $$(e^{i\theta})^n =e^{in\theta} \Leftrightarrow (cos(\theta)+isin(\theta))^n = cos(n\theta)+isin(in\theta)$$
\item[{$\rightarrow$}] z = x+iy = $\rho e^{i\theta}$\\
\item[{$\rightarrow$}] Racine $n^{eme}$ de l'unite : $$z^n = 1 \Leftrightarrow \exists k \in \left\lbrace 0,1,...,n-1\right\rbrace~ z = e^{i\dfrac{k2\pi}{n}}$$
\item[{$\rightarrow$}] Racine $n^{eme}$ d'un complexe non nul, avec $z = \rho e^{i\theta}$, $z_0 = \rho_0 e^{i\theta_0}$ : $$z^n = z_0 \Leftrightarrow \exists k \in \left\lbrace 0,1,...,n-1\right\rbrace~ z =A.e^{i\dfrac{k2\pi}{n}}$$ Avec A la solution évidente (Passage à la racine $n^{eme}$)
\end{itemize}
\chapter{Nombres complexe et géométrie dans le plan}
\section{Alignement, Orthogonalité, Cocyclicité}
Soit $(\overrightarrow{AB};\overrightarrow{AC})$ l'angle formé par ces deux vecteurs.
$$(\overrightarrow{AB};\overrightarrow{AC}) = arg\left(\dfrac{c-a}{b-a}\right) ~ [2\pi]$$
Soit :$$z = \left(\dfrac{c-a}{b-a}\right) $$
\subsection{Alignements}
$$\mbox{A,B,C alignées} \Leftrightarrow \left(\dfrac{c-a}{b-a} \in \mathbb{R} \right) \Leftrightarrow arg\left(\dfrac{c-a}{b-a}\right) = 0  $$
$$\mbox{A,B,C alignées} \Leftrightarrow (z = \overline{z}) \Leftrightarrow (Det(\overrightarrow{AB};\overrightarrow{AC})) = 0$$
avec $Det(\overrightarrow{u},\overrightarrow{v)} = xy' - x'y$ 
\subsection{Orthogonalité}
$$\overrightarrow{AB};\overrightarrow{AC} \mbox{ orthogonaux} \Leftrightarrow \left(\dfrac{c-a}{b-a} \in i\mathbb{R} \right) \Leftrightarrow arg\left(\dfrac{c-a}{b-a}\right) = \dfrac{\pi}{2}~ [\pi]  $$
$$\overrightarrow{AB};\overrightarrow{AC} \mbox{ orthogonaux} \Leftrightarrow (z = -\overline{z}) \Leftrightarrow (\overrightarrow{AB}.\overrightarrow{AC}) = 0$$
\subsection{Cocyclicité}
Soit A,B,C trois points d'un cercle C de centre O.
$$(\overrightarrow{OB};\overrightarrow{OC}) = 2(\overrightarrow{AB};\overrightarrow{AC})~ [2\pi]$$
\subsubsection{Condition de cocyclicité}
\begin{prop}
 Si $(\overrightarrow{AB};\overrightarrow{AC}) = (\overrightarrow{DB};\overrightarrow{DC})~ [\pi]$ alors A,B,C,D sont soit cocyclique, soit alignés.
\end{prop}
\section{Similitude}
Soit z l'affixe de M, z' l'affixe de M'.
\subsection{Translation}
\begin{de}
 Soit $\overrightarrow{u}$ vecteur du plan. On appele translation de vecteur $\overrightarrow{u}$ l'application :
$$t_{\overrightarrow{u}} : P \rightarrow P$$
$$M \mapsto M'$$
avec $\overrightarrow{MM'} = \overrightarrow{u}$
\end{de}
\subsubsection{Expression analytique complexe}
soit $\alpha$ l'affixe de $\overrightarrow{u}$\
Alors :
$$z' = \alpha + z$$
\subsubsection{Bijectivité}
$t_{\overrightarrow{u}}$ est une application bijective. Soit :
$$t_{\overrightarrow{u}}^{-1} : P \rightarrow P$$
$$M \mapsto M'$$
avec z' = z - $\alpha$
\subsection{Homothetie}
\begin{de}
 Soit $\Omega$ un point d'affixe $\omega$. Soit $k \in \mathbb{R}$.\
On appele homothétie de centre $\Omega$ et de rapport k l'application :
$$h :  P \rightarrow P $$
$$M \mapsto M'$$
avec $\overrightarrow{\Omega M'}=k\overrightarrow{\Omega M}$ 
\end{de}
\subsubsection{Expression analytique complexe}
$$z'-\omega = k(z-\omega)$$
On détermine le centre d'une homothétie en déterminant son point fixe, donc en résolvant :
$$z = z'$$
\subsubsection{Bijectivité}
h est une application bijective. Soit :
$$h^{-1} : P \rightarrow P$$
$$M \mapsto M'$$
avec z' - $\omega$ = $\dfrac{1}{k}(z - \omega)$
\subsection{Rotation}
\begin{de}
Soit $\Omega$ un point d'affixe $\omega$ et $\theta$ un réel.\
On appele rotation de centre $\Omega$ et d'angle $\theta$ l'application
$$r : P \rightarrow P$$
$$M \mapsto M'$$
avec $\Omega M'$=$\Omega M$ et $(\overrightarrow{\Omega M};\overrightarrow{\Omega M'})$ = $\theta$ [$2\pi$]
\end{de}
\subsubsection{Expression analytique complexe}
$$z' - \omega = e^{i\theta}(z-w)$$
On détermine le centre d'une rotation en déterminant son point fixe, donc en résolvant :
$$z = z'$$
\subsubsection{Bijectivité}
h est une application bijective. Soit :
$$r^{-1} : P \rightarrow P$$
$$M \mapsto M'$$
avec z' - $\omega = e^{-i\theta}(z - \omega)$
\subsection{Similitude}
\begin{de}
Soit $\Omega$ un point d'affixe $\omega$ et ($\theta$,k) $\in \mathbb{R}^2$.\
On appele similitude direct de centre $\Omega$, d'angle $\theta$, et de rapport k l'application :
$$S : P \rightarrow P$$
$$ M \mapsto M'$$
avec $\Omega M'$=$k\Omega M$ et $(\overrightarrow{\Omega M};\overrightarrow{\Omega M'})$ = $\theta$ [$2\pi$]
\end{de}
\subsubsection{Expression analytique complexe}
$$z' - \omega  = ke^{i\theta}(z-w)$$
On détermine le centre d'une similitude en déterminant son point fixe, donc en résolvant :
$$z = z'$$
\subsubsection{Bijectivité}
S est une application bijective. Soit :
$$S^{-1} : P \rightarrow P$$
$$M \mapsto M'$$
avec z' - $\omega$ = $\dfrac{1}{k}e^{-i\theta}(z - \omega)$
\subsection{Affinité}
Soit $\varphi$ l'application défini par : 
$$\varphi : Plan \mapsto Plan$$
$$P(x,y) \mapsto M (x,\dfrac{b}{a}.y)$$
$\varphi$ est appelé affinité de base Ox, de direction Oy et de rapport $\dfrac{b}{a}$

\part{Polynomes}
\chapter{Les polynomes}
\section{Définitions}
Soit K un corps ( Soit $\mathbb{R}$, soit $\mathbb{C}$)
\begin{de}
Un polynome à coefficiants dans K est une suite d'élement de K tous nul à partir d'un certain rang.
$$P=(a_0,.....,a_n,0,...)$$
On peut l'écrire aussi sous la forme :
$$P=a_0+a_1X+...+a_nX^n$$
avec X l'indéterminé.\\
Il existe aussi la forme suivante :
$$P = \sum_{k=0}^n a_kX^k$$
L'ensemble des polynomes à coefficiants dans K est notée K[X].
\end{de}
Soit P et Q deux polynomes. On as : 
$$P = Q \Leftrightarrow \forall k \in N~ a_k=b_k$$
\subsection{Opérations}
On peut effectuer quatres opérations :
\begin{itemize}
 \item[$\rightarrow$] Une addition : $$P+Q = \sum_{k=0}^{\infty}(a_k+b_k)X^k$$
 \item[$\rightarrow$] Un produit : $$P.Q = \sum_{k,k'\geq0}^{\infty}(a_k.b_k')X^{k+k'}$$
 \item[$\rightarrow$] Un produit par $\lambda$, $\lambda \in K$ : $$\lambda P = \sum_{k=0}^{\infty}(\lambda a_k)X^k$$
\item[$\rightarrow$] Une composée : $$P(Q) = \sum_{k=0}^{\infty}a_k Q^k$$
\end{itemize}
\subsection{Structure}
\subsection{Polynome constante}
On observe qu'un polynome constant s'identifie à un élement du corps. On obtient donc que :
$$K~ c~ K[X]$$
\subsubsection{Structure de (K[X],+,x)}
\begin{itemize}
 \item[$\rightarrow$] (K[X],+) est un groupe commutatif
 \item[$\rightarrow$] (K[X],+,x) est un anneau commutatif : On peut donc utiliser les identités remarquables sur les polymones. Cette anneau est intègre, ce qui signifie que : $$(P.Q = 0) \Leftrightarrow (P=0~ ou~ Q=0)$$

\end{itemize}
\subsection{Fonction polynome associée}
Soit $P \in K[X]$, défini par :
$$P = a_0+a_1X+...+a_nX^n$$
On obtient la fonction polynome associée : 
$$\forall x \in K,~ \widetilde{P}(x) = a_0+a_1x+...+a_nx^n$$
L'application qui lie le polynome à sa fonction associée est une bijection.\\
Toutes les notions de partié se transmette de la fonction polynome associée au polynome.
\subsection{Degrés}
\begin{de}
On défini le degrés d'un polynome par :
$$deg(P) = Max\left\lbrace k\in N / a_k \neq 0\right\rbrace $$
Par convention : 
$$deg(0) = -\infty$$
$$deg(P_{Constant}) = 0$$
\end{de}
\subsubsection{Degrés d'une combinaison}
On peut déterminer le degrés de deux combinaison :
$$deg(P.Q) = deg(P) + deg(Q)$$
$$deg(P+Q) \leq Max(deg(P),deg(Q))$$
\subsection{Valuation}
\begin{de}
On défini la valuation d'un polynome par :
$$val(P) = Min\left\lbrace k\in N / a_k \neq 0\right\rbrace $$
Par convention : 
$$deg(0) = +\infty$$
\end{de}
\subsubsection{Valuation d'une combinaison}
On peut déterminer le degres de deux combinaison :
$$val(P.Q) = val(P) + val(Q)$$
$$val(P+Q) \geq Min(val(P),val(Q))$$
\subsection{Division euclidienne dans K[X]}
\subsubsection{Diviseur, Multiple}
\begin{de}
Soient A,B deux polynomes.\\
On dit que B divise A, ou que A multiplie B, si :
$$\exists Q\in K[X]~ A=B.Q$$
Il en découle que les polynomes constant non nuls divisent tous les autres.
\end{de}
\subsubsection{Division euclidienne}
\begin{de}
Soit A,B deux polynomes, B non nul.\\
$$\exists!(R,Q) \in K[X]~ tq~ A = B.Q + R$$
avec deg(R) $\leq$ deg(Q)-1.\\
On appelle respectivement R et Q le reste et le quotient de la division euclidienne.
\end{de}
\subsection{Formule de Taylor}
Soit a un réel. Soit P un polynome de degrés n.\\
On obtient :
$$P = \sum_{k=0}^n \dfrac{\widetilde{D^k(P)}(a)}{k!}(X-a)^k$$
avec $\widetilde{D^k(P)}(a)$ la dérivé $k^{eme}$ de P prise en a.
\section{Racine d'un polynome}
Soit $P \in K[X]$. Soit r $\in K$
\subsection{Racine simple}
\begin{de}
r est une racine de P si $\widetilde{P}(r) = 0$
\end{de}
\begin{prop}
(r est une racine de P) $\Leftrightarrow$ ((X-r) divise P)
\end{prop}
\subsection{Racine multiple et ordre de multiplicité}
\begin{de}
r est une racine d'ordre $\alpha$ si $(X-r)^{\alpha}$ divise P et $(X-r)^{\alpha+1}$ ne divise pas P.
\end{de}
\begin{prop}
(r est une racine d'ordre $\alpha$ de P) $\Leftrightarrow$ ($\forall i\in \left\lbrace 0,..,\alpha-1\right\rbrace \widetilde{D^i(P)}(r) = 0$ et $\widetilde{D^{\alpha}(P)}(r) \neq 0$)
\end{prop}
\subsection{Polynome scindé}
Soit $P\in K[X]$ de degrés n et de termes dominant $a_nX^n$
\begin{de}
P est scindé si le nombre de racine, en comptant les ordres de multiplicité, est n : $$\exists(r_1,...,r_n)\in K^n, \exists(\alpha_1,...,\alpha_n) \in N^n~ tq~ P = a_n(X-r_1)^{\alpha_1}...(X-r_n)^{\alpha_n}$$
\end{de}
\subsubsection{Lien entre coefficiants et racine d'un polynome scindé}
On obtient les relations suivantes :
$$\sum_{i=1}^n r_i = \dfrac{-a_{n-1}}{a_n}$$
$$r_1...r_n = (-1)^n\dfrac{a_0}{a_n}$$
\subsection{Polynome irréductible}
\subsubsection{Dans $\mathbb{R}[X]$}
\begin{de}
Un polynome est irréductible si il n'est divisible que par les polynomes constant et par les produits de lui-meme par un constante.
\end{de}
\subsubsection{Dans $\mathbb{C}[X]$}
\begin{theo}
 Tout polynome non constant dans $\mathbb{C}[X]$ possède au moins une racine complexe.
\end{theo}
On en déduit donc que :
\begin{itemize}
 \item[$\rightarrow$] Tout polynomes dans $\mathbb{C}[X]$ est scindé
 \item[$\rightarrow$] Les seuls polynomes irréductible de $\mathbb{C}[X]$ sont ceux de degrés 1
\end{itemize}


\part{Espace vectoriel}
\chapter{Espace vectoriel}
\section{Définitions}
\begin{de}
Soit E un espace et K un corps. \\
Donc dit que (E,+,.) est un K-espace vectoriel si il vérifie les propriétés suivantes :
\begin{enumerate}[1-]
 \item + est une loi de composition interne : $$\forall(x,y) \in E^2~ x+y \in E$$
 \item + est une loi associative : $$\forall(x,y,z) \in E^3~ (x+y)+z = x+(y+z)$$
 \item + possède un élement neutre $O_E$ : $$\forall x \in E~ x+O_E = O_E+x =x $$
 \item Tous éléments x de E est symétrisable pour + dans E. Ce symétrique est $-x$ : $$\forall x \in E~ x+(-x) = (-x)+x = O_E$$
 \item + est commutatif dans E : $$\forall(x,y) \in E^2~ x+y=y+x$$
 \item "." est une loi de composition externe : $$\forall x \in E, \forall \lambda \in K,~ \lambda.x \in E$$
 \item "." possède un élement neutre $1_K$ : $$\forall x \in E~ 1_k.x = x$$
 \item "." vérifie : $$\forall x \in E, \forall(\lambda,\mu)\in K^2~ \lambda.(\mu.x) = (\lambda\times\mu).x$$
 \item "." vérifie : $$\forall x \in E, \forall(\lambda,\mu)\in K^2~ (\lambda+\mu).x = (\lambda.x)+(\mu.x)$$
  \item "." vérifie : $$\forall (x,y) \in E^2, \forall\lambda\in K~ \lambda.(x+y) = \lambda.x+\lambda.y$$
\end{enumerate}
Si un espace ne vérifie que les $4^{ere}$ propriétés, on dit que c'est un groupe. Si il vérifie les $5^{ere}$, c'est un groupe commutatif.
\end{de}
\begin{prop}
Soit E un K-espace vectoriel :\\
\begin{itemize}
 \item[$\rightarrow$] $\forall x \in E,~ 0_K.x = 0_E$\\
 \item[$\rightarrow$] $\forall x \in E~ -x = (-1_K).x$\\
 \item[$\rightarrow$] Soit x un vecteur de E, $\lambda \in K$ : $$(\lambda.x = O_E) \Leftrightarrow ( \lambda = O_K~ ou~ x=O_E)$$
\end{itemize}
\end{prop}

\section{Sous-espaces vectoriels}
\subsection{Définitions}
\begin{de}
Soit E un K-espace vectoriel. Soit F un espace.\\
On dit que F est un sous-espace de E si :
$$\left\{\begin{array}{l}
   F~ c~ E\\
   (F,+,.) \mbox{ est un K-espace vectoriel}
  \end{array}\right.$$
\end{de}
\begin{prop}
Soit E un K-espace vectoriel. Soient F et G deux sous espace de E :
\begin{itemize}
 \item[$\rightarrow$] $\{ O_E\}$ est le plus petit sous espace de E.
 \item[$\rightarrow$] $F\cap G$ est un sous espace de E
 \item[$\rightarrow$] $F\cup G$ est un sous espace de E 
\end{itemize}
\end{prop}
\subsection{Critère de reconaissance}
\begin{prop}
( F est un sous espace de E ) $\Leftrightarrow$ $\left\{\begin{array}{l}
   F~ c~ E\\
   F \neq \emptyset\\
   \forall(x,y) \in F^2, \forall (\lambda,\mu) \in K^2,~ \lambda x+\mu y \in F
  \end{array}\right.$
\end{prop}
\subsection{Sous espace supplémentaire}
Soit E un K-espace vectoriel. Soient F,G deux sous espace de E.
\begin{de}
 F et G sont dit en somme  direct si : $$F\cap G = \{ O_E \}$$
\end{de}
\begin{de}
 F et G sont supplementaire si ils sont en somme direct et que :  $$F + G = E$$
On le note : $$F \oplus G = E$$
\end{de}
\begin{prop}
Si F et G sont supplementaires, alors $$\forall x \in E$$, il existe un unique couple (x,y) avec $y\in F$,$z \in G$ telque : $$x = y+z$$
\end{prop}
\subsection{Partie génératrice d'un sous-espace}
\subsubsection{Sous espace engendré par un partie}
Soit A une partie de E.\\
Soit G le plus petit espace contenant A.
\begin{de}
G est le sous espace engendré par A. On le note : $$G = Vect(A)$$
On dit que A est une partie génératrice de G.
\end{de}
\subsubsection{Sous espace engendré par une partie fini}
Soit $u_1,...,u_n$ n vecteur de E. 
$$Vect(\{ u_1,...,u_n \}) = \{ \lambda_1u_1+...+\lambda_nu_n / \lambda_1,...,\lambda_n \in K^n \}$$
\subsection{Produit de deux espaces}
\begin{de}
Soient E et F deux K-espace vectoriel.\\
On munit le produit E$\times$F des deux lois suivant :
$$\forall(x,y) \in E\times F, \forall(x',y') \in E\times F, \forall \lambda \in K $$
$$\left\{\begin{array}{l}
   (x,y)+(x',y') = (x+x',y+y')\\
   \lambda.(x+y) = (\lambda x,\lambda y)
  \end{array}\right.$$
\end{de}
\begin{prop}
 (E$\times$F,+,.) est un K-espace vectoriel, de vecteur nul ($O_E,O_F$)
\end{prop}
\section{Application linéaire}
\begin{de}
Soient E et F deux K-espace vectoriels, f une application de E dans F.\\
f est une application linéaire si :
$$\forall(x,y) \in E^2~ \forall (\lambda,\mu) \in K^2~ f(\lambda x+\mu y) = \lambda f(x) + \mu f(y)$$
\end{de}
\subsection{Vocabulaire}
\begin{itemize}
 \item[$\rightarrow$] Application linéaire $\rightarrow$ Morphisme d'espace vectoriel
 \item[$\rightarrow$] Application linéaire de E dans E $\rightarrow$ Endomorphisme
 \item[$\rightarrow$] Application linéaire bijective $\rightarrow$ Isomorphisme
 \item[$\rightarrow$] Application linéaire bijective de E dans E $\rightarrow$ Automorphisme
\end{itemize}
On note $L(E,F)$ l'ensemble des applications linéaire de E dans F
\begin{prop}
Soit f isomorphisme de E dans F.\\
Alors $f^{-1}$ existe et est linéaire de F dans E
\end{prop}

\subsection{Noyau et Image d'une application linéaire}
Soit $f \in L(E,F)$
\subsubsection{Image}
\begin{de}
On appele image de f l'ensemble des images de tous les vecteurs de E par f :
$$Im(f) = \{f(x) / x\in E \}$$
Im(f) est un sous espace vectoriel de F
\end{de}
\subsubsection{Noyau}
\begin{de}
On appele noyau de f l'ensemble des antécédants $O_F$ par f :
$$Ker(f) = \{ x \in E / f(x) = 0 \}$$
Ker(f) est un sous espace vectoriel de E
\end{de}
\begin{prop}
f est une application injective si et seulement Ker(f) est réduit au vecteur nul :
$$(\mbox{f est injective}) \Leftrightarrow (Ker(f) = \{O_E\})$$
\end{prop}
\subsection{Opérations sur les applications linéaires}
\begin{itemize}
 \item[$\rightarrow$] La combinaison linéaire de deux applications linéaire est une application linéaire
 \item[$\rightarrow$] La composée de deux applications linéaire est linéaire
\end{itemize}
\subsection{Structure}
\begin{itemize}
 \item[$\rightarrow$] (L(E),+,o,.) est un K-Algèbre : On peut donc utiliser les identites remarquables
 \item[$\rightarrow$] GL(E) : Groupe des automorphisme de E. Dans ce groupe : $$(fog)^{-1} = g^{-1}of^{-1}$$
\end{itemize}
\subsection{Projecteur}
\begin{de}
Soit E un K-espace vectoriel, soient F et G deux sous espaces supplémentaire de E.\\
Soit $x \in E$
$$\exists !(y,z), y\in F, z\in G, x=y+z$$
On appelle projeté de x sur F parallement à G, notée p(x), le vecteur y.
\end{de}
\begin{prop}
p est une application linéaire
\end{prop}
\begin{prop}
Soit p la projection de F parallement à G :\\
 \begin{itemize}
 \item[$\rightarrow$] Im(p) = F\\
 \item[$\rightarrow$] Ker(p) = G\\
 \item[$\rightarrow$] pop = p\\
 \item[$\rightarrow$] $\forall x \in E$ (p(x) =x ) $\Leftrightarrow$ (x $\in$ F)\\
\end{itemize}
\end{prop}
\begin{prop}
Soit q la projection de G parallement à F. p et q sont deux projecteur associé.
 \begin{itemize}
 \item[$\rightarrow$] Im(q) = G\\
 \item[$\rightarrow$] Ker(q) = F\\
 \item[$\rightarrow$] poq = $O_E$\\
 \item[$\rightarrow$] p+q = Ide\\
\end{itemize}
\end{prop}
\subsubsection{Propriété caractéristique}
\begin{prop}
Si : 
$$\left\{\begin{array}{l}
   \mbox{ f est linéaire}\\
   fof = f
  \end{array}\right.$$
Alors f est une projection sur F parallement à G avec :
$$\left\{\begin{array}{l}
   F = \{ x\in E / f(x) = x\}\\
  G = Ker(f)
  \end{array}\right.$$
\end{prop}
\subsection{Symétrie}
\begin{de}
Soit E un K-espace vectoriel, soient F et G deux sous espaces supplémentaire de E.\\
Soit $x \in E$
$$\exists !(y,z), y\in F, z\in G, x=y+z$$
On appelle symétrie de x par rapport à F parallement à G :
$$s(x) = y-z$$
avec :
$$\left\{\begin{array}{l}
   (s(x) = x ) \Leftrightarrow ( x \in F)\\
   (s(x) = -x) \Leftrightarrow ( x \in G)
  \end{array}\right.$$
\end{de}
\begin{prop}
Soit s une symétrie :
\begin{itemize}
 \item[$\rightarrow$] s est une application linéaire
 \item[$\rightarrow$] sos = Ide, donc s est une bijection
\end{itemize}
\end{prop}
\subsubsection{Propriété caractéristique}
\begin{prop}
 Si : 
$$\left\{\begin{array}{l}
   \mbox{ f est linéaire}\\
   fof = Ide
  \end{array}\right.$$
Alors f est une symétrie
\end{prop}


\chapter{Espace vectoriel de dimensions finies}
\section{Partie libre - Partie liée - Partie génératrice}
\subsection{Partie finie liée}
\begin{de}
Soient $u_1,...,u_p$ p vecteurs d'un K-espace vectoriels de E.\\
On dit que $\{u_1,...,u_n\}$ est liée ou que les vecteurs $u_1,...,u_n$ sont linéairement dépendants si $\exists(\lambda_1,...,\lambda_p)\in K^p$ non tous nuls telque : $$\lambda_1u_1+...+\lambda_pu_p = O_E$$
\end{de}
\begin{prop}
Toutes parties qui contient le vecteur nul est liée
\end{prop}
\begin{prop}
Si L est liée, et L c L', alors L' est liée.
\end{prop}
\subsubsection{Vecteurs colinéaires}
Soit u,v deux vecteurs de E.
$$(\mbox{ u et v sont colinéaire} \Leftrightarrow (\exists \lambda \in K~ tq~ u=\lambda v~ ou~ v=0))$$
\subsection{Partie fini libre}
\begin{de}
 Soit L partie finie de E.
$$(\mbox{L est libre}) \Leftrightarrow (\mbox{L n'est pas libre})$$
On la caractérise par : Si $\exists(\lambda_1,...,\lambda_p)\in K^p$ tq $\lambda_1u_1+...+\lambda_pu_p = O_E$ alors $$\lambda_1=...=\lambda_p=0$$
On dit que la partie est linéairemement indépendante.
\end{de}
\begin{prop}
Si L est libre, et L' c L, alors L' est aussi libre.
\end{prop}
\subsection{Partie génératrice}
Soit E un K espace vectoriel.
Nous avons l'ensemble des propriétés suivantes :
\begin{enumerate}[1-]
 \item G est une partie génératrice de E si : $$Vect(G)~ c~ E$$
 \item Si A et B sont deux parties de E, si A c B, alors : $$Vect(A)~ c~ Vect(B)$$
 \item Si G est une partie génératrice de E, G' une partie telque G c G', alors G' est une partie génératrice de E
\end{enumerate}
\subsection{Base}
\begin{de}
Une base d'un espace E est une partie génératrice de E et libre.
\end{de}
\begin{prop}
Si B = {$e_1,...,e_n$} est une base fini de E, alors, $\forall u \in E$, $\exists !$ n uplet de scalaire ($x_1,...,x_n$) telque : $$u = x_1e_1+...+x_ne_n$$
Le n-uplet ($x_1,...,x_n$) est appelé le n-uplet de coordonées de u dans la base B. On le note aussi : 
$$mat_B(u) = \begin{bmatrix}
 x_1 \\
  . \\
  . \\
 x_n \\
\end{bmatrix}$$
\end{prop}
\subsubsection{Base de référence}
\begin{itemize}
 \item[$\rightarrow$] $\mathbb{R}_n[X]$ : $\{$1,X,$X^2$,...,$X^n$$\}$\\
 \item[$\rightarrow$] $\mathbb{R}^n$ : $\{$(1,...,0),...,(0,...,1)$\}$
\end{itemize}
\section{Dimension d'un espace de dimension finie}
\begin{de}
Soit E un K-espace vectoriel.\\
Si E possède une partie génératrice finie, on dit que E est un espace de dimension finies.
\end{de}
\begin{prop}
Soit E un espace de dimension finies et G = $\{ u_1,...,u_p \}$ une partie génératrice de E.\\
Si G est libre, alors G est une base de E.
\end{prop}
\begin{prop}
De toute partie génératrice finie, on peut extraire une base.
\end{prop}
\begin{theo}
Théorème de la base incomplète :\\
Soit L = $\{ u_1,...,u_n \}$.\\
Si L est une partie libre, on peut la completer en une base.
\end{theo}
\begin{prop}
Lemme de Steiniz :\\
Si E possède une partie génératrice de n vecteurs, alors toute partie de n+1 vecteurs est liée
\end{prop}
\begin{theo}
 Si E est de dimension finie, toutes les bases de E ont le meme nombre d'élements et ce nombre commun est la dimension de l'espace.\\
Si B est une base de E :
$$dim(E) = card(B)$$
\end{theo}
\begin{prop}
Soit E un espace de dimension n, soit A une partie de E 
\begin{itemize}
 \item[$\rightarrow$] Si A est une partie génératrice de E, alors card(A) $\geq$ n
 \item[$\rightarrow$] Si card(A) < n, alors A n'est pas génératrice de E
 \item[$\rightarrow$] Si A est libre, alors card(A) $\leq$ n
 \item[$\rightarrow$] Si card(A) > n, alors A est liée.
\end{itemize}
\end{prop}
\subsection{Caractérisation des bases}
Soit E un espace vectoriel de dimension n et A une partie de E.\\
Si A est une base, alors A est libre, A est génératrice de E et card(A) = dim(E).\\
\begin{itemize}
 \item[$\rightarrow$] A est une base $\Leftrightarrow$ A est libre et card(A) = dim(E)
 \item[$\rightarrow$] A est une base $\Leftrightarrow$ A est génératrice de E et card(A) = dim(E)
\end{itemize}
\section{Sous-espace d'un espace de dimension finie}
Soit E un K-espace vectoriel de dimension finie :
\begin{prop}
Soit F sous espace de E. \\
F est de dimension finie et dim(F) $\leq$ dim(E).
\end{prop}
\begin{prop}
Soient F et G deux sous-espace de E :
$$(F = G) \Leftrightarrow \left\{\begin{array}{l}
   F~ c~ G\\
   dim(F) = dim(G)
  \end{array}\right.$$
\end{prop}

\begin{prop}
Formule de Grassman : \\
Soit F et G deux sous-espace de E :
$$dim(F+G) = dim(F) + dim(G) - dim(F\cap G)$$
\end{prop}
\begin{prop}
 Soit E espace de dimension finie.\\
Soient F et G deux sous espaces de E. F et G sont supplémentaire si :
$$\left\{\begin{array}{l}
   F+G = E \\
   dim(E) = dim(F) + dim(G)
  \end{array}\right.$$
\end{prop}
\begin{prop}
Tous sous-espace possède au moins un supplémentaire
\end{prop}
\begin{prop}
\begin{itemize}
 \item[$\rightarrow$]dim($\emptyset$) = 0
 \item[$\rightarrow$]Si D est un sous espace de dimension 1, c'est une droite vectorielle
 \item[$\rightarrow$]Si P est un sous espace de dimension 2, c'est un plan vectoriel
 \item[$\rightarrow$]Si H est un sous espace de dimension n-1, c'est un hyperplan
 \item[$\rightarrow$]Tout supplémentaires d'un hyperplan est une droite vectoriel
\end{itemize}
\end{prop}
\subsection{Rang d'une partie}
\begin{de}
Soit A une partie d'un espace E.\\
Le rang de A est la dimension du sous espace engendré par A :
$$rang(A) = dim(Vect{A})$$
\end{de}
\begin{prop}
Soit A c E : 
\begin{itemize}
 \item[$\rightarrow$] (rang(A) = dim(E)) $\Leftrightarrow$ (A est une partie génératrice de E)
 \item[$\rightarrow$] (A est libre) $\Leftrightarrow$ ($rang(A) = card(A)$)
\end{itemize}
\end{prop}
\subsection{Sous espace supplémentaire et base}
Soient F,G deux sous espaces de E, de base $B_F,B_G$ :
$$( \mbox{ F et G sont supplémentaire }) \Leftrightarrow \left\{\begin{array}{l}
  B_F \cup B_G = B_E \\
   B_F \cap B_G = \emptyset
  \end{array}\right.$$
\section{Application linéaire entre deux espaces de dimension finies}
Soient E et F deux K-espaces vectoriel de dimension finies
\subsection{Caractérisation par l'image d'une base de E}
Soit $B_E$ = {$e_1,...,e_p$} une base de E.\\
Soit u un vecteur de E telque :$$mat_B(u) = \begin{bmatrix}
 x_1 \\
  . \\
  . \\
 x_p \\
\end{bmatrix}$$
Si f $\in$ L(E,F), alors :
$$f(u) = x_1f(e_1)+...+x_pf(e_p)$$
\subsection{Image d'une partie libre, liée ou génératrice de E}
Soit f $\in$ L(E,F).\\
\begin{itemize}
 \item[$\rightarrow$]Si $L_1$ = {$u_1,...,u_p$} est une partie liée, alors {$f(u_1),...,f(u_p)$}\\
 \item[$\rightarrow$]Si $L_1$ = {$u_1,...,u_p$} est une partie libre, alors ??????\\
 \item[$\rightarrow$]L'image d'une partie génératrice de E est une partie génératrice de Im(f)
 \item[$\rightarrow$]Si B est une base de E, alors f(B) est génératrice de Im(f)
\end{itemize}
\subsection{Rang d'une application linéaire}
\begin{de}
Soit f $\in$ L(E,F).\\
Le rang de f est la dimension de l'image de f :
$$rang(f) = dim(Im(f))$$
\end{de}
\begin{prop}
(f est surjective) $\Leftrightarrow$ (rang(f) = dim(F))
\end{prop}
\subsection{Théorème du rang}
Soit f $\in$ L(E,F) :
$$dim(Ker(f)) + rang(f) = dim(E)$$
\begin{prop}
(f est injective) $\Leftrightarrow$ (rang(f) = dim(E))
\end{prop}
\subsection{Forme linéaire}
\begin{de}
Une forme linéaire d'un espace est une application linéaire de E dans son corps K.
\end{de}
\begin{prop}
Si $\varphi$ est une forme linéaire : $rang(\varphi) \leq 1$
\end{prop}

\begin{prop}
Le noyau d'une forme linéaire non nuls est un hyperplan
\end{prop}
\section{Isomorphisme}
\begin{de}
On considère E,F deux espaces de dimension finie.\\
On dit que E et F sont isomorphe si il existe un isomorphisme de E dans F.\\
Dans cette situation, on obtient :
\begin{itemize}
 \item[$\rightarrow$] Ker(f) = {$O_E$}
 \item[$\rightarrow$] Im(f) = F
 \item[$\rightarrow$] rang(f) = dim(F) = dim(E)
 \item[$\rightarrow$] Deux espaces isomorphe ont meme dimension
 \item[$\rightarrow$] Soit $B_E$ une base de E, f un isomorphisme, alors f($B_E$) est une base.
\end{itemize}
\subsection{Caractérisation des isomorphismes}
Soit $\varphi$ une application linéaire de E dans F :
\begin{prop}
 $$( \varphi~ est~ bijective) \Leftrightarrow \left\{\begin{array}{l}
  Ker(\varphi) = \{O_E\} \\
   dim(E) = dim(F)
  \end{array}\right.$$
 $$( \varphi~ est~ bijective) \Leftrightarrow \left\{\begin{array}{l}
  rang(\varphi) = dim(F) \\
   dim(E) = dim(F)
  \end{array}\right.$$
 $$( \varphi~ est~ bijective) \Leftrightarrow (\varphi(B_E) \mbox{ est une base de F})$$
\end{prop}
\end{de}
\subsection{Espace isomorphe}
\begin{theo}
Si F est un espace de dimension finie, si E et F sont isomorphe, alors E est aussi de dimension finie, et dim(E) = dim(F)
\end{theo}

\part{Espace vectoriel euclidien}

\chapter{Espaces vectoriels euclidiens}
\section{Produit scalaire}
\begin{de}
Soit E un $\mathbb{R}-$espace vectoriel et :
$$\varphi : E\times E \rightarrow \mathbb{R}$$
$\varphi$ est un produit scalaire sur E si :
\begin{itemize}
 \item[$\rightarrow$] $\varphi$ est bilinéaire
 \item[$\rightarrow$] $\varphi$ est symétrique
 \item[$\rightarrow$] $\varphi$ est positive
 \item[$\rightarrow$] $\varphi$ est définie
\end{itemize}
\end{de}
\subsection{Notation et Vocabulaire}
\begin{itemize}
\item[$\rightarrow$] Si $\varphi$ est un produit scalaire sur E, on note : $$\forall(u,v) \in E^2~ \varphi(u,v) = <u,v>$$
 \item[$\rightarrow$] On défini la norme de u par : $$\forall u \in E~ ||u|| = \sqrt{<u,u>}$$
 \item[$\rightarrow$] Sachant que $\varphi$ est définie, on obtient : $$\forall u \in E,~ (||u||=0) \Leftrightarrow (u=0)$$
 \item[$\rightarrow$] On dit que u et v sont orthogonaux si : $$<u,v>=0$$
 \item[$\rightarrow$] Un espace vectoriel est dit euclidien si : \begin{enumerate}[1-]
 \item E est de dimension finies 
 \item On a défini un produit scalaire sur E
\end{enumerate}
\end{itemize}
\section{Propriétés}
\begin{prop}
Soit E un $\mathbb{R}$-espace euclidien.\\
Soient u,v deux vecteurs de E : 
$$||u+v||^2 = ||u||^2 + 2<u,v> + ||v||^2$$
\end{prop}
\begin{theo}
Théorème de Pythagore : $$(\mbox{u et v sont orthogonaux}) \Leftrightarrow (||u+v|| = ||u||^2+||v||^2)$$
\end{theo}
\begin{prop}
 Soit $u \in E$, soit $\lambda \in \mathbb{R}$
$$||\lambda u|| = |\lambda|.||u||$$
\end{prop}

\begin{prop}
Inégalité de Cauchy : \\
Soient u,v deux vecteurs :
$$|<u,v>| \leq ||u||\times||v||$$
Si il y a égalité, alors u et v sont colinéaire
\end{prop}

\begin{prop}
Inégalité de Minkouskay :
$$||x+y|| \leq (||x||+||y||)$$
\end{prop}
\begin{prop}
Si $\{u_1,...,u_n\}$ sont des vecteurs non nuls et 2 à 2 orthogonaux, alors la partie est libre.
\end{prop}
\section{Base orthonormée}
\begin{de}
Soit ($e_1,...,e_n$) n vecteur de E, avec E espace de dimension n, deux à deux ortogonaux (famille orthogonale) et unitaire (famille normée) ( $\forall k~ ||e_k|| = 1$).\\
Alors,($e_1,...,e_n$) est une famille dites orthonormée, qui, de plus, est ici une base.
\end{de}
\begin{prop}
Tout $\mathbb{R}$ espace vectoriel euclidien de dimension finie admet au moins une base orthonormée.
\end{prop}
\begin{prop}
Soit B une base orthonormée de E. Soient u,v deux vecteurs de E de coordonnée respectif ($x_1,...,x_n$) et ($y_1,...,y_n$), alors :
$$<u,v> = x_1y_1+...+x_ny_n$$
\end{prop}
\begin{prop}
On défini dans ce cas la norme de u par : 
$$||u|| = \sqrt{x_1^2+...+x_n^2}$$
\end{prop}
\begin{prop}
Pour déterminer les coordonnées dans une base orthonormée, on détermine : 
$$\forall k \in \{1,...,p\}~ <u,e_k> = x_k$$
\end{prop}
\subsection{Matrice orthogonales}
\begin{de}
Une matrice de passage entre deux bases orthonormées est dites orthogonale.
\end{de}
\begin{prop}
 Soit B une base orthonormée. Soit B' une autre base. Soit P = $mat_B(B')$.
$$(\mbox{B' est orthogonale}) \Leftrightarrow (P^{-1} = ^tP)$$
\end{prop}
\begin{prop}
Soit P une matrice orthogonale :
$$det(P) = \pm 1$$
\end{prop}
\begin{prop}
Si P est orthogonale, alors $^tP$ est orthogonale et ($l_1,...,l_n$) forme aussi une base orthonormée de $\mathbb{R}^n$
\end{prop}
\subsection{Orientation de l'espace vectoriel}
\begin{de}
Soit E un espace vectoriel.\\
On oriente une base en définisant une base dites directe.
\end{de}
\begin{prop}
Soit $B_0$ une base directe.\\
Si B est une base de E : 
\[\left\{\begin{array}{l}
   det_{B_0}(B) > 0 \mbox{ B est directe}\\
   det_{B_0}(B) < 0 \mbox{ B est indirecte}\\
  \end{array}\right.\]
\end{prop}
\begin{prop}
Soit B,B' deux bases de E :
$$(det_B(B') > 0) \Leftrightarrow (\mbox{ B et B' ont la même orientation})$$
Dans le cas des bases orthonormée, on as : 
\end{prop}
\subsubsection{Bases orthonormée}
\begin{prop}
Soit $B_1$ une base orthonormée directe
\[\left\{\begin{array}{l}
   det_{B_1}(B) = 1 \mbox{ alors B est orthonormée directe}\\
   det_{B_1}(B) = -1 \mbox{ alors B est orthonormée indirecte}\\
  \end{array}\right.\]
\end{prop}
\begin{prop}
Soit $(u_1,...,u_n) \in \mathbb{R}^n$.\\
Si B et B' sont deux bases orthonormée directe : 
$$det_B(u_1,...,u_n) = det_{B'}(u_1,...,u_n)$$
Ce déterminant commun à toutes les bases orthonormée directe est notée Det($u_1,...,u_n$)
\end{prop}
\subsection{Orthogonalité et sous-espace}
Soit E un espace euclidien
\subsubsection{Sous espace orthogonaux}
\begin{de}
Soient F et G deux sous-espaces.
$$(\mbox{F est orthogonal à G})\Leftrightarrow (\forall x \in F,\forall y \in G,~ <x,y>=0)$$
\end{de}
\begin{prop}
Deux sous espaces orthogonaux sont en somme directe.
\end{prop}
\begin{prop}
On en déduit que : 
$$dim(F)+dim(G) \leq dim(E)$$
\end{prop}
\subsubsection{Orthogonal d'un sous-espace}
\begin{de}
Soit F un sous espace de E.\\
On appele orthogonale de F l'ensemble des vecteurs de E qui sont orthogonaux à tous les vecteurs de F.\\
On note :
$$F^{\bot} = \{x\in E / \forall y \in F~ <x,y>=0\}$$
\end{de}
\begin{prop}
$F^{\bot}$ est un espace vectoriel.
\end{prop}
\begin{prop}
F et $F^{\bot}$ sont deux sous espaces supplémentaire
\end{prop}
\begin{prop}
On en déduit que :
$$dim(F^{\bot})=dim(E) - dim(F)$$
\end{prop}
\begin{prop}
L'orthogonale de l'orthogonale de F :
$$(F^{\bot})^{\bot} = F$$
\end{prop}
\begin{prop}
Soit $\varphi \in L(E,\mathbb{R})$.\\
Soit u un vecteur de E de coordonée $(x_1,...,x_n)$. Il existe donc ($a_1,...,a_n$) telque :
$$\varphi(u) = a_1x_1+...+a_nx_n$$
On obtient donc : 
$$\varphi(u) = <a,u>$$
Par conséquence :
$$Ker(\varphi) = (Vect(a))^{\bot}$$
\end{prop}
\subsection{Projection orthogonale}
\begin{de}
La projection orthogonale sur un sous espace vectoriel F est la projection sur F parallèlement à $F^{\bot}$
\end{de}
\begin{de}
Soit B($e_1,...e_p$) une base orthonormée de F.\\
Soit x un vecteur de E de coordonnées ($x_1,...,x_p$).\\
On obtient donc le projetté orthogonale de x sur F, notée p(x) : 
$$p(x)=<x,e_1>e_1+...+<x,e_p>e_p$$
\end{de}
\begin{prop}
Si p est la projection orthogonale sur F. Si $u \in E$, alors :
$$Inf\{||x-y|| / y \in F \} = ||x - p(x)||$$
Et on note : 
$$d(x,F) = \underset{y \in F}\inf \parallel x - y\parallel$$
Et on appelle ceci distance de x à F.
\end{prop}
\begin{prop}
Si : 
\begin{itemize}
 \item[$\rightarrow$] $y \in F$
 \item[$\rightarrow$] $x - y \in F^{\bot}$
\end{itemize}
Alors y = p(x).
\end{prop}



\chapter{Espace euclidien de dimension 3}
\section{Définitions}
\section{Angle de deux vecteurs non nuls}
Soient $\overrightarrow{u},\overrightarrow{v}$ deux vecteurs non nuls, non colinéaire.\\
Soit $\overrightarrow{n}$ vecteur orthogonale à $\overrightarrow{u}$ et $\overrightarrow{v}$.
On obtient : 
$$\left\{\begin{array}{l}
   cos(\theta) = \dfrac{<\overrightarrow{u};\overrightarrow{v}>}{||\overrightarrow{u}||.||\overrightarrow{v}||}\\
   et~ : \\
   sin(\theta) = \dfrac{Det(\overrightarrow{u};\overrightarrow{v};\overrightarrow{n})}{||\overrightarrow{u}||.||\overrightarrow{v}||.||\overrightarrow{n}||}
  \end{array}\right.$$
\subsection{Produit vectoriel}
\begin{de}
 Soit ($\overrightarrow{i},\overrightarrow{j},\overrightarrow{k}$) une base orthonormée directe de E.\\
Le produit vectoriel est l'unique application de E$\times$E dans E :\\
\begin{itemize}
 \item Alternée : $\overrightarrow{u}\wedge\overrightarrow{v} = -\overrightarrow{v}\wedge\overrightarrow{u}$
 \item Bilinéaire
 \item Vérifiant : 
$\left\{\begin{array}{l}
\overrightarrow{i}\wedge\overrightarrow{j} = \overrightarrow{k}\\
\overrightarrow{j}\wedge\overrightarrow{k} = \overrightarrow{i}\\
\overrightarrow{k}\wedge\overrightarrow{i} = \overrightarrow{j}\\  
\end{array}\right.$
\end{itemize}
La définition du produit vectoriel est indépendante du choix de la base orthonormée.
\end{de}
\begin{prop}
Soit $\overrightarrow{u},\overrightarrow{v}$ deux vecteurs :
$$(\overrightarrow{u}\wedge\overrightarrow{v}=\overrightarrow{0}) \Leftrightarrow ( \overrightarrow{u}~ et~ \overrightarrow{v} \mbox{ sont colinéaire})$$
\end{prop}
\begin{prop}
Soient $\overrightarrow{u},\overrightarrow{v}$ deux vecteurs non colinéaire : 
$$\overrightarrow{u}\wedge\overrightarrow{v}\bot\overrightarrow{u}$$
\end{prop}
\begin{prop}
On obtient la propriété suivante, pour la norme du produit vectoriel : 
$$||\overrightarrow{u}\wedge\overrightarrow{v}|| = ||\overrightarrow{u}||.||\overrightarrow{v}||.|sin(\theta)|$$
\end{prop}
\subsection{Double produit vectoriel}
\begin{prop}
Soient $\overrightarrow{a},\overrightarrow{b},\overrightarrow{c}$ trois vecteurs :
$$\overrightarrow{a}\wedge(\overrightarrow{b}\wedge\overrightarrow{c}) = \overrightarrow{b}.(\overrightarrow{a}.\overrightarrow{c}) - \overrightarrow{c}.(\overrightarrow{a}.\overrightarrow{b})$$ 
\end{prop}

\subsection{Produit mixte}
\begin{de}
 Soient $\overrightarrow{u},\overrightarrow{v},\overrightarrow{w}$ trois vecteurs de E :
$$\overrightarrow{u}\wedge\overrightarrow{v}.\overrightarrow{w} = Det(\overrightarrow{u},\overrightarrow{v},\overrightarrow{w})$$
\end{de}


\chapter{Isométrie Vectorielle}
\section{Généralités}
\begin{de}
Soit E un espace euclidien, soit f $\in L(E)$.\\
f est une isométrie vectorielle si : 
$$\forall x \in E~ ||f(x)||=||x||$$
\end{de}
\begin{prop}
Soit f $\in L(E)$. 
$$(\mbox{f est une isométrie vectorielle}) \Leftrightarrow (\forall x \in E~ \forall y \in E,~ <f(x),f(y)>=<x,y>)$$
\end{prop}
On dit que f est un endomorphisme orthogonale
\begin{prop}
Une isométrie vectorielle est bijective
\end{prop}
\begin{prop}
La réciproque d'une isométrie vectorielle est une isométrie vectorielle
\end{prop}
\begin{prop}
 La composée de deux isométries vectorielles est une isométrie vectorielle
\end{prop}
\begin{prop}
L'ensemble des isométrie vectorielles, munie de la loi de composition des applications est un groupe, appelé le groupe orthogonale de E, notée O(E)
\end{prop}
\begin{prop}
Soit f endomorphisme de E, et B base orthonormée.\\
$$(\mbox{f est une isométrie vectorielle}) \Leftrightarrow (\mbox{ f(B) est une base orthonomée})$$
$$(\mbox{f est une isométrie vectorielle}) \Leftrightarrow (mat_B(f) \mbox{ est une base orthogonale})$$
\end{prop}
\section{Isométrie vectorielle plane}
\subsection{Classification}
\begin{center}
% use packages: array
\begin{tabular}{|l|l|l|l|}
\hline
 Nom &Matrice & Déterminant & Vecteur invariant \\ \hline
Rotation d'angle $\theta$ & $\begin{bmatrix}
 \cos(x) & -\sin (x) \\
  \sin (x) & \cos(x) \\
\end{bmatrix}$  & +1 & $\{0\}$ ou E pour Ide \\ \hline
Symétrie $\bot$ / à une droite D & $\begin{bmatrix}
 \cos(x) & \sin (x) \\
  \sin (x) & -\cos(x) \\
\end{bmatrix}$  & -1 & Droite vectorielle \\ \hline
\end{tabular}
\end{center}
\subsection{Cas particulier des rotations}
\begin{prop}
L'ensemble des rotations vectorielle planes est un sous groupe de O(E), appelé groupe spéciale orthogonale. Il est notée SO(E)
\end{prop}
\subsection{Rotation orthogonales}
\begin{prop}
La composée de deux symétries orthogonales par rapport à deux droites est une rotation d'angle deux fois l'angle entre les deux droites.
\end{prop}
\section{Isométrie vectorielle d'un espace euclidien de dimension 3}
\subsection{Symétrie orthogonale}
Soit E un espace euclidien de base B=(i,j,k) orthonormée directe.\\
Soit F un sous espace de E, et $s_f$ la symétrie orthogonale par rapport à F : 
\begin{center}
% use packages: array
\begin{tabular}{|l|l|l|l|l|}
\hline
Définition de F & Base & Matrice & Déterminant & Type \\ \hline
F = $\{0\}$ & Quelconque & $\begin{bmatrix}
 -1 & 0 &  0\\
  0 & -1 & 0\\
 0 & 0 & -1 \\
\end{bmatrix}$ & -1 & $s_f$ = $h_{-1}$ \\ \hline
dim(F) = 1 & ($e_1,e_2,e_3$) D =Vect($e_1$)& $\begin{bmatrix}
 1 & 0 &  0\\
  0 & -1 & 0\\
 0 & 0 & -1 \\
\end{bmatrix}$& 1 & × \\ \hline
dim(F) = 2 & ($e_1,e_2,e_3$) P =Vect($e_1,e_2$) &  $\begin{bmatrix}
 1 & 0 &  0\\
  0 & 1 & 0\\
 0 & 0 & -1 \\
\end{bmatrix}$& -1 & Réflection \\ \hline
dim(F) = 3 & Quelconque & $\begin{bmatrix}
 1 & 0 &  0\\
  0 & 1 & 0\\
 0 & 0 & 1 \\
\end{bmatrix}$ & 1 & $s_f = Ide$ \\ \hline
\end{tabular}
\end{center}
\subsection{Propriété de la matrice d'un symétrie orthogonale dans une base orthonormée}
Si B est une base orthonormée et s une symétrie orthogonale.\\
Soit M=$mat_B(s)$.\\
Sachant que s est une symétrie, M est inversible et $M^{-1} = M$. De plus, la symétrie est orthogonale, donc la matrice l'est aussi, donc $M^{-1} = ^tM$.\\
On obtient donc M = $^tM$. Donc M est une symétrie.
\subsection{Rotation}
\begin{de}
Soit D une droite vectorielle orienté et $\theta$ un réel.\\
La rotation d'axe de D et d'angle $\theta$ est l'application linéaire r telque :
\begin{itemize}
 \item[$\rightarrow$]$\forall u \in D$ r(u)=u
 \item[$\rightarrow$]Le plan $D^{\bot}$ est stable par r et la restriction de r à ce plan est une rotation plane d'angle $\theta$ orienté par D.
\end{itemize}
\end{de}
\begin{prop}


Soit $B_=(e_1,e_2,e_3)$ base orthonormée directe telle que D=Vect($e_1$) et $D^{\bot}=Vect(e_2,e_3)$, alors : 
$$\begin{bmatrix}
 1 & 0 &  0\\
  0 & cos(\theta) & -sin(\theta)\\
 0 & sin(\theta) & cos(\theta) \\
\end{bmatrix}$$
Donc : $$det(r) = 1$$
 
\end{prop}
\begin{prop}
Cas particulier :
\begin{itemize}
 \item[$\rightarrow$] $\theta = 0$, alors r = Ide
 \item[$\rightarrow$] $\theta = \pi$, alors r=$s_D$
\end{itemize}
\end{prop}
\begin{prop}
Composée de deux réflections :\\
Soient P,P' deux plans distincts telque $P\cap P' = D$ :
\begin{itemize}
 \item[$\rightarrow$] Si $u \in D$ : $s_{P'}(s_P(u))=u$
 \item[$\rightarrow$] Si $u \in D^{\bot}$ r est stable dans $D^{\bot}$
  \item[$\rightarrow$] $s_{P'}o s_P(u)$ est une rotation d'axe D.
\end{itemize}
\end{prop}
\subsection{Calcul de l'image d'un vecteur de x par une rotation}
Soit r rotation d'axe D, orienté par $\overrightarrow{d}$, vecteur directeur de D, et d'angle $\theta$.\\
Soit $\overrightarrow{x} \notin D$ : 
$$r(\overrightarrow{x}) = \dfrac{<\overrightarrow{x},\overrightarrow{d}>}{||d||^2}.\overrightarrow{d} + cos(\theta)\dfrac{(\overrightarrow{d}\wedge\overrightarrow{x})\wedge\overrightarrow{d}}{||\overrightarrow{d}||^2}+sin(\theta)\dfrac{\overrightarrow{d}\wedge\overrightarrow{x}}{||\overrightarrow{d}||}$$
Si $||\overrightarrow{d}|| = 1$ : 
$$r(\overrightarrow{x}) = <\overrightarrow{x},\overrightarrow{d}>.\overrightarrow{d} + cos(\theta)(\overrightarrow{d}\wedge\overrightarrow{x})\wedge\overrightarrow{d}+sin(\theta)\overrightarrow{d}\wedge\overrightarrow{x}$$
\subsection{Classification}
Soit Inv(f) l'ensemble des vecteurs invariants.
\begin{center}
% use packages: array
\begin{tabular}{|l|l|l|l|l|}
\hline
dim(Inv(f)) & f & Base & Matrice & Det \\ \hline
3 & Ide & Quelconque & $\begin{bmatrix}
 1 & 0 &  0\\
  0 & 1 & 0\\
 0 & 0 & 1 \\
\end{bmatrix}$ & 1 \\ \hline
2 & Réflexion $s_p$ & Base adaptée & $\begin{bmatrix}
 1 & 0 &  0\\
  0 & 1 & 0\\
 0 & 0 & -1 \\
\end{bmatrix}$ & -1 \\ \hline
1 & Rotation d'axe D, d'angle $\theta$ & Base adaptée & $\begin{bmatrix}
 1 & 0 &  0\\
  0 & cos(\theta) & -sin(\theta)\\
 0 & sin(\theta) & cos(\theta) \\
\end{bmatrix}$ & 1 \\ \hline
0 & Réflexion o rotation  & Base adaptée & $\begin{bmatrix}
 1 & 0 &  0\\
  0 & cos(\theta) & -sin(\theta)\\
 0 & sin(\theta) & cos(\theta) \\
\end{bmatrix}$ & -1 \\ \hline
\end{tabular}
\end{center}
\subsection{Élements caractéristiques d'une rotation}
\begin{itemize}
 \item[$\rightarrow$] L'axe : L'ensemble des vecteurs invariante
 \item[$\rightarrow$] L'angle : 
 \begin{itemize}
 \item[$\rightarrow$] cos($\theta$) est obtenu par la Trace(r) = 1 + 2cos($\theta$) dans une base adaptée.
 \item[$\rightarrow$] Le signe de sin($\theta$) est obtenu par le signe de Det(x,r(x),d), avec x espace non invariant de l'espace, r(x) la rotation et d un vecteur directeur de l'axe
\end{itemize}
\end{itemize}
\subsection{Autres résultats}
\begin{prop}
La matrice d'une projection orthogonale dans une base orthonormée est symétrique
\end{prop}

\part{Espace Affine}

\chapter{Espace Affine}
\section{Définitions}
\begin{de}
Soit E un $\Re$ espace vectoriel.\\
Soit $\xi$ un ensemble.\\
On dit que $\xi$ est un espace affine de direction l'espace vectoriel E si il existe $\varphi$ défini par : 
$$\varphi : \xi \times \xi \rightarrow E$$
$$(a,b) \mapsto \overrightarrow{ab}$$
telle que : 
\begin{itemize}
 \item[$\rightarrow$] $\forall(A,B,C)\in \xi^3~ \varphi(A,B)+\varphi(B,C) = \varphi(A,C)$
 \item[$\rightarrow$] $\forall A \in \xi,$ $\forall u \in E,~ \exists ! B \in \xi$ tq $\overrightarrow{AB}=\overrightarrow{u}$
\end{itemize}
\end{de}
\begin{voc}
Si $\xi$ est un espace affine, ses éléments sont appelé points.
\end{voc}
\begin{voc}
Si B est une base de E, O$\in \xi$, alors (O,B) est un repère de $\xi$
\end{voc}
\begin{voc}
Si M$\in \xi$, les coordonées de M dans (O,B) sont celles de $\overrightarrow{OM}$ dans B.
\end{voc}
\begin{prop}
$\Im$ est un sous espace affine de $\xi$ si :
\begin{itemize}
 \item[$\rightarrow$] $\Im$ = $\emptyset$
 \item[$\rightarrow$] ou $\exists$ A $\in \xi$ et F sous espace vectoriel de E telque : $$\Im = A + F$$
\end{itemize}
\end{prop}
\section{Applications affines}
\begin{de}
Soit $\xi$ un espace affine, E un espace vectoriel.\\
On appelle application affine de $\xi$ toute application f de $\xi$ dans $\xi$ telle qu'il existe $\varphi \in L(E)$ et O,O' deux points telque :
$$\forall M \in \xi~ \overrightarrow{O'f(M)} = \varphi(\overrightarrow{OM})$$
$$\forall M \in \xi~ f(M) = O'+\varphi(\overrightarrow{OM})$$
On dit que $\varphi$ est l'application linéaire associée à f.
\end{de}
\begin{prop}
Si f est une application affine associée à $\varphi$ : 
$$\forall (A,B) \in \xi^2~ \overrightarrow{f(A)f(B)} = \varphi(\overrightarrow{AB})$$
\end{prop}
\begin{prop}
 La composée de deux applications affines est affine et l'application linéaire associée est la composée des applications linéaires associées.
\end{prop}
\subsection{Homothétie affine}
\begin{de}
Soit A $\in \xi$ et $k \in \Re$.\\
On appelle homothétie de centre A et de rapport k l'application :
$$h : \xi \rightarrow \xi$$
$$M \mapsto M'$$
avec $\overrightarrow{AM'} = k\overrightarrow{AM}$.\\
h est une application affine.
\end{de}
\subsection{Conservation du barycentre}
\begin{prop}
Si f est l'application affine associée à $\varphi$, et G le barycentre de $\{(M_i,\lambda_i)/ i=1,...,p\}$, alors :\\
f(G) est le barycentre de $\{(f(M_i),\lambda_i)/i=1,..,p\}$
\end{prop}
\subsection{Expression analytique dans un repère}
\begin{de}
Soit f application affine de E, et R=(O,B) un repère de $\xi$.\\
Il existe des réels $a_{i,j},b_i$ telque si M à pour coordonnées ($x_1,...,x_n$) et M' à pour coordonées ($x'_1,...,x'_n$) :
\[\left\{\begin{array}{l}
   x'_1 = a_{1,1}x_1+...+a_{1,n}x_n+b_1  \\
   ....\\
   ....\\
   x'_n = a_{n,1}x_1+...+a_{n,n}x_n+b_n\\
  \end{array}\right.\]
\end{de}
\section{Isométries affines}
\subsection{Généralités}
\begin{de}
Soit f une application affine d'un espace affine $\xi$.\\
f est une isométrie si : 
$$\forall M,N \in \xi^2~ ||\overrightarrow{f(M)f(N)}|| = ||\overrightarrow{MN}||$$
\end{de}
\begin{prop}
Si $\varphi$ est l'application linéaire associée à f :
$$(\mbox{ f est une isométrie }) \Leftrightarrow (\varphi \mbox{ est un endomorphisme orthogonale})$$
\end{prop}
\begin{voc}
Si : 
\begin{itemize}
 \item[$\rightarrow$] det($\varphi$) = 1, alors f est un déplacement 
 \item[$\rightarrow$] det($\varphi$) = -1, alors f est un anti-déplacement
\end{itemize}
\end{voc}
\subsection{Déplacement du plan}
Soit f un déplacement plan.
\begin{center}
% use packages: array
\begin{tabular}{|l|l|l|}
\hline
Application linéaire & Isométrie \\ \hline
Identité & Translation \\ \hline
Rotation d'angle $\theta$ [2$\pi$] & Rotation affine d'angle $\theta$ [2$\pi$] de centre $\Omega$ \\ \hline
\end{tabular}
\end{center}
\subsection{Déplacement de l'espace affine de dimension 3}
\begin{center}
% use packages: array
\begin{tabular}{|l|l|l|}
\hline
Application linéaire & Point Fixe & Isométrie \\ \hline
Identité & Aucun & Translation \\ \hline
Rotation d'angle $\theta$ & Un point & Rotation affine d'axe affine $\Delta$, orienté par D, d'angle $\theta$ \\ \hline
Rotation d'angle $\theta$ & Aucun & Vissage d'axe affine $\Delta$, de vecteur $u$, d'angle $\theta$ \\ \hline
\end{tabular}
\end{center}
Un vissage est la composée d'une translation de vecteur $\bot$ au plan et d'une rotation plane.
\begin{prop}
Si f est un vissage, r une rotation plane, et $\overrightarrow{u_1}$ un vecteur $\bot$ à ce plan, alors :
$$f = \overrightarrow{u_1} o r = r o \overrightarrow{u_1}$$
\end{prop}

\chapter{Equations linéaires}
\section{Espace affine}
\begin{de}
Soit E un K-espace vectoriel.\\
Soit F un sous-espace vectoriel de E et $x_0 \in E$
$$\{ x_0 \} + F = \{x_0 +y / y \in F \}$$
est appelé espace affine de direction F.
\end{de}
\begin{prop}
Les espaces vectorielles sont des espaces affines particuliers.
\end{prop}
\begin{prop}
Si F est de dimension finies, on dit que $\{x_0\} + F$ est un espace affine de dimension finies.\\
Si F est une droite vectorielle, alors $\{x_0\} + F$ est une droite affine.
\end{prop}
\section{Equations linéaires}
\begin{de}
Soient E et F deux K-espace vectoriel.\\
Soit $f \in L(E,F)$. Soit $b \in F$.\\
L'équation d'inconnue $x$, vecteur de E :
$$f(x) = b$$
est appelé équation linéaire.
\end{de}
\subsection{Structure de l'ensemble des solutions}
\begin{itemize}
 \item[$\rightarrow$] Si $b \in Im(f)$, l'espace des solutions est un espace affine de dimenseion Ker(f)
 \item[$\rightarrow$] Si $b \notin Im(f)$, l'espaces des solutions est l'ensemble vide.
\end{itemize}
\begin{prop}
Si l'équation f(x) = b à une unique solution, alors :
\begin{itemize}
 \item[$\rightarrow$]$b \in Im(f)$
 \item[$\rightarrow$]Ker(f) = $\{ O_E \}$
\end{itemize}
Donc : 
\begin{itemize}
 \item[$\rightarrow$]$b \in Im(f)$
 \item[$\rightarrow$]f est injective
\end{itemize}
\end{prop}
\section{Système linéaire}
\begin{de}
Soit (S) un système d'inconnu ($x_1,...,x_p$).\\
Posons :
$$K^p \rightarrow K^n$$
$$(x_1,...,x_n) \rightarrow (a_{1,1}x_1+...+a_{1,p}x_p,...,a_{n,1}x_1+...+a_{n,p}x_p)$$
Notons : 
\begin{itemize}
 \item[$\rightarrow$] x = $(x_1,...,x_n)$
 \item[$\rightarrow$] b = $(b_1,...,b_n)$
\end{itemize}
On dit que :
\begin{itemize}
 \item[$\rightarrow$] f est l'application associée à (S)
 \item[$\rightarrow$] Le rang du système est le rang de A ou rang de f
 \item[$\rightarrow$] Si b $\in$ Im(f), alors S est un espace affine de dimension p-rang(A) = p-rang(S)
\end{itemize}
\end{de}
\subsection{Système de Cramer}
\begin{de}
Un système de Cramer est un système linéaire de n équations, à n inconnues, de rang n.
\end{de}
On obtient la formule :
$$x_j = \dfrac{1}{det(A)}det_{\varphi}(c_1,...,c_{j-1},b,c_{j+1},...,c_n)$$

\part{Matrice}

\chapter{Matrice et espaces vectoriel de dimension finies}
\section{Matrice}
\subsection{Définition}
\begin{de}
La matrice à n lignes et p colonnes est défini par :
$$M = [a_{i,j}]$$
avec i variant de 1 à n, et j variant de 1 à p
\end{de}
\subsection{Matrice carrée}
\begin{de}
Une matrice M est carrée si n=p. On défini la diagonale de A comme le n-uplet : $(a_{11},a_{22},...,a_{nn})$ 
\end{de}
\subsection{Vecteur ligne}
\begin{de}
On défini le vecteur ligne comme le p-uplet :
$$l_i = (x_{i,1},...,x_{i,p})$$
\end{de}
\subsection{Vecteur colonne}
\begin{de}
On défini le vecteur colonne comme le n-uplet :
$$l_j = (x_{1,j},...,x_{n,j})$$
\end{de}
\subsection{Matrice carrée particulière}
Soit T = $[x_{i,j}] \in M_n(K)$
\begin{de}
 On dit que T est triangulaire supérieur si :
$$T=\begin{bmatrix}
a_{1,1} & a_{1,2} & a_{1,3}  \\
0 & a_{2,2} & a_{2,3} \\
0 & 0 & a_{3,3} \\
\end{bmatrix}$$
\end{de}
\begin{de}
 On dit que T est triangulaire inférieur si :
$$T=\begin{bmatrix}
a_{1,1} & 0 & 0  \\
a_{2,1} & a_{2,2} & 0 \\
a_{3,1} & a_{3,2} & a_{3,3} \\
\end{bmatrix}$$
\end{de}
\begin{de}
 On dit que T est une matrice scalaire si :
$$T=\begin{bmatrix}
\lambda & 0 & 0  \\
0 & \lambda & 0 \\
0 & 0 &\lambda \\
\end{bmatrix}$$
\end{de}
Si $\lambda$=1, alors la matrice est la matrice unité, noté $I_{n}$
\subsection{Matrice carrée symétrique et antisymétrique}
\begin{de}
Soit S = $[x_{i,j}] \in M_n(K)$ \\
S est symétrique si $x_{i,j}=x_{j,i}$\\
S est antisymétrique si $x_{i,j}=-x_{j,i}$. Ceci implique que la diagonale de S est forcément nul dans ce cas
\end{de}
\subsection{Transposition et trace}
\begin{de}
La transposée de M noté $^{t}M$ est défini par :
$$M=\begin{bmatrix}
a & b & c & d\\
e & f & g & h  \\
i & j & k & l \\
\end{bmatrix}$$
alors
$$^{t}M=\begin{bmatrix}
a & e & i \\
d & f & j  \\
c & g & k  \\
d & h & l 
\end{bmatrix}$$
On a : 
$$^{t}(^{t}M) = M $$
\end{de}
\begin{de}
On défini la trace d'un matrice carrée comme la somme des termes de sa diagonale :
$$Trace(A) = \sum_{i=1}^nx_{k,k}$$
\end{de}
\subsection{Espace vectoriel des matrices}
On note $M_{n,p}(K)$ l'ensemble des matrices à n lignes et p colonne. \\
L'addition des matrices est une addition termes à termes \\
$(M_{n,p}(K),+)$ est un groupe commutatif d'élément neutre [O] \\
$(M_{n,p}(K),+,o)$ est un K espace vectoriel \\
Soit $$A_{1,1} = \begin{bmatrix}
1 & 0 & 0 & 0\\
0 & 0 & 0 & 0\\
0 & 0 & 0 & 0\\
\end{bmatrix}$$
L'ensemble {$A_{1,1},....,A_{n,p}$} est une base de $M_{n,p}(K)$ et dim($M_{n,p}(K)$)=n.p\\
$(M_n(K),+,x,o)$ est un K-algèbre de dimension $n^2$. Si AB=BA, alors les identités remarquables sont utilisables.
\subsection{Transposition}
\begin{de}
 Soit :
$$\varphi : M_{n,p}(K) \rightarrow M_{p,n}(K)$$
$$M \mapsto ^tM$$
$\varphi$ est un isomorphisme, donc c'est une application linéaire.
De plus, si la matrice est une matrice carrée :
$$\varphi \mbox{ est symétrie de }M_n(K)$$
\[\left.\begin{array}{l}
   ^tM=M \Leftrightarrow \mbox{M est symétrique}\\
    ^tM=-M \Leftrightarrow \mbox{M est antisymétrique} \\
  \end{array}\right\}
\mbox{Ce sont deux espaces supplémentaire}\]
\end{de}
\subsection{Produit de matrice}
\begin{de}
On défini le produit de matrice par : \\
Soit $l_1$ la première ligne de la matrice A\\
Soit $c_1$ la première colonne de la matrice B\\
Soit $a_1$ le première terme de la matrice produit\\
$$AB = \begin{bmatrix}
a & b & c & d
\end{bmatrix}\times\begin{bmatrix}
e \\
f \\
g \\
h \end{bmatrix} = [ae+bf+cg+dh]$$$$$$
avec [ae+bf+cg+dh] = $a_1$\
Le produit $l_2.c_1$ donne le terme $a_{2,1}$\
Le produit est non commutatif
\end{de}
\subsubsection{Cas particuliers}
Si $M \in M_{n,p}(K) et [0] \in M_{p,q}(K) $ alors :
$$M \times [0] = 0$$
Soit T une matrice scalaire $\in M_{p,q}(K)$, alors :
$$M \times T = \lambda M$$
Soit $I_n$ matrice unité d'ordre n, et $M \in M_{n}(K)$ alors :
$$MI_n=I_nM=M$$ 
\subsection{Transposition et trace du produit}
Soit A et B deux matrice $\in M_{n,p}(K)$. Alors :
$$^t(AB) = ^tB\times^tA$$
et 
$$\mbox{Trace(AB) = Trace (BA), mais généralement AB }\neq BA$$
\subsection{Matrice Carrée inversible}
\begin{de}
 Soit M $\in M_n(K)$. On dit que M est inversible si 
$$\exists N \in M_n(K) \mbox{ telque } MN = NM = I_n $$
On pose $N = M^{-1}$. On note $GL_n(K)$ l'ensemble des matrice carrée inversible d'ordre n. Cette ensemble est un groupe linéaire. Et :
$$(AB)^{-1} = B^{-1}A^{-1}$$
\end{de}
\begin{prop}
Soit (A,B)$\in (M_n(k))^2$ telque : $$AB = I_n$$
On obtient que:
\[\left\{\begin{array}{l}
   \mbox{ A est inversible et B = }A^{-1}\\
   \mbox{ B est inversible et A = }B^{-1} \\
  \end{array}\right.\]
\end{prop}

\subsubsection{Matrice carrée et inverse}
Voir Méthodologie.
\subsection{Rang d'une matrice}
\begin{de}
Le rang d'une matrice est le rang de ses vecteurs colonnes.
Si A $\in M_{n,p}(K)$ et qu'on note $c_i$ son $i^{eme}$ vecteurs colonnes, alors :
$$rang(A)=rang(\left\{c_1,...,c_p\right\})=dim(Vect \left\{c_1,...,c_p\right\})$$
et de plus : 
$$0 \leq rang(A) \leq Min\left\{n,p\right\} $$
\end{de}
\subsubsection{Rang de matrice particulière}
Le rang d'une matrice diagonale est r, si : 
$$\forall i \in \left\{1,....,r\right\} \lambda_{ii} \neq 0$$
Si dans une matrice carrée d'ordre n, les termes diagonaux sont non tous nul, alors $rang(A)=n$
\subsection{Opération élémentaire}
\begin{de}
 Il existe trois opérations élémentaire :
\begin{enumerate}[I) ]
 \item L'échange de deux colonnes : $c_i \leftrightarrow c_j$
 \item Le produit d'une colonne par un scalaire non nuls
 \item L'addition à une colonne d'une combinaison linéaire des autres
\end{enumerate}
Ces opérations élémentaire ne modifie par le rang de A
\end{de}
\section{Matrice et espaces vectoriel de dimension finies}
\subsection{Matrice de coordonnée d'un vecteur dans une base}
\begin{de}
Soit $B=\left\{e_1,...,e_n\right\}$ base d'un espace vectoriel de E\\
Soit $u \in E$, $\exists !(x_1,...,x_n) \in K^n$ telque : 
$$ u = x_1e_1+...+x_ne_n$$
On note $mat_{B}(u)=\begin{bmatrix}
x_1\\
.\\
.\\
x_n\\
\end{bmatrix} \in M_{1,n}(K)$ la matrice de de u dans B 
\end{de}
\subsection{Matrice d'une famille de vecteurs}
\begin{de}
 Si $ \forall j \in \left\{1,..p\right\}$, $u_j \in E$ et $mat_B(u)= \begin{bmatrix}
x_{1,j}\\
.\\
.\\
x_{n,j}\\
\end{bmatrix}$, alors :
$$mat_B(u_1,...,u_p) = \begin{bmatrix}
x_{1,1} & . & . & x_{1,p}\\
. & . & . & .\\
. & . & . & . \\
x_{n,1} & . & . & x_{n,p}\\
\end{bmatrix}$$
De plus, le rang de la matrice est le rang de la famille de vecteurs.
\end{de}
\subsection{Matrice de passage entre deux bases}
\begin{de}
 On note $mat_B(B')$ la matrice de passage de B à B'. On la note P
\end{de}
\subsection{Coordonnée d'un vecteur dans deux bases}
\begin{de}
On note B et B' deux bases de E.\\
On note $P=mat_B(B') \in M_n(K)$ la matrice de passage de B à B'\\
Soit $u \in E$\\
On pose $X=mat_B(u)$ et $X'=mat_{B'}(u)$ \\
On obtient la relation :
$$X = PX'$$
De plus, P est inversible, et son inverse est : 
$$P^{-1}=mat_{B'}(B)$$
\end{de}
\subsection{Matrice d'une application linéaire}
\begin{de}
Soit $f \in L(E,F)$. \\
Soit $B_E = \left\{e_1,...,e_p\right\}$ base de E et  $B_F = \left\{f_1,...,f_n\right\}$ base de F.\\
Soit M la matrice de f dans $B_E,B_F$ :
$$M = mat_{B_E,B_F}(f) = mat_{B_F}(\left\{f(e_1),...,f(e_p)\right\})$$
Le nombre de colonne de la matrice est défini par la dimension de l'espace de départ, celui des ligne par la dimension de l'espace d'arrivé
\end{de}
\subsubsection{Cas Particuliers}
\begin{enumerate}[I) ]
 \item La matrice de l'application nul $\in L(E,F)$ est la matrice nul de $M_{dim(F),dim(E)}(K)$\\
 \item La matrice d'un endomorphisme de E est une matrice carrée d'ordre dim(E)
 \item La matrice de l'identité est $I_n$
 \item La matrice de l'homothétie est de rapport k par rapport à $I_n$
\end{enumerate}
\subsection{Coordonnée de l'image d'un vecteur}
\begin{de}
Soit $B_E$ base de E, $B_F$ base de F. \\
Soit u $\in$ E \\
Posons $M = mat_{B_E,B_F}(f)$, $X = mat_{B_E}(u)$, $Y = mat_{B_F}(u)$. Alors :
$$Y = M.X$$
\end{de}
De plus : 
\begin{theo}
 Si f est une application de E dans F telque $\exists M \in M_{n,p}$ telque $\forall u \in E$, l'égalité ci-dessus est vérifié, alors f est une application linéaire
\end{theo}
\subsection{Unicité de la matrice, pour les bases fixes}
\begin{de}
Soient $B_E,B_F$ bases de E et de F, avec dim(E) = p, dim(F)=n
Soit : 
$$\varphi : L(E,F) \rightarrow M_{n,p}(K)$$
$$f \rightarrow mat_{B_E,B_F}(f)$$
$\varphi$ est une application linéaire. On en déduit donc que :
$$(f=g)\Leftrightarrow(mat_{B_E,B_F}(f)=mat_{B_E,B_F}(g))$$
\end{de}
\begin{prop}
Soit $f \in L(E,F)$, $B_E,B_F$ bases de E et F\\
Soit $A \in M_{n,p}(K)$ \\
Soit $x \in E$. Supposons que $X = mat_{B_E}(x)$ et $Y=mat_{B_F}(f(x))$, et qu'on obtient :
$$Y = AX $$
Alors $A = mat_{B_E,B_F}(f)$
\end{prop}
\subsection{Matrice et opérations}
\begin{de}
 Soient $B_E,B_F$ bases fixées de E et de F. \\
$\varphi$ est une application linéaire, c'est donc un isomorphisme. Nous avons en effet montré que $\varphi$ est bijective. On en déduit que $M_{n,p}(K)$ est de dimension finies, donc L(E,F) l'est aussi.
$$dim(L(E,F)=dim(E)\times dim(F)=dim(M_{n,p})$$
\end{de}
\subsection{Composée d'application linéaire}
\begin{de}
Soient E,F,G espaces vectoriel de dimension finie, et de bases respective $B_E,B_F,B_G$.\\
Soit $f \in L(E,F)$, $g \in L(F,G)$. Alors : 
$$mat_{B_E,B_G}(gof)=mat_{B_F,B_G}(g)\times mat_{B_E,B_F}(f)$$
\end{de}
\subsection{Matrice inversible et isomorphisme - Endomorphisme}
\begin{de}
Si f est un isomorphisme de E dans F, alors $mat_{B_E,B_F}(f)$ est inversible est : 
$$(mat_{B_E,B_F}(f))^{-1}=mat_{B_F,B_E}(f^{-1})$$
Si f est un endomorphisme, on a : 
$$(mat_{B_E}(f))^n = mat_{B_E}(f^n)$$
\end{de}
\subsection{Changement de bases}
\begin{de}
Soit $f \in L(E,F)$. Soient $B_E,B_{E'}$ bases de E. Soient $B_F,B_{F'}$ bases de F. \\
On pose : \[\left\{
  \begin{array}{ll}
    M = mat_{B_E,B_F}(f)\\
    M' = mat_{B_{E'},B_{F'}}(f) \\
    P = mat_{B_E}(B_{E'}) \\
    Q = mat_{B_{F'}}(B_{F})
  \end{array}\right.\]
Alors : 
$$M' = Q^{-1}MP$$
ou, si f est un endomorphisme :
$$M' = P^{-1}MP$$
\end{de}
\subsection{Trace d'un endomorphisme}
Soit f un endomorphisme, A,B deux matrices d'ordre n. On sait déjà que Trace(AB)=Trace(BA) et si :
\[\left\{
  \begin{array}{ll}
    M = mat_{B_E}(f)\\
    M' = mat_{B_{E'}}(f) \\
  \end{array}\right.\]
alors \[\left\{
  \begin{array}{ll}
   rang(M)=rang(M')=rang(f)\\
   Trace(M)=Trace(M')=rang(f) \\
   det(M)=det(M')=det(f)
  \end{array}\right.\]
\subsection{Matrice semblable}
\begin{de}
 Soient A,B deux matrice carrée d'ordre n. On dit que A et B sont semblable si $\exists$E espace vectoriel, $\exists B_E,B_{E'}$ bases de E, $\exists$f endomorphisme de E telque : 
$$B = P^{-1}MP$$
Ce qui revient à : 
$$(\mbox{A et B sont semblables}) \Leftrightarrow (\exists P  \in GL_n(K) \mbox{ telque } B = P^{-1}MP) $$
\end{de}
\subsection{Rang d'une application linéaire}
On peut toujours ramener la matrice dans des bases $B_{E'},B_{F'}$ de f à $J_r$ : 
$$J_r = \begin{bmatrix}
1 & . & . & . & 0\\
0 & 1 & . & . & .\\
. & . & 1 & . & . \\
. & . & . & . & . \\
0 & . & . & . & 0
\end{bmatrix}$$
Donc si : \\
f $\in$ L(E,F) tel qu'il $\exists B_{E'},B_{F'}$ bases de E et F telque $mat_{B_{E'},B_{F'}}(f)=J_r$, alors rang(f)=r\\
De plus, on a :
$$(^tP)^{-1}=^t(P^{-1})$$
On montre que $^tA$ et $^tJ_r$ sont semblable, donc le rang d'une matrice est le rang de ses vecteurs colonnes comme celui de ses vecteurs lignes.\\
Et on obtient : 
$$rang(^tA) = rang(A)$$


\chapter{Déterminants}
\section{Forme n-linéaire}
Soit E un K-espace vectoriel de dimension n.
\begin{de}
Soit :
$$\varphi : E^n \rightarrow K$$
$$(u_1,...,u_n) \mapsto \varphi(u_1,...,u_n)$$
$\varphi$ est une forme n-linéaire si elle est linéaire par rapport à chacune de ses variables.
\end{de}
\subsubsection{Expression dans une base}
\begin{de}
Soit B = $(e_1,...,e_n)$ base de E.\\
Soit $u_1,...,u_n$ n vecteurs de E.\\
Notons pour $j \in \left\lbrace 1,...,n\right\rbrace$ $mat_B(u_j) = 
\begin{bmatrix}
  x_{1,j} \\
  . \\
  . \\
  x_{n,j}  \\
\end{bmatrix}$\\
Si $\varphi$ est une forme n-linéaire, alors : $$\varphi(u_1,...,u_n) = \sum_{1 \leq i_1,...,i_n \leq n} x_{i_1,1}...x_{i_n,n}\varphi(e_{i_1},...,e_{i_n})$$ 
\end{de}
\subsection{Forme n-linéaire alterné}
\begin{de}
Soit $\varphi$ forme n-linéaire.\\ On dit que $\varphi$ est alternée si :\\
$\forall u_1,...,u_n \in E^n$\\
$\forall i,j \mbox{ éléments distinct de }  \left\lbrace 1,...,n \right\rbrace$
$$\varphi(u_1,...,u_i,...,u_j,...,u_n) = -\varphi(u_1,...,u_j,...,u_i,...,u_n)$$
\end{de}
\begin{prop}
Si $\varphi$ est une forme n-linéaire alterné.\\
Si $\left\lbrace u_1,...,u_n \right\rbrace$ est une partie liée de E.\\
Alors : 
$$\varphi(u_1,...,u_n) = 0$$ 
\end{prop}
\subsubsection{Expression dans une base}
Soit $\varphi$ forme n linéaire alterné et B base de E.
$$\varphi(u_1,...,u_n) = \sum_{1 \leq i_1,...,i_n \leq n} x_{i_1,1}...x_{i_n,n}\varphi(e_{i_1},...,e_{i_n})$$ 
Soit $S_n$ l'ensemble des bijections de $\left\lbrace 1,...,n \right\rbrace$ dans lui même. 
$$\varphi(u_1,...,u_n) = \sum_{\sigma \in S_n} x_{\sigma_1,1}...x_{\sigma_n,n}\varphi(e_{\sigma_1},...,e_{\sigma_n})$$ 
Si $\sigma \in S_n$, on note $\varphi(e_{\sigma_1},...,e_{\sigma_n}) = \varepsilon(\sigma)\varphi(e_{1},...,e_{n})$\\
On dit que $\varepsilon(\sigma)$ est la signature de $\sigma$, avec $\varepsilon(\sigma) = (-1)^p$, avec p nombre de changement effectuer pour obtenir le bon ordre de la base.\\
On obtient donc :
$$\varphi(u_1,...,u_n) = \left[ \sum_{\sigma \in S_n} \varepsilon(\sigma)x_{\sigma_1,1}...x_{\sigma_n,n}\right]\varphi(e_1,...,e_n) $$
\begin{de}
 Le déterminant dans la base B est l'unique forme n-linéaire alternée $\varphi$ vérifiant :
$$ \varphi(e_1,...,e_n) = 1$$
On le note : $det_B$
\end{de}
\begin{prop}
Toutes les applications de $E\times E\times .... \times E \rightarrow K$ défini par $\forall(u_1,...,u_n) \in E^n$ : 
$$\varphi(u_1,...,u_n) = A\sum_{\sigma \in S_n}\varepsilon(\sigma)x_{\sigma(1),1}...x_{\sigma(n),n}$$
avec A scalaire fixé, est une forme n-linéaire de alterné et $\varphi(B)=A$, avec B base de E.
\end{prop}
Autre formulation : 
\begin{prop}
L'ensemble des formes n-linéaire alternée est une droite vectorielle.
$$Vect(det_b) = \left\lbrace \mbox{ A.}det_b\mbox{ / A scalaire quelconque } \right\rbrace $$ 
\end{prop}
\section{Déterminant dans une base B}
\begin{de}
Soit B base de E.\\
$det_B$ est l'unique forme n-linéaire alternée vérifiant $det_B(B)=1$.
$$det_B(u_1,...,u_n) = \sum_{\sigma \in S_n} \varepsilon(\sigma)x_{\sigma(1),1}...x_{\sigma(n),n}$$
\end{de}
\subsection{Déterminant dans deux bases différentes}
Soit B,B' deux bases de E. $\forall u_1,...,u_n $ vecteurs de E :
$$det_{B'}(u_1,...,u_n) = det_{B'}(B).det_B(u_1,...,u_n)$$
\begin{prop}
$$(det_B(u_1,...,u_n) = 0)\Leftrightarrow (\left\lbrace u_1,...,u_n \right\rbrace \mbox{ est liée}) $$
$$(det_B(u_1,...,u_n) \neq 0)\Leftrightarrow (\left\lbrace u_1,...,u_n \right\rbrace \mbox{ est une base}) $$
\end{prop}
\begin{prop}
 Si B et B' sont deux bases :
$$det_{B'}(B) = \dfrac{1}{det_B(B')}$$
\end{prop}
\subsubsection{Formulaire}
\begin{itemize}
 \item[$\rightarrow$] $det_B(B)=1$
 \item[$\rightarrow$] $det_B(\lambda u_1,...,\lambda u_n) = \lambda^n.det(u_1,...,u_n)$
 \item[$\rightarrow$] $det(Ide) = 1$
\end{itemize}
\section{Déterminant d'un endomorphisme}
\begin{de}
Soit f $\in L(E)$\\
Soient B et B' deux bases de E. 
$$det_B(f(B)) = det_{B'}(f(B'))$$
Ce scalaire, indépendant du choix de la base, est appelé déterminant de f. On le note : det(f) 
\end{de}
\begin{prop}
Si f,g sont deux endomorphisme de E :
$$det(fog) = det(g).det(f)$$
\end{prop}
\begin{prop}
Si $f \in L(E)$ :
$$(\mbox{ f est bijectif }) \Leftrightarrow ( det(f) \neq 0)$$
\end{prop}
\begin{prop}
Si f est un automorphisme de E : 
$$det(f^{-1})=\dfrac{1}{det(f)}=(det(f))^{-1}$$
\end{prop}
\section{Déterminant d'une matrice carrée}
\begin{de}
Soit $A \in M_n(K)$.\\
Notons $c_1,...,c_n$ ses vecteurs colonnes, élément de $K^n$.\\
Notons $l_1,...,l_n$ ses vecteurs lignes, élément de $K^n$.\\
Soit $\varphi$ la base canonique de $K^n$.
$$det(A) = det_{\varphi}(c_1,...,c_n) = det_{\varphi}(l_1,...,l_n)$$
\end{de}
\begin{prop}
Soit $f \in L(E)$, avec B base de E.\\
Notons M = $mat_B(f)$.\\
On obtient :
$$det(f) = det_B(f(B)) = det(M)$$
\end{prop}
\subsection{Lien entre vecteurs lignes et vecteurs colonnes}
Si $A = [a_{i,j}]_{1 \leq i,j \leq n}$ :
$$det(A) = \sum_{\sigma \in S_n} \varepsilon(\sigma)a_{\sigma(1),1}...a_{\sigma(n),n} = \sum_{\sigma \in S_n} \varepsilon(\sigma)a_{1,\sigma(1)}...a_{n,\sigma(n)}$$
\begin{prop}
D'après l'égalité ci-dessus, on établie que :
$$det(^tA) = det(A)$$
\end{prop}
\subsection{Déterminant singulier}
Soient (A,B) $\in M_n(K)^2$, $\lambda \in K$.
$$det(AB) = det(A).det(B)$$
$$det(\lambda A) = \lambda^ndet(A)$$
$$det(A+B) = ????$$
\subsection{Opérations élémentaires et déterminant}
On peut effectuer des opérations élémentaires, tout comme pour le calcul de rang, pour calculer le déterminant d'une matrice. Ceci implique les règles suivantes :
\begin{itemize}
 \item[$\rightarrow$] $c_i \leftrightarrow c_j$ : Changement de signe du déterminant
 \item[$\rightarrow$] $c_i \leftarrow \lambda c_j$ : $\lambda$ fois le déterminant
 \item[$\rightarrow$] $c_i \leftarrow \underset{k=1,k\neq i}\sum \lambda_k.c_k$ : Aucun changement
\end{itemize}
\subsection{Déterminant remarquable}
\subsubsection{Matrice diagonale}
Soit A, matrice diagonale de diagonale : $(d_1,...,d_n)$
On obtient :
$$det(A)=d_1d_2...d_n$$
\subsubsection{Matrice triangulaire}
Soit A, matrice triangulaire de diagonale : $(t_{11},...,t_{nn}$
On obtient :
$$det(A)=t_{11}...t_{nn}$$
\section{Développement de déterminant d'une matrice}
\begin{de}
Soit A = $[a_{i,j}]_{1 \leq i,j \leq n}$.\\
Soit $j \in \left\lbrace 1,...,n\right\rbrace $.\\
Le développement par rapport à la j-ème colonne donne :
$$det(A) = a_{1,j}\Delta_{1,j}+...+a_{n,j}\Delta_{n,j}$$
Le développement par rapport à la i-ème ligne donne :
$$det(A) = a_{i,1}\Delta_{i,1}+...+a_{i,n}\Delta_{i,n}$$
On appelle cofacteur de $a_{i,j}$ dans le développement $\Delta_{i,j}$ :
$$\Delta_{i,j} = \sum_{\sigma \in S_n,\sigma(j)=i} \varepsilon(\sigma)a_{\sigma(1),1}...a_{\sigma(j-1),j-1}a_{\sigma(j+1),j+1}...a_{\sigma(n),n}$$
\end{de}
\subsection{Calcul des cofacteurs}
\begin{prop}
On détermine le cofacteurs à l'aide de l'égalité suivantes :
$$\Delta_{i,j} = (-1)^{i+j}\times A$$
Avec :
\begin{itemize}
 \item[$\rightarrow$] A : Déterminant de la matrice obtenu en enlevant la ligne i et la colonne j.
\end{itemize}
\end{prop}
\subsection{Inverse d'une matrice inversible}
Soit A = $[a_{i,j}]_{1 \leq i,j \leq n}$.\\
Soit la comatrice de A : 
$$Com(A) = [\Delta_{i,j}]_{1 \leq i,j \leq n}$$
Si A est inversible, donc det(A) $\neq$ 0, alors :
$$A^{-1} = \dfrac{1}{det(A)}^tCom(A)$$
\subsection{Formule de Sarrus, pour n=3}
La formule de Sarrus est de reporte la $1^{er}$ et la $2^{nd}$ ligne de la matrice sous la matrice, puis de trace les diagonales et les anti-diagonales. Les diagonales sont comptées positivement, les anti-diagonales négative

\appendix                     % Les annexes
\part{Annexe}
\chapter{Matrices}
\section{Inversibilité et inverse}
Soit $A \in M_n(K)$ \\
On peut déterminer si A est inversible et trouver son inverse à l'aide de cinq méthodes :
\subsection{Interprétation}
\subsubsection{1ère méthode}
On pose que A est la matrice d'une application linéaire f dans la base canonique de $\mathbb{R}^n$. On montre que f est un isomorphisme et on détermine $f^{-1}$. Dans ce cas, $A^{-1}=mat_B(f^{-1})$
\subsubsection{2nd méthode}
Soit $(e_1,e_2,...,e_n)$ vecteurs de $\mathbb{R}^n$ telque A=$mat_{C_n}(e_1,...,e_n)$, avec $C_n$ la base $\left\{c_1,c_2,...c_n\right\}$. On pose le système correspondant par lecture en colonne de la matrice, à savoir :
  \[\left\{\begin{array}{c}
   e_1 = a_1c_1+....+a_nc_n\\
   e_2 = b_1c_1+....+b_nc_n\\
   ........
  \end{array}\right.
\]
Puis on retourne le système, on exprime $\left\{c_1,c_2,...c_n\right\}$ en fonction de $(e_1,...,e_n)$. On en déduit donc que la base de $C^n$ est génératrice, donc base car le bon nombre d'élément. A est donc inversible. On injecte le tout par colonne dans une matrice, et on obtient la matrice inverse
\subsection{Opérations élémentaires}
On effectue des opérations élémentaire sur A jusqu'a obtenir la matrice unité $I_n$. On en déduit que rang(A)=n, donc A est inversible, et on reportent les opérations effectué sur A sur $I_n$. La matrice qui découle de $I_n$ est $A^{-1}$
\subsection{Astuces et propriété}
Si n est assez faible, 2 ou 3, on multiplie A par elle même jusqu'a obtenir une matrice de la forme : 
$$A^{n}=\lambda A + \mu I_n $$
On obtient $A^{-1}$ grâce à ceci : $$A.\dfrac{1}{\mu}(A^{n-1}- \lambda I_n) = I_n$$
Si il s'agit de monter uniquement le caractère inversible de A, on utilise la propriété suivante :
\begin{prop}
Si $rang(A)=n$, alors A est inversible
\end{prop}
\section{Opérations sur les matrices}
\subsection{Changement de base}
\subsubsection{1ère méthode}
On utilise la formule, dans le cas d'un endomorphisme, pour passer d'une base $B_E$ à une base $B'_E$ :
$$M' = P^{-1}MP$$
avec :
  \[\left\{\begin{array}{l}
   M'=mat_{B'_E}(f)\\
   M=mat_{B_E}(f)\\
   P=mat_{B_E}(B'_E)
  \end{array}\right.
\]
\subsubsection{2nd méthode}
On sait que M', la matrice dans la nouvelle base, est constitué de l'image par f de l'ancien base,$B_E$ en fonction de la nouvelle, $B'_E$. On calcule donc l'image des vecteurs de $B_E$ par f, puis on les expriment en fonction des vecteurs de la base $B'_E$
\subsection{Calcul des coordonnée d'un vecteur dans une autre base}
On utilise la formule suivante: 
$$X'=P^{-1}X$$
avec :
  \[\left\{\begin{array}{l}
   X = mat_{B_E}(x)\\
   X' = mat_{B'E}(x)\\
   P^{-1} = mat_{B'_E}(B_E)
  \end{array}\right.
\]
\subsection{Coordonnée de l'image d'un vecteur dans un base}
On utilise la formule suivante:
$$Y=MX$$
avec 
  \[\left\{\begin{array}{l}
   Y = mat_{B_E}(f(x))\\
   X = mat_{B_E}(x)\\
   M = mat_{B_E}(f)
  \end{array}\right.
\]
\section{Base de l'image et du noyau d'une application}
\subsection{Base de l'image}
On détermine tout d'abord le rang de la matrice, puis on utilise la propriété qui dit que si 
$$f : E \rightarrow E$$
avec $B_E=(e_1,e_2,...,e_n)$ base de E, alors :
$$Im f = Vect\left\{f(e_1),...,f(e_n)\right\}$$
\subsection{Base du noyau}
Si dim(Ker(f)) est 1 ou 2, alors on observe la matrice de l'application, est on cherche une combinaison de colonne qui fourni la colonne nul. Sachant que f est linéaire et que les colonne représente les images des vecteurs de base, on rassemble ces colonne et on obtient un vecteur du noyau.\\
Exemple : 
$$\begin{bmatrix}
  1 & 2 \\
  0 & 0 \\
\end{bmatrix}$$
On remarque que $2f(e_1)-f(e_2)=0$, donc que $2e_1-e_2 \in Ker(f)$ 

\chapter{Développement limité}
\section{Obtenir le développement}
\subsection{Au voisinage de 0}
\subsubsection{Développement connu}
On utilise les développement de référence, en vérifiant toujour que le "u" utilisé tend toujours vers 0 dans x tend vers 0. Si ce n'est pas le cas, faire un changement de variable.
\subsubsection{Changement de variable}
Quand on effectue un changement de variable, on pose toujours une variable qui tend vers 0. Par la suite, quand on a exprimé le développement limité usuel en fonction de t, on détermine $t,t^2,...,t^n$ jusqu'à obtenir juste un $o(x^p)$ si on cherche un développement limité d'ordre p.\\
On peut être aussi amené à factoriser pour obtenir la forme voulu
Ex : \\
Soit f, la fonction : 
$$f : x \rightarrow ln(1+\sqrt{1+x})$$
On observe bien que $\sqrt{1+x}$ tend vers 1 quand x tend vers 0, et non vers 0. Posons donc :
$$t = \sqrt{1+x} - 1$$
Donc, quand $x\rightarrow 0$, t$\rightarrow 0$. D'où, au voisinage de 0:
$$f(x) = ln (1 + t + 1)=ln(2+t)$$
En effet, quand $t\rightarrow 0$, f(x) tend bien vers 2.
Or le développement usuel est $ln(1+u)$, avec u qui tend vers 0. Dans ce genre de situation, on factorise toujour par 2. En effet si :
$$f(x) = ln(a+u)$$
$$f(x) = ln(a(1+\dfrac{u}{a}))$$
$$f(x) = ln(a) + ln(1+\dfrac{u}{a})$$
Et on effectue le développement limité de ln(1+t) à ce niveau.
\section{Asymptote}
Pour determiner l'asymptote à une courbe, on détermine en premier lieu :
$$\lim_{x \rightarrow d} \dfrac{f(x)}{x} = a$$
Avec a $\neq$ 0
Puis : 
$$\lim_{x \rightarrow d} f(x)-ax = b$$
Alors l'asymptote est ax+b en d
\chapter{Fraction Rationnelle}
\section{Partie entière}
Pour déterminer la partie entière, on fait la division euclidienne du numérateur par le dénominateur, sans l'ordre de multiplicité si il existe.
\section{Décomposition en élements simple}
On décompose en élements simple et on détermine ces coefficent en travaillant sur l'égalité :
$$\dfrac{Num.}{(X-a)^{\alpha}(X-b)^{\beta}} = \dfrac{\lambda_1}{(X-a)} \dfrac{\lambda_2}{(X-a)^2}...\dfrac{\lambda_{\alpha}}{(X-a)^{\alpha}} \dfrac{\mu_1}{(X-b)} \dfrac{\mu_2}{(X-b)^2}...\dfrac{\mu_{\beta}}{(X-b)^{\beta}}$$
Si deg(Dénomin.) > 1 (ex : $(X^2+1)^{\alpha}$), dans la décomposition, alors :\
$\forall i \in \left\lbrace 1,2,...,\alpha \right\rbrace $
$$\lambda_i = \lambda_{i_1}X+\lambda_{i_2}$$
\subsection{Dans C}
\subsubsection{Pôle simple}
Si $F = \dfrac{P}{Q} = \dfrac{P}{(X-a)Q_1}$, avec a qui n'est pas racine de $Q_1$, on obtient donc : 
$$F = F_0 + \dfrac{\lambda}{X-a}$$
Avec a qui n'est pas un pôle de $F_0$. Il existe deux techniques pour déterminer $\lambda$: 
\begin{enumerate}[I) ]
 \item On multiple F par le dénominateur, X-a, et on détermine la valeur en a. $$(X-a)F= \dfrac{P}{Q_1}=F_0(X-a) + \lambda$$
$$\lambda = \dfrac{\tilde{P}(a)}{\tilde{Q_1}(a)}$$
 \item On utilise la dérivé. On applique la formule de Taylor en a pour Q. 
$$Q = 0 + \tilde{Q'}(a)(X-a) + (X-a)^2R$$
avec $R \in C[X]$.\
D'où :
$$(X-a)F = (X-a)\dfrac{P}{\tilde{Q'}(a)(X-a) + (X-a)^2R} = \dfrac{P}{\tilde{Q'}(a) + (X-a)R}$$
Et on prend la valeur en a de cette expression. On obtient : 
$$\lambda = \dfrac{\tilde{P}(a)}{\tilde{Q'}(a)}$$
\end{enumerate}
\subsubsection{Pôle double}
\begin{itemize}
 \item[$\rightarrow$]Sans ordre de multiplicité :\ Si une fraction F possède un pôle double, a et b, alors on détermine les coefficiants c et d en prenant la valeur en a de (X-a)F et la valeur en b de (X-b)F.
 \item[$\rightarrow$] Avec ordre de multiplicité :\ Si une fraction F possède un pôle double, a et b, avec les ordres de multiplicité respectif $\alpha$ et $\beta$, alors on détermine deux des coefficiants en prenant la valeur en a de $(X-a)^{\alpha}$ et la valeur en b de $(X-b)^{\beta}$.\
Pour déterminer les autres coefficiants, on détermine des valeurs particulière. En géneral, pour $\alpha$=2, on prend la valeur en 0 et la limite en $+\infty$
\end{itemize}
\subsubsection{Parité}
Si on a : $F(-X) = -F(X)$ ou $F(-X)=F(X)$ on développe les expressions et on obtient entre les différentes coefficiants des relations, ce qui limite les calculs.
\subsection{Dans R}
\subsubsection{Décomposition indirecte}
Si on a la décomposition dans C, avec des pôles imaginaire, on met sous le même dénominateur les fractions avec les pôles conjugés. On obtient la décomposition dans R.
\subsubsection{Décomposition direct}
\begin{itemize}
 \item[$\rightarrow$] Soit on passe par un ensemble de valeur particulier, pour déterminer les différents coefficiants
 \item[$\rightarrow$] Soit, si possible, on utilise un pole complexe, et on détermine les coefficiant.\ Ex: On utilise la valeur en i de ($X^2$+1)F.
\end{itemize}
\chapter{Intégrale}
\section{Étude de la fonction}
\subsection{Définition}
Soit $$g: x\mapsto \int_{u(x)}^{v(x)} f$$
Pour étudier le domaine de définition d'une fonction :
\begin{itemize}
 \item[{$\rightarrow$}] On travaille à x fixé : Soit $x \in \mathbb{R}$
 \item[{$\rightarrow$}] Puis : (f(x) existe) $\leftarrow$
$\left\{\begin{array}{l}
   \mbox{u(x) existe}\\
   \mbox{v(x) existe} \\
   \mbox{f est continue par morceaux sur [u(x);v(x)]}
  \end{array}\right. $
\end{itemize}
\section{Primitive}
Soit f défini sur un ensemble E : 
\begin{itemize}
 \item[$\rightarrow$] Si E n'est pas un intervalle, alors ????
 \item[$\rightarrow$] Si E est un intervalle : \begin{itemize}
 \item[$\rightarrow$]Si f n'est pas continue en un réel de E, alors ???
 \item[$\rightarrow$]Si f est continue sur E, alors f admet un ensemble de primitive

\end{itemize}


\end{itemize}

\subsection{Dérivabilité}
On défini une fonction primitive de f, qui est une fonction continue par morceaux, sur son domaine de définition. On la note F.\ On obtient :
$$g(x) = F(v(x)) - F(u(x))$$
Ce qui permet de dire que f est dérivable, comme F est par définition dérivable.

\chapter{Procédé d'orthonormalisation de Gram-Schmidt}
Soit (u,v,w) une base de $\mathbb{R}^3$ ( On peut étendre cette méthode pour une base de $\mathbb{R}^n$)
On recherche une base orthonormée de $\mathbb{R}^3$ : ($e_1,e_2,e_3$).\\
Cette base doit vérifier : 
\begin{itemize}
 \item[$\rightarrow$]Vect(u) = Vect($e_1$)
 \item[$\rightarrow$]Vect($\{u,v\}$ = Vect($\{e_1,e_2\}$)
 \item[$\rightarrow$]Vect($\{e_1,e_2,e_3\}$) = $\mathbb{R}^3$
\end{itemize}
Pour $e_1$, on pose directement : 
$$e_1 = \dfrac{u}{||u||}$$
Pour $e_2$, on détermine tout d'abord un vecteur $A_2$, colinéaire à $e_2$. A l'aide d'une représation plan, on déduit que : 
$$A_2 = v + \lambda u$$
On détermine $\lambda$ sachant qu'il faut que : $$<A_2,e_1> = 0$$
On obtient donc $\lambda$. On obtient donc : 
$$e_2 = \dfrac{A_2}{||A_2||}$$
Pour $e_3$, on effectue une representation dans l'espace, et on déduit que :
$$A_3 = w + \lambda u + \mu v$$
On détermine $\lambda$ et $\mu$ sachant que $A_3$ doit vérifier : 
$$<A_3,e_1> = <A_3,e_2> = 0$$
Et on obtient $e_3$ : 
$$e_3 = \dfrac{A_3}{||A_3||}$$
\chapter{Arithmétique}
\section{Théorème de Bezout}
Pour résoudre une équation de la forme : 
$$a.u+b.v = c$$ 
d'inconnus (u,v), avec c le P.G.C.D de u et v, on détermine tout d'abord le P.G.C.D de u et v par le théorème d'Euclide. On procède de la façon suivante : 
\begin{enumerate}[1-]
 \item u = $q_1$.v + $r_1$
 \item v = $q_2$.$r_1$ + $r_2$
 \item $r_1$ = $q_3$.$r_2$ + $r_3$ ......
 \item Puis on arrive à : 
 \item $r_{n-1}$ = $q_{n-1}$.$r_n$ + $r_{n+1}$
 \item $r_n$ = $q_n.r_{n+1} + 0$
\end{enumerate}
Alors le P.G.C.D de u et v est le dernier reste non nul, $r_{n+1}$\\
Puis, par application du théorème de Bezout qui dit que : Il existe deux entier a et b tel que, si le P.C.G.D de u et v est d, alors : 
$$a.u+b.v = d$$
On exprime le premier reste en fonction de u et v, puis on retrouve les autres expressions qu'on injecte dans le première. On obtient au final a et b.
\chapter{Fonctions de $\mathbb{R}^2$ dans $\mathbb{R}$}
\section{Caractérisation de l'existence d'une limite}
Pour montrer qu'il existe une limite, on peut procéder de trois façon differentes : 
\begin{itemize}
 \item[$\rightarrow$] Par utilisation de la définition : $$\forall \varepsilon > 0~ \exists \alpha > 0~ tq~ \forall (x,y)\in B((x_0,y_0),\alpha)~ :~ |f(x,y) -l| < \varepsilon$$
 \item[$\rightarrow$] Par utilisation des propriétés sur les sommes, produits et compositions
 \item[$\rightarrow$] À l'aide du théorème d'encadrement
\end{itemize}
\section{Caractérisation de la non-existence de limite}
Pour caractériser le fait qu'une limite n'existe pas, on utilise l'idée des "chemins".\\
On recherche tout d'abord deux couples différents, ($x_1,y_1$) et ($x_2,y_2$), qui converge vers (a,b) par exemple :  
\begin{itemize}
 \item[$\rightarrow$] $\underset{t_0}\lim x_1 = a$ 
 \item[$\rightarrow$] $\underset{t_0}\lim y_1 = b$ 
 \item[$\rightarrow$] $\underset{\alpha}\lim x_2 = a$ 
 \item[$\rightarrow$] $\underset{\alpha}\lim y_2 = b$ 
\end{itemize}
Puis si on obtient, avec l différents de l' :
\begin{itemize}
 \item[$\rightarrow$] $\underset{t \rightarrow t_0}\lim f(x_1(t),y_1(t)) = l$
 \item[$\rightarrow$] $\underset{u \rightarrow \alpha}\lim f(x_2(y),y_2(u)) = l'$
\end{itemize}
Alors on en déduit que la limite en (a,b) de f n'existe pas
\chapter{Vrac - Analyse}
\section{Équation de droite}
Si on a un point A et un vecteur directeur $\overrightarrow{u}$, alors on pose un point M et on détermine :
$$det(\overrightarrow{AM},\overrightarrow{u}) = 0$$
Si on a un point A et un vecteur directeur $\overrightarrow{n}$, alors on pose un point M et on détermine :
$$\overrightarrow{n}.\overrightarrow{AM}=0$$ 
\section{Etude d'une limite}
Comment étudier la limite en a de f ?\\
Soit f une fonction définie sur I, sauf peut etre en a.\\
Soit b$\in \bar{\mathbb{R}}$.\\
Pour déterminer si la limite en a de f est b : 
\begin{itemize}
 \item[$\rightarrow$] Regarder si f est une "composée" au voisinage de a de fonction continue.
 \item[$\rightarrow$] Utiliser le théorème d'encadrement
 \item[$\rightarrow$] Décomposer en limite à gauche et limite à droite
\end{itemize}


\chapter{Trigonométrie}
\section{Formules}
\subsection{Décomposition}
\begin{itemize}
 \item[{$\rightarrow$}] cos(a+b) = cos(a).cos(b) - sin(a).sin(b)\\
 \item[{$\rightarrow$}] cos(a-b) = cos(a).cos(b) + sin(a).sin(b)\\
 \item[{$\rightarrow$}] sin(a+b) = sin(a).cos(b) + sin(b).cos(a)\\
 \item[{$\rightarrow$}] sin(a-b) = sin(a).cos(b) - sin(b).cos(a)\\
 \item[{$\rightarrow$}] tan(a+b) = $\dfrac{tan(a) + tan(b)}{1 - tan(a).tan(b)}$\\
 \item[{$\rightarrow$}] tan(a-b) = $\dfrac{tan(a) - tan(b)}{1 + tan(a).tan(b)}$
\end{itemize}
\subsection{Angle double}
\begin{itemize}
\item[{$\rightarrow$}] sin(2a) = 2.sin(a).cos(a)\\
\item[{$\rightarrow$}] cos(2a) = $2.cos^2(a)-1$ = $1 - 2.sin^2(a)$\\
\item[{$\rightarrow$}] tan(2a) = $\dfrac{2.tan(a)}{1-tan^2(a)}$
\end{itemize}
\subsection{Linéarisation}
\begin{itemize}
 \item[{$\rightarrow$}] 2.cos(a).cos(b) = cos(a+b) + cos(a-b)\\
\item[{$\rightarrow$}] 2.sin(a).sin(b) = cos(a-b) - cos(a+b)\\
\item[{$\rightarrow$}] 2.sin(a).cos(b) = sin(a+b) + sin(a-b)\\
\item[{$\rightarrow$}] $cos^2(a) = \dfrac{1+cos(2a)}{2}$\\
\item[{$\rightarrow$}] $sin^2(a) = \dfrac{1-cos(2a)}{2}$
\end{itemize}
\subsection{Somme}
Soit $(p,q) \in \mathbb{R}^2$
\begin{itemize}
\item[{$\rightarrow$}]cos(p) + cos (q) = $2.(cos(\dfrac{p+q}{2}).cos(\dfrac{p-q}{2}))$ \\
\item[{$\rightarrow$}]cos(p) - cos (q) = -$2.(sin(\dfrac{p+q}{2}).sin(\dfrac{p-q}{2}))$\\
\item[{$\rightarrow$}]sin(p) + sin (q) = $2.(sin(\dfrac{p+q}{2}).cos(\dfrac{p-q}{2}))$\\
\item[{$\rightarrow$}]sin(p) - sin (q) = $2.(cos(\dfrac{p+q}{2}).sin(\dfrac{p-q}{2}))$
\end{itemize}
\section{Fonction inverse}
\subsection{Fonction Hyperbolique}
\begin{center}
% use packages: array
\begin{tabular}{llll}
Fonction & $D_f$ & $D_{f'}$ & f'(x) \\
Argch & $]1:+\infty$ & ]1;$+\infty$[ & $\dfrac{1}{\sqrt{x^2-1}}$ \\  
Argsh & $\mathbb{R}$ & $\mathbb{R}$ & $\dfrac{1}{\sqrt{x^2+1}}$ \\ 
Argth & $]-1;1[$ & $]-1;1[$ & $\dfrac{1}{1-x^2}$ \\ 
\end{tabular}
\end{center}
\subsection{Fonction Trigonométrique}
\begin{center}
% use packages: array
\begin{tabular}{|l|l|l|l|}
\hline
Fonction & $D_f$ & $D_{f'}$ & f'(x) \\ \hline
Arccos & [-1:1] & ]-1;1[ & $\dfrac{-1}{\sqrt{1-x^2}}$ \\  \hline
Arcsin & [-1:1] & ]-1;1[ & $\dfrac{1}{\sqrt{1-x^2}}$ \\ \hline
Arctan & $\mathbb{R}$ & $\mathbb{R}$ & $\dfrac{1}{1+x^2}$ \\ \hline
\end{tabular}
\end{center}
\begin{itemize}
 \item[{$\rightarrow$}] $\forall x \in \mathbb{R}$ arcsin(x) + arccos(x) = $\dfrac{\pi}{2}$\\
 \item[{$\rightarrow$}] $\forall x \in \mathbb{R}^*$ arctan(x) + arctan($\dfrac{1}{x}$) = $\pm\dfrac{\pi}{2}$ (dépend du signe de x)\\
 \item[{$\rightarrow$}] $\forall x \in \mathbb{R}~ cos^2(x)+sin^2(x) = 1$\\
 \item[{$\rightarrow$}] $\forall x \in \mathbb{R}~ ch^2(x) - sh^2(x) = 1$
\end{itemize}


\backmatter
\tableofcontents            % Table des matières
\end{document}
