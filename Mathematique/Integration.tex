\chapter{Intégration}
\begin{de}
L'intégrale d'une fonction est par définition un nombre.\\
Une primitive est une autre fonction.
\end{de}
\section{Fonctions continues par morceaux}
\subsection{Subdivision}
Soit [a,b] segment de $\mathbb{R}$, avec a<b.
\begin{de}
On appelle subdivision de [a,b], une liste vérifiant :
$$a=x_0<x_1<....<x_n=b$$
On note $\sigma=(x_i)_{a\leq i\leq n}$.\\
Le pas d'une subdivision, qui est la longueur d'intervalle la plus importante, est défini comme :
$$\underset{1 \leq i \leq n}\max |x_i - x_{i-1}|$$
\end{de}
Si les intervalles sont tous de la mêmes longueurs, la subdivision est dite régulière. De plus, si $\sigma$ est celle d'une subdivision régulière de [a,b], alors :
$$\forall k,~ x_k = a+k\dfrac{b-a}{n}$$ 
\subsection{Fonction en escalier sur [a,b]}
Soit f fonction définie sur [a,b].
\begin{de}
 On dit que f est en escalier si il existe une subdivision de [a,b] : $(x_i)_{0 \leq i \leq n}$ telle que :
$$\forall i \in \left\lbrace1,...n\right\rbrace, \mbox{ f est constante sur } ]x_{i-1};x_i[$$
\end{de}
\begin{prop}
On défini les propriétés suivantes :
 \begin{itemize}
 \item[$\rightarrow$] Une fonction en escalier est bornée
 \item[$\rightarrow$] L'ensemble des fonctions en escalier sur [a,b] est un $\mathbb{R}$-espace vectoriel
\end{itemize}
\end{prop}
\subsection{Fonction continue par morceaux}
Soit f défini sur [a,b]
\begin{de}
 f est continue par morceaux sur [a,b] si :
\[\left\{\begin{array}{l}
   \forall i \in \left\lbrace1,2,...,n\right\rbrace \mbox{ f est continue sur }]x_{i-1};x_i[\\
   \forall i \in \left\lbrace1,2,...,n\right\rbrace~ \underset{x_{i}^-}\lim\mbox{ existe }\\
    \forall i \in \left\lbrace0,2,...,n-1\right\rbrace~ \underset{x_{i}^+}\lim\mbox{ existe }\\
  \end{array}\right.
\]
\end{de}
\begin{prop}
 On défini les propriétés suivantes :
\begin{itemize}
 \item[$\rightarrow$] Une fonction continue par morceaux est bornée
 \item[$\rightarrow$] L'ensemble des fonctions continue par morceaux sur [a,b] est un espace vectoriel
 \item[$\rightarrow$] Une fonction en escalier est continue par morceaux
 \item[$\rightarrow$] Une fonction continue sur [a,b] est continue par morceaux
\end{itemize}
\end{prop}

\subsection{Approximation d'une fonction continue par morceaux par des fonctions en escalier}
Soit f continue par morceaux sur [a,b] ( ce qui comprend les fonctions continues sur [a,b])
\begin{de}
$$\exists \varepsilon > 0 \mbox{ Il existe deux fonctions en escalier sur [a,b], }\varphi \mbox{ et } \psi \mbox{ telque } $$
\[\left\{\begin{array}{c}
  \forall x \in [a,b]~ \varphi(x)\leq f(x) \leq \psi(x)\\
  0 \leq \psi(x) - \varphi(x) \leq \varepsilon 
  \end{array}\right.
\]
\end{de}
\section{Intégrale de Riemann}
\subsection{Intégrale d'une fonction en escalier}
\begin{de}
Soit $\varphi$ fonction en escalier sur [a,b].\\
Notons $(x_i)_{0 \leq i \leq n}$ une subdivision adaptée à $\varphi$, et posons : 
$$\forall i \in \left\lbrace 1,..,n\right\rbrace,~ \forall x \in ]x_{i-1};x_i[~ \varphi(x) = \lambda_i$$
L'intégrale sur [a,b] de $\varphi$ est : 
$$\int_{[a,b]}\varphi = \sum_{i=1}^n(x_i - x_{i-1})\lambda_i$$
\end{de}
On note aussi cette intégrale de la façon suivante :\\
Si a < b, alors :
$$\int_{[a,b]}\varphi = \int_a^b\varphi$$
Si b < a : 
$$\int_{[a,b]}\varphi = \int_b^a\varphi$$
Convention : 
$$\int_a^b\varphi =-\int_b^a\varphi$$
$$\int_a^a\varphi = 0$$
\begin{prop}
Si $\varphi$ et $\psi$ sont deux fonctions en escalier sur [a,b].\\
Si $\lambda$ et $\mu$ sont deux réels :
$$\int_{[a,b]}\lambda\varphi + \mu\varphi = \lambda\int_{[a,b]}\varphi + \mu\int_{[a,b]}\psi$$
\end{prop}
\begin{prop}
Soient $\varphi$ et $\psi$ deux fonctions en escalier sur [a,b].\\
Si $\forall x \in [a,b]$, $\varphi(x) \leq \psi(x)$, alors :
$$\int_{[a,b]}\varphi \leq \int_{[a,b]}\psi$$
\end{prop}
De plus, si $\varphi \geq 0$ sur [a,b] alors : 
$$\int_{[a,b]}\varphi \geq 0$$
\section{Intégrale d'une fonction continue par morceaux}
Soit f fonction continue par morceaux sur [a,b]. Notons :
\[\left\{\begin{array}{l}
   E_+ = \left\lbrace \varphi \mbox{ en escalier sur [a,b] } / \varphi \geq g \right\rbrace \\
   E_- = \left\lbrace \varphi \mbox{ en escalier sur [a,b] } / \varphi \leq g \right\rbrace \\
   A_+ = \left\lbrace \int_{[a,b]} \varphi / \varphi \in E_+ \right\rbrace\\
   A_- = \left\lbrace \int_{[a,b]} \varphi / \varphi \in E_- \right\rbrace\\
  \end{array}\right.
\]
\begin{prop}
 Inf($A_+$) et Sup($A_-$) existent et sont égaux
\end{prop}
\begin{de}
 La valeur commune de ces deux réels est l'intégrale de Riemannsur [a,b] de f. On la note :
$$\int_{[a,b]}f$$
\end{de}
\subsection{Somme de Riemann}
\begin{prop}
Soit f fonction continue par morceaux sur [a,b].\\
Notons $\forall n \in N$:
Alors $(u_n)_{n \geq 0}$ et $(v_n)_{n \geq 0}$ converge vers $\int_{[a,b]}f$
\[\left\{\begin{array}{l}
   u_n = \frac{b-a}{n} \sum_{k=1}^n f(a+k\frac{b-a}{n} ) \\
   v_n = \frac{b-a}{n} \sum_{k=0}^{n-1} f(a+k\frac{b-a}{n} ) \\
  \end{array}\right.
\]
\end{prop}
\subsection{Linéarité}
\begin{prop}
Soient $\lambda$,$\mu$ réels, f, g fonctions continues par morceaux.
$$\int_{[a,b]}\lambda\varphi + \mu\varphi = \lambda\int_{[a,b]}\varphi + \mu\int_{[a,b]}\psi$$
\end{prop}
\subsection{Transmition de l'ordre}
\begin{prop}
 Si f et g sont continue par morceaux et $f \leq g$, alors :
$$\int_{[a,b]}f \leq \int_{[a,b]}g$$
\end{prop}
\subsection{Intégrale et valeur absolu}
\begin{prop}
Soit f fonction continue par morceaux sur [a,b], donc : 
$$\int_{[a,b]}|f| \geq \mid\int_{[a,b]}f\mid$$
\end{prop}
\subsection{Relation de Chasles}
\begin{prop}
Soit f continues par morceaux sur [a,b] et $b \in [a,c]$.
$$\int_{[a,c]}f = \int_{[a,b]}f + \int_{[b,c]}f$$
\end{prop}
\subsection{Inégalité de la moyenne}
\begin{prop}
Soit f,g continues par morceaux sur [a,b], g est bornée, avec a < b. Donc M = $\underset{[a,b]}\sup|g|$ existe.
$$|\int_{[a,b]}fg| \leq \underset{[a,b]}\sup|g|\times\int_{[a,b]}|f|$$
\end{prop}
\begin{de}
 $\dfrac{1}{b-a}\int_{[a,b]}g$ est la valeur moyenne de g sur [a,b].
$$\int_{[a,b]}g = \mu(b-a)$$
\end{de}
\section{Intégrale et primitive d'une fonction continue}
On obtient les propriétés suivant : 
\[\left\{\begin{array}{l}
   \mbox{Soit } x \in I.\mbox{ f est continue sur I, donc } \int_a^x f \mbox{ existe} \\
   \mbox{Si } x_0 \mbox{ est à l'intérieur de I, si f est continue sur I, alors } \int_a^x f \mbox { est aussi continue sur I} \\
   \mbox{ Si f est continue sur I, alors } g : x \mapsto \int_a^x f \mbox{ est dérivable sur I et sa dérivé est f.}
  \end{array}\right.
\]
De la dernière propriété, on déduit que g est de classe $C^1$ sur I. En résumé, si f est de classe $C^n$ alors g est de classe $C^{n+1}$
\begin{prop}
Si $\varphi$ est une fonction positive et continue sur [a,b] d'inégalité nulle, alors :
$$\varphi = 0$$
\end{prop}

\subsection{Utilisation des primitives d'une fonction continue}
\begin{de}
 Soit f définie sur un intervalle I. Une primitive de f sur I, c'est une fonction dérivable sur I dont la dérivée est f.
\end{de}
\subsection{Ensemble des primitives d'une fonction continue}
Soit f continue sur un intervalle I, si F est une primitive de f sur I, alors :
$$(\mbox{G est une autre primitive de f sur I}) \Leftrightarrow (\exists K \in \mathbb{R} \mbox{ tq } \forall x \in I~ G(x)=F(x)+K)$$
Il en découle que : 
$$( \mbox{F est une primitive de f sur I }) \Leftrightarrow (\exists K \in \mathbb{R} \mbox{ tq } \forall x \in I~ F(x)=\int_a^x f(t)dt + K)$$
Et que $\forall (a,b)^2 \in I^2$
$$\int_a^b f(t)dt = F(b) - F(a)$$
\subsection{Notation}
\begin{de}
 Si f est continue sur I : $\int f(x)dx$ désigne la valeur de x d'une primitive de f.
\end{de}
\subsection{Technique de calcul d'une intégrale}
\subsubsection{Intégrale par partie}
\begin{de}
 Si f,g sont de classe $C^1$ sur I, $(a,b)\in I^2$, alors :
$$\int_a^b fg' = [fg]_a^b - \int_a^b f'g$$
\end{de}
\subsubsection{Changement de variables}
\begin{de}
 Soit u une bijection de classe $C^1$ de $[\alpha,\beta]$ sur un intervalle [a,b]\\
Soit f continue sur [a,b].
$$\int_a^b f(u)du = \int_{\alpha}^{\beta} f(u(t))u'(t)dt$$
\end{de}
\subsection{Intégrale d'une fonction paire, impaire, periodique}
\subsubsection{Fonction paire}
\begin{prop}
Soit a $\in \mathbb{R}$\\
Si f est continue et paire sur [-a,a], alors :
$$\int_{-a}^a f = 2\int_0^a f$$
\end{prop}
\subsubsection{Fonction impaire}
\begin{prop}
Si f est continue et impaire sur [-a,a], alors :
$$\int_{-a}^a f = 0$$
\end{prop}
\subsubsection{Fonction periodique}
\begin{prop}
Si f est continue et T periodique sur $\mathbb{R}$\\
Soit a $\in \mathbb{R}$
$$\int_a^{a+T} \mbox{ f est indépendant de a}$$
\end{prop}
\section{Inégalité de Cauchy-Schwarz}
\begin{de}
Soient f,g continues sur [a,b] : 
$$|\int_{[a,b]}fg| \leq \sqrt{\int_{[a,b]}f^2\int_{[a,b]}g^2}$$
\end{de}
Si cette inégalité devient une égalité, alors $\exists \lambda_0 \in \mathbb{R}$ telque 
$$g = -\lambda_0f$$
\section{Formule de Taylor avec reste intégrale}
\begin{de}
Soit f fonction de classe $C^n$.\\
$\forall x \in D_f$, au voisinage de a :
$$f(x)=f(a)+....+\dfrac{(x-a)^{n-1}}{(n-1)!}f^{(n-1)}(a) + \int_a^x \dfrac{(x-a)^{n-1}}{(n-1)!}f^{(n)}(t)dt$$
\end{de}
\section{Inégalité de Taylor-Lagrange}
\begin{de}
Soit f de classe $C^{n+1}$ sur I, $a \in I$.\\
Supposons que $f^{(n+1)}$ soit majorée sur I.\\
Notons $M_{n+1} = \underset{I}\sup|f^{(n+1)}|$\\
$\forall x \in I$ :
$$|\int_a^x\dfrac{(x-t)^n}{n!}(t)dt| \leq \dfrac{(x-a)^{n+1}}{(n+1)!}M_{n+1}$$
\end{de}


