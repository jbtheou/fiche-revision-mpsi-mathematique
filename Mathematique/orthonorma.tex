\chapter{Procédé d'orthonormalisation de Gram-Schmidt}
Soit (u,v,w) une base de $\mathbb{R}^3$ ( On peut étendre cette méthode pour une base de $\mathbb{R}^n$)
On recherche une base orthonormée de $\mathbb{R}^3$ : ($e_1,e_2,e_3$).\\
Cette base doit vérifier : 
\begin{itemize}
 \item[$\rightarrow$]Vect(u) = Vect($e_1$)
 \item[$\rightarrow$]Vect($\{u,v\}$ = Vect($\{e_1,e_2\}$)
 \item[$\rightarrow$]Vect($\{e_1,e_2,e_3\}$) = $\mathbb{R}^3$
\end{itemize}
Pour $e_1$, on pose directement : 
$$e_1 = \dfrac{u}{||u||}$$
Pour $e_2$, on détermine tout d'abord un vecteur $A_2$, colinéaire à $e_2$. A l'aide d'une représation plan, on déduit que : 
$$A_2 = v + \lambda u$$
On détermine $\lambda$ sachant qu'il faut que : $$<A_2,e_1> = 0$$
On obtient donc $\lambda$. On obtient donc : 
$$e_2 = \dfrac{A_2}{||A_2||}$$
Pour $e_3$, on effectue une representation dans l'espace, et on déduit que :
$$A_3 = w + \lambda u + \mu v$$
On détermine $\lambda$ et $\mu$ sachant que $A_3$ doit vérifier : 
$$<A_3,e_1> = <A_3,e_2> = 0$$
Et on obtient $e_3$ : 
$$e_3 = \dfrac{A_3}{||A_3||}$$