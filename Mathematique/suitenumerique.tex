\chapter{Suite num\'erique - G\'eneralit\'e}
\section{Propriétés}
\subsection{Opérations}
Soit (a),(b),(c) trois suites. On peut effectuer trois types d'opérations sur les suites :
\begin{itemize}
 \item[$\rightarrow$] Une somme : 
$$((a)+(b) = (c)) \Leftrightarrow (\forall n \in N~ a+b =c)$$
 \item[$\rightarrow$] Un produit : 
$$((a).(b) = (c)) \Leftrightarrow (\forall n \in N~ a.b =c)$$
 \item[$\rightarrow$] Un produit par un scalaire : 
$$((a) = \lambda(c)) \Leftrightarrow (\forall n \in N~ a = \lambda c)$$
\end{itemize}
\section{Suites particulière}
\subsection{Suite arithmétiques}
Soit ($u_n$) une suite arithmétiques :
$$\sum_{k=0}^n u_k = (n+1).\dfrac{u_0+u_n}{2}$$
\subsection{Suite géométrique}
Soit ($u_n$) une suite géométrique, de raison q:
$$\sum_{k=0}^n u_k = u_0.\dfrac{1-q^{n+1}}{1-q}$$
\section{Suites vérifiant une relation de récurrence linéaire à coefficiants constants}
Soit ($u_n$) une suite vérifiant la récurrence :
$$u_{n+2} = a.u_{n+1} + b.u_{n}$$
Alors, on obtient l'équation caractéristique, en simplifiant par $r^n$ :
$$r^2 = ar+b$$
Donc :
$$\exists(A,B)\in \Re^2~ tq~ (u_n) = A(r_1)^n+B(r_2)^n$$
