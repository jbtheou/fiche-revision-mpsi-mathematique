\documentclass[a4paper,12pt,twocolumn]{report}

\usepackage[utf8x]{inputenc}			  % Utilisation du UTF8
\usepackage{textcomp}				  % Accents dans les titres
\usepackage [ french ] {babel}                    % Titres en français
\usepackage [T1] {fontenc} 			  % Correspondance clavier -> document
\usepackage[Lenny]{fncychap}                      % Beau Chapitre
\usepackage{dsfont}                    	  % Pour afficher N,Z,D,Q,R,C
\usepackage{fancyhdr}                             % Entete et pied de pages
\usepackage [outerbars] {changebar}               % Positionnement barre en marge externe
\usepackage{amsmath}				  % Utilisation de la librairie de Maths
%\usepackage{amsfont}				  % Utilisation des polices de Maths
\usepackage{cite}                                 % Citations de la bibliographie
\usepackage{openbib}                              % Gestion avancée de Bibtex
\usepackage{enumerate}				  % Permet d'utiliser la fonction énumerate
\usepackage{dsfont}				  % Utilisation des polices Dsfont
\usepackage{ae}					  % Rend le PDF plus lisible

\newtheorem{de}{Définition}
\newtheorem{theo}{Théorème}
\newtheorem{prop}{Propriété}

\title{Limite}
\author{MPSI}
\begin{document}
\begin{center}
\begin{Large}\textbf{ \textit{Limites remarquable}}
\end{Large}
\end{center}

\subsubsection{Fonctions trigonométrique}
$$\lim_{x \to 0} \dfrac{sin(x)}{x}=1$$
$$\lim_{x \to 0} \dfrac{1 - cos(x)}{x^2}=\dfrac{1}{2}$$
$$\lim_{x \to 0} \dfrac{arcsin(x)}{x}=1$$
$$\lim_{x \to 0} \dfrac{tan(x)}{x}=1$$
\subsubsection{Fonctions hyperbolique}
$$\lim_{x \to +\infty} \dfrac{ch(x)}{e^x}=\dfrac{1}{2}$$
$$\lim_{x \to +\infty} \dfrac{sh(x)}{e^x}=\dfrac{1}{2}$$
$$\lim_{x \to +\infty} \dfrac{ch(x)}{x}=1$$
$$\lim_{x \to +\infty} \dfrac{ch(x)-1}{x^2}=\dfrac{1}{2}$$
\subsubsection{Exponentielle}
$$\lim_{x \to +\infty} \dfrac{e^x}{x^{\alpha}}=+\infty$$
$$\lim_{x \to 0} \dfrac{e^x-1}{x}=1$$
$$\lim_{x \to -\infty} |x^{\alpha}|e^x=0$$
\subsubsection{Logarithme}
$$\lim_{x \to +\infty} \dfrac{(ln(x))^{\alpha}}{x^{\beta}}=0$$
$$\lim_{x \to 0} \dfrac{ln(1+x)}{x}=1$$
$$\lim_{x \to 0} x^{\alpha}|ln(x)|^{\beta}=0$$
\subsubsection{Polynomes}
$$\lim_{0} \dfrac{P}{Q}=\mbox{Limite des termes de plus bas degres}$$
$$\lim_{\infty} \dfrac{P}{Q}=\mbox{Limite des termes de plus haut degres}$$
\subsubsection{Autres}
$$\lim_{x \to 0} \dfrac{(1+x)^{\alpha}-1}{x}=\alpha$$
\subsubsection{Les formes indéterminée}
$$\dfrac{\infty}{\infty}$$
$$\infty-\infty$$
$$\infty \times0$$
$$1^{\infty}$$
\end{document}
