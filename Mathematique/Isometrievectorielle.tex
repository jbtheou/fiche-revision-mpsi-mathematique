\chapter{Isométrie Vectorielle}
\section{Généralités}
\begin{de}
Soit E un espace euclidien, soit f $\in L(E)$.\\
f est une isométrie vectorielle si : 
$$\forall x \in E~ ||f(x)||=||x||$$
\end{de}
\begin{prop}
Soit f $\in L(E)$. 
$$(\mbox{f est une isométrie vectorielle}) \Leftrightarrow (\forall x \in E~ \forall y \in E,~ <f(x),f(y)>=<x,y>)$$
\end{prop}
On dit que f est un endomorphisme orthogonale
\begin{prop}
Une isométrie vectorielle est bijective
\end{prop}
\begin{prop}
La réciproque d'une isométrie vectorielle est une isométrie vectorielle
\end{prop}
\begin{prop}
 La composée de deux isométries vectorielles est une isométrie vectorielle
\end{prop}
\begin{prop}
L'ensemble des isométrie vectorielles, munie de la loi de composition des applications est un groupe, appelé le groupe orthogonale de E, notée O(E)
\end{prop}
\begin{prop}
Soit f endomorphisme de E, et B base orthonormée.\\
$$(\mbox{f est une isométrie vectorielle}) \Leftrightarrow (\mbox{ f(B) est une base orthonomée})$$
$$(\mbox{f est une isométrie vectorielle}) \Leftrightarrow (mat_B(f) \mbox{ est une base orthogonale})$$
\end{prop}
\section{Isométrie vectorielle plane}
\subsection{Classification}
\begin{center}
% use packages: array
\begin{tabular}{|l|l|l|l|}
\hline
 Nom &Matrice & Déterminant & Vecteur invariant \\ \hline
Rotation d'angle $\theta$ & $\begin{bmatrix}
 \cos(x) & -\sin (x) \\
  \sin (x) & \cos(x) \\
\end{bmatrix}$  & +1 & $\{0\}$ ou E pour Ide \\ \hline
Symétrie $\bot$ / à une droite D & $\begin{bmatrix}
 \cos(x) & \sin (x) \\
  \sin (x) & -\cos(x) \\
\end{bmatrix}$  & -1 & Droite vectorielle \\ \hline
\end{tabular}
\end{center}
\subsection{Cas particulier des rotations}
\begin{prop}
L'ensemble des rotations vectorielle planes est un sous groupe de O(E), appelé groupe spéciale orthogonale. Il est notée SO(E)
\end{prop}
\subsection{Rotation orthogonales}
\begin{prop}
La composée de deux symétries orthogonales par rapport à deux droites est une rotation d'angle deux fois l'angle entre les deux droites.
\end{prop}
\section{Isométrie vectorielle d'un espace euclidien de dimension 3}
\subsection{Symétrie orthogonale}
Soit E un espace euclidien de base B=(i,j,k) orthonormée directe.\\
Soit F un sous espace de E, et $s_f$ la symétrie orthogonale par rapport à F : 
\begin{center}
% use packages: array
\begin{tabular}{|l|l|l|l|l|}
\hline
Définition de F & Base & Matrice & Déterminant & Type \\ \hline
F = $\{0\}$ & Quelconque & $\begin{bmatrix}
 -1 & 0 &  0\\
  0 & -1 & 0\\
 0 & 0 & -1 \\
\end{bmatrix}$ & -1 & $s_f$ = $h_{-1}$ \\ \hline
dim(F) = 1 & ($e_1,e_2,e_3$) D =Vect($e_1$)& $\begin{bmatrix}
 1 & 0 &  0\\
  0 & -1 & 0\\
 0 & 0 & -1 \\
\end{bmatrix}$& 1 & × \\ \hline
dim(F) = 2 & ($e_1,e_2,e_3$) P =Vect($e_1,e_2$) &  $\begin{bmatrix}
 1 & 0 &  0\\
  0 & 1 & 0\\
 0 & 0 & -1 \\
\end{bmatrix}$& -1 & Réflection \\ \hline
dim(F) = 3 & Quelconque & $\begin{bmatrix}
 1 & 0 &  0\\
  0 & 1 & 0\\
 0 & 0 & 1 \\
\end{bmatrix}$ & 1 & $s_f = Ide$ \\ \hline
\end{tabular}
\end{center}
\subsection{Propriété de la matrice d'un symétrie orthogonale dans une base orthonormée}
Si B est une base orthonormée et s une symétrie orthogonale.\\
Soit M=$mat_B(s)$.\\
Sachant que s est une symétrie, M est inversible et $M^{-1} = M$. De plus, la symétrie est orthogonale, donc la matrice l'est aussi, donc $M^{-1} = ^tM$.\\
On obtient donc M = $^tM$. Donc M est une symétrie.
\subsection{Rotation}
\begin{de}
Soit D une droite vectorielle orienté et $\theta$ un réel.\\
La rotation d'axe de D et d'angle $\theta$ est l'application linéaire r telque :
\begin{itemize}
 \item[$\rightarrow$]$\forall u \in D$ r(u)=u
 \item[$\rightarrow$]Le plan $D^{\bot}$ est stable par r et la restriction de r à ce plan est une rotation plane d'angle $\theta$ orienté par D.
\end{itemize}
\end{de}
\begin{prop}


Soit $B_=(e_1,e_2,e_3)$ base orthonormée directe telle que D=Vect($e_1$) et $D^{\bot}=Vect(e_2,e_3)$, alors : 
$$\begin{bmatrix}
 1 & 0 &  0\\
  0 & cos(\theta) & -sin(\theta)\\
 0 & sin(\theta) & cos(\theta) \\
\end{bmatrix}$$
Donc : $$det(r) = 1$$
 
\end{prop}
\begin{prop}
Cas particulier :
\begin{itemize}
 \item[$\rightarrow$] $\theta = 0$, alors r = Ide
 \item[$\rightarrow$] $\theta = \pi$, alors r=$s_D$
\end{itemize}
\end{prop}
\begin{prop}
Composée de deux réflections :\\
Soient P,P' deux plans distincts telque $P\cap P' = D$ :
\begin{itemize}
 \item[$\rightarrow$] Si $u \in D$ : $s_{P'}(s_P(u))=u$
 \item[$\rightarrow$] Si $u \in D^{\bot}$ r est stable dans $D^{\bot}$
  \item[$\rightarrow$] $s_{P'}o s_P(u)$ est une rotation d'axe D.
\end{itemize}
\end{prop}
\subsection{Calcul de l'image d'un vecteur de x par une rotation}
Soit r rotation d'axe D, orienté par $\overrightarrow{d}$, vecteur directeur de D, et d'angle $\theta$.\\
Soit $\overrightarrow{x} \notin D$ : 
$$r(\overrightarrow{x}) = \dfrac{<\overrightarrow{x},\overrightarrow{d}>}{||d||^2}.\overrightarrow{d} + cos(\theta)\dfrac{(\overrightarrow{d}\wedge\overrightarrow{x})\wedge\overrightarrow{d}}{||\overrightarrow{d}||^2}+sin(\theta)\dfrac{\overrightarrow{d}\wedge\overrightarrow{x}}{||\overrightarrow{d}||}$$
Si $||\overrightarrow{d}|| = 1$ : 
$$r(\overrightarrow{x}) = <\overrightarrow{x},\overrightarrow{d}>.\overrightarrow{d} + cos(\theta)(\overrightarrow{d}\wedge\overrightarrow{x})\wedge\overrightarrow{d}+sin(\theta)\overrightarrow{d}\wedge\overrightarrow{x}$$
\subsection{Classification}
Soit Inv(f) l'ensemble des vecteurs invariants.
\begin{center}
% use packages: array
\begin{tabular}{|l|l|l|l|l|}
\hline
dim(Inv(f)) & f & Base & Matrice & Det \\ \hline
3 & Ide & Quelconque & $\begin{bmatrix}
 1 & 0 &  0\\
  0 & 1 & 0\\
 0 & 0 & 1 \\
\end{bmatrix}$ & 1 \\ \hline
2 & Réflexion $s_p$ & Base adaptée & $\begin{bmatrix}
 1 & 0 &  0\\
  0 & 1 & 0\\
 0 & 0 & -1 \\
\end{bmatrix}$ & -1 \\ \hline
1 & Rotation d'axe D, d'angle $\theta$ & Base adaptée & $\begin{bmatrix}
 1 & 0 &  0\\
  0 & cos(\theta) & -sin(\theta)\\
 0 & sin(\theta) & cos(\theta) \\
\end{bmatrix}$ & 1 \\ \hline
0 & Réflexion o rotation  & Base adaptée & $\begin{bmatrix}
 1 & 0 &  0\\
  0 & cos(\theta) & -sin(\theta)\\
 0 & sin(\theta) & cos(\theta) \\
\end{bmatrix}$ & -1 \\ \hline
\end{tabular}
\end{center}
\subsection{Élements caractéristiques d'une rotation}
\begin{itemize}
 \item[$\rightarrow$] L'axe : L'ensemble des vecteurs invariante
 \item[$\rightarrow$] L'angle : 
 \begin{itemize}
 \item[$\rightarrow$] cos($\theta$) est obtenu par la Trace(r) = 1 + 2cos($\theta$) dans une base adaptée.
 \item[$\rightarrow$] Le signe de sin($\theta$) est obtenu par le signe de Det(x,r(x),d), avec x espace non invariant de l'espace, r(x) la rotation et d un vecteur directeur de l'axe
\end{itemize}
\end{itemize}
\subsection{Autres résultats}
\begin{prop}
La matrice d'une projection orthogonale dans une base orthonormée est symétrique
\end{prop}
