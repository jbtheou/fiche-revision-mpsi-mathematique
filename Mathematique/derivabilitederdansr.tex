\chapter{Dérivation des fonctions de $\mathbb{R}$ dans $\mathbb{R}$}
\section{Définitions}
\subsection{Définitions}
\begin{de}
Soit f définie sur un voisinage d'un réel a.
f est dérivable en a si :
$$\lim_{x \mapsto a} \dfrac{f(x)-f(a)}{x-a}$$
existe dans $\mathbb{R}$\\
Si f est dérivable, cette limite est le nombre dérivée de f en a.
\end{de}
\begin{prop}
Le nombre dérivé est la pente d'une droite passant par a. Cette droite est appelé tangente à la courbe représentative de f au point d'abscisse a.\\
L'équation de cette tangente est :
$$y = f(a) +f'(a)(x-a)$$
\end{prop}
\subsection{Lien entre tangente et dérivabilité}
Soit $x \in D_f$.
\begin{prop}
 Si on peut écrire f(x) sous la forme :
$$f(x) = f(a)+A(x-a)+\varepsilon(x)(x-a)$$
avec :
$$\lim_{x \rightarrow a} \varepsilon(x) = 0$$
alors f est dérivable en a et f'(a) = A.
\end{prop}
\subsection{Continuité et dérivabilité}
\begin{prop}
Si f est dérivable en a, alors f est continue en a
\end{prop}
\begin{prop}
f est dérivable en a si et seulement si :
$$\left\{\begin{array}{l}
    \mbox{ f est dérivable à droite en a}\\
    \mbox{ f est dérivable à gauche en a}\\
    f'_d(a) = f'_g(a)
  \end{array}\right.$$
\end{prop}

\subsection{Théorème de Rolle}
\begin{theo}
 Si :
$$\left\{\begin{array}{l}
     \mbox{ f est continue sur [a,b]}\\
     \mbox{ f est dérivable sur ]a,b[}\\
     f(a) = f(b)
  \end{array}\right.$$
alors : 
$$\exists c \in ]a,b[~ tq~ f'(c)=0$$
\end{theo}
\subsection{Théorème des accroissement finies}
\begin{theo}
Si :
$$\left\{\begin{array}{l}
     \mbox{ f est continue sur [a,b]}\\
     \mbox{ f est dérivable sur ]a,b[}\\
  \end{array}\right.$$
alors :
$$\exists c \in ]a,b[~ tq~ \dfrac{f(b)-f(a)}{b-a} = f'(c)$$
\end{theo}
\subsection{Inégalité des accroissement finies}
\begin{theo}
Si :
$$\left\{\begin{array}{l}
     \mbox{ f est continue sur [a,b]}\\
     \mbox{ f est dérivable sur ]a,b[}\\
     \mbox{ f' est borné sur ]a,b[}
  \end{array}\right.$$
Soit M un majorant de |f'|, alors :
$$|f(b) - f(a)| \leq M|b-a|$$
\end{theo}
\subsubsection{Conséquence}
\begin{itemize}
 \item[$\rightarrow$] Si f est croissante sur I, alors f'(a) $\geq 0$
 \item[$\rightarrow$] Si f et g sont dérivable sur [a,b], avec : $\forall x \in [a,b]~ f'(x) \leq g'(x)$, alors :
$$f(b)-f(a) \leq g(b)-g(a)$$
\end{itemize}
\subsection{Classe d'une fonction}
Soit f fonction, I un intervalle
\begin{de}
f est de classe $C^n$ sur I si $f^{(n)}$ est définie et continue sur I
\end{de}
\subsubsection{Opération}
\begin{prop}
Soit I un intervalle, soit $n \in N$.\\
La somme, le produit, la composé de fonction $C^n$ sur I, sont des fonctions $C^n$ sur I
\end{prop}
\subsection{Formulaire}
$\forall n \in N$, $\forall x \in \mathbb{R}$ :
$$cos^{(n)}(x) = cos(x+\dfrac{n\pi}{2})$$
$$sin^{(n)}(x) = sin(x+\dfrac{n\pi}{2})$$
\subsection{Formule de Leinbniz}
Soient f et g deux fonctions n fois dérivable sur I :
$$(fg)^{n)} = \sum_{k=0}^n \left( \dfrac{n}{k}\right) f^{(k)}.g^{(n-k)}$$
