\documentclass[a4paper,12pt,oneside]{report}

\usepackage[utf8x]{inputenc}			  % Utilisation du UTF8
\usepackage{textcomp}				  % Accents dans les titres
\usepackage [ french ] {babel}                    % Titres en français
\usepackage [T1] {fontenc} 			  % Correspondance clavier -> document
\usepackage[Lenny]{fncychap}                      % Beau Chapitre
\usepackage{dsfont}                    	          % Pour afficher N,Z,D,Q,R,C
\usepackage{fancyhdr}                             % Entete et pied de pages
\usepackage [outerbars] {changebar}               % Positionnement barre en marge externe
\usepackage{amsmath}				  % Utilisation de la librairie de Maths
%\usepackage{amsfont}				  % Utilisation des polices de Maths
\usepackage{cite}                                 % Citations de la bibliographie
\usepackage{openbib}                              % Gestion avancée de Bibtex
\usepackage{enumerate}				  % Permet d'utiliser la fonction énumerate
\usepackage{dsfont}				  % Utilisation des polices Dsfont
\usepackage{ae}					  % Rend le PDF plus lisible

\newtheorem{de}{Définition}
\newtheorem{theo}{Théorème}
\newtheorem{prop}{Propriété}
\newtheorem{conv}{Convention}
\newtheorem{loi}{Loi}

\begin{document}
\begin{center}
\begin{LARGE}Liste des TD Maple (Cours) \end{LARGE}
\end{center}\begin{itemize}
 \item \textbf{Introduction} : Stockage, Nombres réels, Nombres complexes \\
 \item \textbf{Equations Differentielles} : Manipulation des égalites, Résolution d'équation différentielles, Tracé \\
 \item \textbf{Arcs paramétrés} : Sequence, Tracé, Equation polaire, Enveloppe d'une famille de droite \\
 \item \textbf{Algorithmique} : Boucles et Instruction conditielle\\
 \item \textbf{Algorithmique (2)} : Procédure\\
\end{itemize}
\end{document}
