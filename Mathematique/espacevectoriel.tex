\chapter{Espace vectoriel}
\section{Définitions}
\begin{de}
Soit E un espace et K un corps. \\
Donc dit que (E,+,.) est un K-espace vectoriel si il vérifie les propriétés suivantes :
\begin{enumerate}[1-]
 \item + est une loi de composition interne : $$\forall(x,y) \in E^2~ x+y \in E$$
 \item + est une loi associative : $$\forall(x,y,z) \in E^3~ (x+y)+z = x+(y+z)$$
 \item + possède un élement neutre $O_E$ : $$\forall x \in E~ x+O_E = O_E+x =x $$
 \item Tous éléments x de E est symétrisable pour + dans E. Ce symétrique est $-x$ : $$\forall x \in E~ x+(-x) = (-x)+x = O_E$$
 \item + est commutatif dans E : $$\forall(x,y) \in E^2~ x+y=y+x$$
 \item "." est une loi de composition externe : $$\forall x \in E, \forall \lambda \in K,~ \lambda.x \in E$$
 \item "." possède un élement neutre $1_K$ : $$\forall x \in E~ 1_k.x = x$$
 \item "." vérifie : $$\forall x \in E, \forall(\lambda,\mu)\in K^2~ \lambda.(\mu.x) = (\lambda\times\mu).x$$
 \item "." vérifie : $$\forall x \in E, \forall(\lambda,\mu)\in K^2~ (\lambda+\mu).x = (\lambda.x)+(\mu.x)$$
  \item "." vérifie : $$\forall (x,y) \in E^2, \forall\lambda\in K~ \lambda.(x+y) = \lambda.x+\lambda.y$$
\end{enumerate}
Si un espace ne vérifie que les $4^{ere}$ propriétés, on dit que c'est un groupe. Si il vérifie les $5^{ere}$, c'est un groupe commutatif.
\end{de}
\begin{prop}
Soit E un K-espace vectoriel :\\
\begin{itemize}
 \item[$\rightarrow$] $\forall x \in E,~ 0_K.x = 0_E$\\
 \item[$\rightarrow$] $\forall x \in E~ -x = (-1_K).x$\\
 \item[$\rightarrow$] Soit x un vecteur de E, $\lambda \in K$ : $$(\lambda.x = O_E) \Leftrightarrow ( \lambda = O_K~ ou~ x=O_E)$$
\end{itemize}
\end{prop}

\section{Sous-espaces vectoriels}
\subsection{Définitions}
\begin{de}
Soit E un K-espace vectoriel. Soit F un espace.\\
On dit que F est un sous-espace de E si :
$$\left\{\begin{array}{l}
   F~ c~ E\\
   (F,+,.) \mbox{ est un K-espace vectoriel}
  \end{array}\right.$$
\end{de}
\begin{prop}
Soit E un K-espace vectoriel. Soient F et G deux sous espace de E :
\begin{itemize}
 \item[$\rightarrow$] $\{ O_E\}$ est le plus petit sous espace de E.
 \item[$\rightarrow$] $F\cap G$ est un sous espace de E
 \item[$\rightarrow$] $F\cup G$ est un sous espace de E 
\end{itemize}
\end{prop}
\subsection{Critère de reconaissance}
\begin{prop}
( F est un sous espace de E ) $\Leftrightarrow$ $\left\{\begin{array}{l}
   F~ c~ E\\
   F \neq \emptyset\\
   \forall(x,y) \in F^2, \forall (\lambda,\mu) \in K^2,~ \lambda x+\mu y \in F
  \end{array}\right.$
\end{prop}
\subsection{Sous espace supplémentaire}
Soit E un K-espace vectoriel. Soient F,G deux sous espace de E.
\begin{de}
 F et G sont dit en somme  direct si : $$F\cap G = \{ O_E \}$$
\end{de}
\begin{de}
 F et G sont supplementaire si ils sont en somme direct et que :  $$F + G = E$$
On le note : $$F \oplus G = E$$
\end{de}
\begin{prop}
Si F et G sont supplementaires, alors $$\forall x \in E$$, il existe un unique couple (x,y) avec $y\in F$,$z \in G$ telque : $$x = y+z$$
\end{prop}
\subsection{Partie génératrice d'un sous-espace}
\subsubsection{Sous espace engendré par un partie}
Soit A une partie de E.\\
Soit G le plus petit espace contenant A.
\begin{de}
G est le sous espace engendré par A. On le note : $$G = Vect(A)$$
On dit que A est une partie génératrice de G.
\end{de}
\subsubsection{Sous espace engendré par une partie fini}
Soit $u_1,...,u_n$ n vecteur de E. 
$$Vect(\{ u_1,...,u_n \}) = \{ \lambda_1u_1+...+\lambda_nu_n / \lambda_1,...,\lambda_n \in K^n \}$$
\subsection{Produit de deux espaces}
\begin{de}
Soient E et F deux K-espace vectoriel.\\
On munit le produit E$\times$F des deux lois suivant :
$$\forall(x,y) \in E\times F, \forall(x',y') \in E\times F, \forall \lambda \in K $$
$$\left\{\begin{array}{l}
   (x,y)+(x',y') = (x+x',y+y')\\
   \lambda.(x+y) = (\lambda x,\lambda y)
  \end{array}\right.$$
\end{de}
\begin{prop}
 (E$\times$F,+,.) est un K-espace vectoriel, de vecteur nul ($O_E,O_F$)
\end{prop}
\section{Application linéaire}
\begin{de}
Soient E et F deux K-espace vectoriels, f une application de E dans F.\\
f est une application linéaire si :
$$\forall(x,y) \in E^2~ \forall (\lambda,\mu) \in K^2~ f(\lambda x+\mu y) = \lambda f(x) + \mu f(y)$$
\end{de}
\subsection{Vocabulaire}
\begin{itemize}
 \item[$\rightarrow$] Application linéaire $\rightarrow$ Morphisme d'espace vectoriel
 \item[$\rightarrow$] Application linéaire de E dans E $\rightarrow$ Endomorphisme
 \item[$\rightarrow$] Application linéaire bijective $\rightarrow$ Isomorphisme
 \item[$\rightarrow$] Application linéaire bijective de E dans E $\rightarrow$ Automorphisme
\end{itemize}
On note $L(E,F)$ l'ensemble des applications linéaire de E dans F
\begin{prop}
Soit f isomorphisme de E dans F.\\
Alors $f^{-1}$ existe et est linéaire de F dans E
\end{prop}

\subsection{Noyau et Image d'une application linéaire}
Soit $f \in L(E,F)$
\subsubsection{Image}
\begin{de}
On appele image de f l'ensemble des images de tous les vecteurs de E par f :
$$Im(f) = \{f(x) / x\in E \}$$
Im(f) est un sous espace vectoriel de F
\end{de}
\subsubsection{Noyau}
\begin{de}
On appele noyau de f l'ensemble des antécédants $O_F$ par f :
$$Ker(f) = \{ x \in E / f(x) = 0 \}$$
Ker(f) est un sous espace vectoriel de E
\end{de}
\begin{prop}
f est une application injective si et seulement Ker(f) est réduit au vecteur nul :
$$(\mbox{f est injective}) \Leftrightarrow (Ker(f) = \{O_E\})$$
\end{prop}
\subsection{Opérations sur les applications linéaires}
\begin{itemize}
 \item[$\rightarrow$] La combinaison linéaire de deux applications linéaire est une application linéaire
 \item[$\rightarrow$] La composée de deux applications linéaire est linéaire
\end{itemize}
\subsection{Structure}
\begin{itemize}
 \item[$\rightarrow$] (L(E),+,o,.) est un K-Algèbre : On peut donc utiliser les identites remarquables
 \item[$\rightarrow$] GL(E) : Groupe des automorphisme de E. Dans ce groupe : $$(fog)^{-1} = g^{-1}of^{-1}$$
\end{itemize}
\subsection{Projecteur}
\begin{de}
Soit E un K-espace vectoriel, soient F et G deux sous espaces supplémentaire de E.\\
Soit $x \in E$
$$\exists !(y,z), y\in F, z\in G, x=y+z$$
On appelle projeté de x sur F parallement à G, notée p(x), le vecteur y.
\end{de}
\begin{prop}
p est une application linéaire
\end{prop}
\begin{prop}
Soit p la projection de F parallement à G :\\
 \begin{itemize}
 \item[$\rightarrow$] Im(p) = F\\
 \item[$\rightarrow$] Ker(p) = G\\
 \item[$\rightarrow$] pop = p\\
 \item[$\rightarrow$] $\forall x \in E$ (p(x) =x ) $\Leftrightarrow$ (x $\in$ F)\\
\end{itemize}
\end{prop}
\begin{prop}
Soit q la projection de G parallement à F. p et q sont deux projecteur associé.
 \begin{itemize}
 \item[$\rightarrow$] Im(q) = G\\
 \item[$\rightarrow$] Ker(q) = F\\
 \item[$\rightarrow$] poq = $O_E$\\
 \item[$\rightarrow$] p+q = Ide\\
\end{itemize}
\end{prop}
\subsubsection{Propriété caractéristique}
\begin{prop}
Si : 
$$\left\{\begin{array}{l}
   \mbox{ f est linéaire}\\
   fof = f
  \end{array}\right.$$
Alors f est une projection sur F parallement à G avec :
$$\left\{\begin{array}{l}
   F = \{ x\in E / f(x) = x\}\\
  G = Ker(f)
  \end{array}\right.$$
\end{prop}
\subsection{Symétrie}
\begin{de}
Soit E un K-espace vectoriel, soient F et G deux sous espaces supplémentaire de E.\\
Soit $x \in E$
$$\exists !(y,z), y\in F, z\in G, x=y+z$$
On appelle symétrie de x par rapport à F parallement à G :
$$s(x) = y-z$$
avec :
$$\left\{\begin{array}{l}
   (s(x) = x ) \Leftrightarrow ( x \in F)\\
   (s(x) = -x) \Leftrightarrow ( x \in G)
  \end{array}\right.$$
\end{de}
\begin{prop}
Soit s une symétrie :
\begin{itemize}
 \item[$\rightarrow$] s est une application linéaire
 \item[$\rightarrow$] sos = Ide, donc s est une bijection
\end{itemize}
\end{prop}
\subsubsection{Propriété caractéristique}
\begin{prop}
 Si : 
$$\left\{\begin{array}{l}
   \mbox{ f est linéaire}\\
   fof = Ide
  \end{array}\right.$$
Alors f est une symétrie
\end{prop}

