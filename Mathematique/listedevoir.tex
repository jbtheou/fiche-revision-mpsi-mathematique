\documentclass[a4paper,12pt,oneside]{report}

\usepackage[utf8x]{inputenc}			  % Utilisation du UTF8
\usepackage{textcomp}				  % Accents dans les titres
\usepackage [ french ] {babel}                    % Titres en français
\usepackage [T1] {fontenc} 			  % Correspondance clavier -> document
\usepackage[Lenny]{fncychap}                      % Beau Chapitre
\usepackage{dsfont}                    	          % Pour afficher N,Z,D,Q,R,C
\usepackage{fancyhdr}                             % Entete et pied de pages
\usepackage [outerbars] {changebar}               % Positionnement barre en marge externe
\usepackage{amsmath}				  % Utilisation de la librairie de Maths
%\usepackage{amsfont}				  % Utilisation des polices de Maths
\usepackage{cite}                                 % Citations de la bibliographie
\usepackage{openbib}                              % Gestion avancée de Bibtex
\usepackage{enumerate}				  % Permet d'utiliser la fonction énumerate
\usepackage{dsfont}				  % Utilisation des polices Dsfont
\usepackage{ae}					  % Rend le PDF plus lisible

\newtheorem{de}{Définition}
\newtheorem{theo}{Théorème}
\newtheorem{prop}{Propriété}
\newtheorem{conv}{Convention}
\newtheorem{loi}{Loi}

\begin{document}
\begin{center}
\begin{LARGE}Liste des devoirs\end{LARGE}
\end{center}\begin{itemize}
 \item \textbf{Devoir n°1} \textit{\textbf{(Maison)}} : Résolutions et démonstrations, Problème sur une fonction, Enigme \\
 \item \textbf{Devoir n°2} : Etude d'une suite, Tracer des courbes, Etude d'une fonction.\\
 \item \textbf{Devoir n°3} \textit{\textbf{(Maison)}} : Etude de fonctions (Fonction hyperbolique)\\
 \item \textbf{Devoir n°4} : Tracer une courbe (Fonction hyperbolique), Résolution d'une équation differentielle à coefficiant constant, Résolution d'une équation dans C  \\
 \item \textbf{Devoir n°5} \textit{\textbf{(Maison)}} : Ensemble de point en complexe, utilisations des affixes \\
 \item \textbf{Devoir n°6} \textit{\textbf{(Maison)}} : Equation polaire, Equation differentielle à coefficiant constant, Equation dans C \\
 \item \textbf{Devoir n°7} \textit{\textbf{(Maison)}} : Tracer un arc polaire, Ensemble de point dans un repère, Famille de courbes \\
 \item \textbf{Devoir n°8} : Enveloppe des normales à une ellipse, Equation differentielle, Lemniscate de Bernouilli \\
 \item \textbf{Devoir n°9} \textit{\textbf{(Maison)}} : Morphisme, Tribu, Ensemble \\
 \item \textbf{Devoir n°10} \textit{\textbf{(Maison)}} : Suites d'entier, Fractions continues \\
 \item \textbf{Devoir n°11} \textit{\textbf{(Maison)}} : Etude d'une suite (récurrence et algorithme) \\
 \item \textbf{Devoir n°12} : Etude d'une suite, Développement d'Engel \\
 \item \textbf{Devoir n°13} \textit{\textbf{(Maison)}} : Complexe, Polynome, Suite \\
 \item \textbf{Devoir n°14} \textit{\textbf{(Maison)}} : Ensemble vectoriel \\
 \item \textbf{Devoir n°15} : Etude d'une suite défini par récurrence, Endomorphisme, Polynome d'interpolation \\
 \item \textbf{Devoir n°16} \textit{\textbf{(Maison)}} : Classe d'une fonction, Etude d'une fonction, Algorithme \\
 \item \textbf{Devoir n°17} \textit{\textbf{(Maison)}} : Ensemble de suite, Dimension, Approximation polynomiale \\
 \item \textbf{Devoir n°18} : Classe d'une fonction, Polynome, Etude d'une fonction \\
 \item \textbf{Devoir n°19} \textit{\textbf{(Maison)}} : Etude d'une fonction, Espace vectoriel, Matrice \\
 \item \textbf{Devoir n°20} : Etude d'une fonction, Etude de Matrice, Espace Vectoriel \\
 \item \textbf{Devoir n°21} \textit{\textbf{(Maison)}} : Intégrale \\
 \item \textbf{Devoir n°22} \textit{\textbf{(Maison)}} : Etude d'une fonction, Intégrale \\
 \item \textbf{Devoir n°23} : Etude d'une fonction, Intégrale, Intégrale de Wallis, Fonction Hyperbolique \\
 \item \textbf{Devoir n°24} : Polynomes orthogonaux, Endomorphisme, Matrice \\
\end{document}
