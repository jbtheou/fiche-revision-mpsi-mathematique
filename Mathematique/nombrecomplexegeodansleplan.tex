\chapter{Nombres complexe et géométrie dans le plan}
\section{Alignement, Orthogonalité, Cocyclicité}
Soit $(\overrightarrow{AB};\overrightarrow{AC})$ l'angle formé par ces deux vecteurs.
$$(\overrightarrow{AB};\overrightarrow{AC}) = arg\left(\dfrac{c-a}{b-a}\right) ~ [2\pi]$$
Soit :$$z = \left(\dfrac{c-a}{b-a}\right) $$
\subsection{Alignements}
$$\mbox{A,B,C alignées} \Leftrightarrow \left(\dfrac{c-a}{b-a} \in \mathbb{R} \right) \Leftrightarrow arg\left(\dfrac{c-a}{b-a}\right) = 0  $$
$$\mbox{A,B,C alignées} \Leftrightarrow (z = \overline{z}) \Leftrightarrow (Det(\overrightarrow{AB};\overrightarrow{AC})) = 0$$
avec $Det(\overrightarrow{u},\overrightarrow{v)} = xy' - x'y$ 
\subsection{Orthogonalité}
$$\overrightarrow{AB};\overrightarrow{AC} \mbox{ orthogonaux} \Leftrightarrow \left(\dfrac{c-a}{b-a} \in i\mathbb{R} \right) \Leftrightarrow arg\left(\dfrac{c-a}{b-a}\right) = \dfrac{\pi}{2}~ [\pi]  $$
$$\overrightarrow{AB};\overrightarrow{AC} \mbox{ orthogonaux} \Leftrightarrow (z = -\overline{z}) \Leftrightarrow (\overrightarrow{AB}.\overrightarrow{AC}) = 0$$
\subsection{Cocyclicité}
Soit A,B,C trois points d'un cercle C de centre O.
$$(\overrightarrow{OB};\overrightarrow{OC}) = 2(\overrightarrow{AB};\overrightarrow{AC})~ [2\pi]$$
\subsubsection{Condition de cocyclicité}
\begin{prop}
 Si $(\overrightarrow{AB};\overrightarrow{AC}) = (\overrightarrow{DB};\overrightarrow{DC})~ [\pi]$ alors A,B,C,D sont soit cocyclique, soit alignés.
\end{prop}
\section{Similitude}
Soit z l'affixe de M, z' l'affixe de M'.
\subsection{Translation}
\begin{de}
 Soit $\overrightarrow{u}$ vecteur du plan. On appele translation de vecteur $\overrightarrow{u}$ l'application :
$$t_{\overrightarrow{u}} : P \rightarrow P$$
$$M \mapsto M'$$
avec $\overrightarrow{MM'} = \overrightarrow{u}$
\end{de}
\subsubsection{Expression analytique complexe}
soit $\alpha$ l'affixe de $\overrightarrow{u}$\
Alors :
$$z' = \alpha + z$$
\subsubsection{Bijectivité}
$t_{\overrightarrow{u}}$ est une application bijective. Soit :
$$t_{\overrightarrow{u}}^{-1} : P \rightarrow P$$
$$M \mapsto M'$$
avec z' = z - $\alpha$
\subsection{Homothetie}
\begin{de}
 Soit $\Omega$ un point d'affixe $\omega$. Soit $k \in \mathbb{R}$.\
On appele homothétie de centre $\Omega$ et de rapport k l'application :
$$h :  P \rightarrow P $$
$$M \mapsto M'$$
avec $\overrightarrow{\Omega M'}=k\overrightarrow{\Omega M}$ 
\end{de}
\subsubsection{Expression analytique complexe}
$$z'-\omega = k(z-\omega)$$
On détermine le centre d'une homothétie en déterminant son point fixe, donc en résolvant :
$$z = z'$$
\subsubsection{Bijectivité}
h est une application bijective. Soit :
$$h^{-1} : P \rightarrow P$$
$$M \mapsto M'$$
avec z' - $\omega$ = $\dfrac{1}{k}(z - \omega)$
\subsection{Rotation}
\begin{de}
Soit $\Omega$ un point d'affixe $\omega$ et $\theta$ un réel.\
On appele rotation de centre $\Omega$ et d'angle $\theta$ l'application
$$r : P \rightarrow P$$
$$M \mapsto M'$$
avec $\Omega M'$=$\Omega M$ et $(\overrightarrow{\Omega M};\overrightarrow{\Omega M'})$ = $\theta$ [$2\pi$]
\end{de}
\subsubsection{Expression analytique complexe}
$$z' - \omega = e^{i\theta}(z-w)$$
On détermine le centre d'une rotation en déterminant son point fixe, donc en résolvant :
$$z = z'$$
\subsubsection{Bijectivité}
h est une application bijective. Soit :
$$r^{-1} : P \rightarrow P$$
$$M \mapsto M'$$
avec z' - $\omega = e^{-i\theta}(z - \omega)$
\subsection{Similitude}
\begin{de}
Soit $\Omega$ un point d'affixe $\omega$ et ($\theta$,k) $\in \mathbb{R}^2$.\
On appele similitude direct de centre $\Omega$, d'angle $\theta$, et de rapport k l'application :
$$S : P \rightarrow P$$
$$ M \mapsto M'$$
avec $\Omega M'$=$k\Omega M$ et $(\overrightarrow{\Omega M};\overrightarrow{\Omega M'})$ = $\theta$ [$2\pi$]
\end{de}
\subsubsection{Expression analytique complexe}
$$z' - \omega  = ke^{i\theta}(z-w)$$
On détermine le centre d'une similitude en déterminant son point fixe, donc en résolvant :
$$z = z'$$
\subsubsection{Bijectivité}
S est une application bijective. Soit :
$$S^{-1} : P \rightarrow P$$
$$M \mapsto M'$$
avec z' - $\omega$ = $\dfrac{1}{k}e^{-i\theta}(z - \omega)$
\subsection{Affinité}
Soit $\varphi$ l'application défini par : 
$$\varphi : Plan \mapsto Plan$$
$$P(x,y) \mapsto M (x,\dfrac{b}{a}.y)$$
$\varphi$ est appelé affinité de base Ox, de direction Oy et de rapport $\dfrac{b}{a}$
