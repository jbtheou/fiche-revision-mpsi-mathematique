\chapter{Etude des suites}
\section{Suite complexe}
Soit $(z_n)$ une suite de complexe
\begin{prop}
$$((z_n)\mbox{ converge vers }\lambda ) \Leftrightarrow (|z_n - \lambda| \mbox{ converge vers 0 })$$
\end{prop}
\begin{prop}
$$((z_n)\mbox{ converge vers }\lambda ) \Leftrightarrow (Re(z_n) \mbox{ converge vers Re(}\lambda)~ et~ Im(z_n) \mbox{ converge vers Im(}\lambda))$$
\end{prop}
\begin{prop}
Si $(z_n)$ converge vers $\lambda$, alors $(|z_n|)$ converge vers |$\lambda$|.\\
Mais aucune information sur le comportement de l'argument.
\end{prop}
\begin{prop}
Si le module de $z_n$ converge vers R et que l'argument de $z_n$ converge vers $\alpha$, alors $(z_n)$ converge vers R$e^{i\alpha}$
\end{prop}
\begin{prop}
Soit $(z_n)$ suite complexe défini par :
$$\forall n \in \mathbb{N}~ z_n = a^n$$
Avec $a \in C$.
Si :
\begin{itemize}
 \item[$\rightarrow$] a=0, $(z_n)$ est une suite constante\\
 \item[$\rightarrow$] a $\in$ ]0,1[, $(z_n)$ converge vers 0\\
 \item[$\rightarrow$] |a| > 1, $(z_n)$ diverge vers $+\infty$\\
 \item[$\rightarrow$] |a| = 1.\\ \begin{itemize}
 \item[$\rightarrow$] Si a $\neq$ 1, alors la suite diverge\\
 \item[$\rightarrow$] Si a = 1, $(z_n)$ est une suite constante\\
\end{itemize}
\end{itemize}
\end{prop}
\section{Suites définies par récurrence}
\begin{de}
Soit f une fonction de $\mathbb{R}$ dans $\mathbb{R}$, définie sur $D_f$, et $(u_n)$ une suite définie telque :
$$u_0 \in \mathbb{R}$$
$$\forall n,~ u_{n+1} = f(u_n)$$
\end{de}
\subsection{Existance de la suite}
\begin{prop}
Soit $(u_n)$ une suite de réel, défini par récurrence à l'aide de la fonction f.
$$(\forall n,~ u_n \mbox{ existe}) \Leftarrow (u_0 \in D_f~ et~ D_f \mbox{ est stable par f})$$
$$(\forall n,~ u_n \mbox{ existe}) \Leftarrow (u_0 \in D_f~ et~ f(D_f) C D_f) $$
\end{prop}
\subsection{Sens de variation}
\begin{prop}
Soit $(u_n)$ une suite de réel, défini par récurrence à l'aide de la fonction f.
$$( (u_n) \mbox{ est croissante }) \Leftrightarrow (\forall n \in \mathbb{N},~ u_{n+1}\geq u_n)$$
$$( (u_n) \mbox{ est croissante }) \Leftrightarrow (\forall n \in \mathbb{N},~ f(u_n)\geq u_n)$$
En pratique, si $\forall x \in I,~ f(x) \geq x$ et si $\forall n \in \mathbb{N},~ u_n\in I$, alors ($u_n$) est croissante. 
\end{prop}
\subsection{Limite éventuelle}
Si $(u_n)$ converge vers l et que f est continue en l, alors l = f(l)
\section{Règle de d'Alembert}
Soit ($u_n$) une suite de réels positifs.\\
Supposons que :
$$\lim_{n \rightarrow \infty} \dfrac{u_{n+1}}{u_n} = a$$
\begin{itemize}
 \item[$\rightarrow$] Si a $\in \left[0,1\right[$, ($u_n$) converge vers 0
 \item[$\rightarrow$] Si a>1, $(u_n)$ diverge vers $+\infty$
\end{itemize}
\section{Comparaison des suites}
\subsection{Définitions}
\begin{de}
Soit $(u_n)$ et $(v_n)$ deux suites à valeur réelle.\\
On dit que $(v_n)$ domine $(u_n)$ si :
$$\exists A \in \mathbb{R}^{+}~ tq~ \forall n \in \mathbb{N}~ |u_n| \leq A.|v_n|$$
On note $u_n$=O($v_n$)
\end{de}
\begin{de}
On dit que $(u_n)$ est négligable devant $(v_n)$ ou que $(v_n)$ est préponderant devant $(u_n)$ si : 
$$\forall \varepsilon > 0~ \exists n_0 \in \mathbb{N}~ tq~ \forall n \geq n_0~ |u_n|\leq \varepsilon|v_n|$$
On note $u_n \ll v_n$ ou $u_n = o(v_n)$
\end{de}
\begin{de}
On dit que $(u_n)$ est équivalent à $(v_n)$ si ($u_n - v_n$) est négligable devant $(v_n)$
\end{de}
\subsection{Comparaison des suites de référence}
\begin{prop}
Soit $(c_n)$ une suite telleque :
$$\lim_{n \rightarrow +\infty}|c_n| = +\infty$$
Si $(a_n)$ converge vers 0, si $(b_n)$ converge vers l, l$\neq 0$, alors :
$$a_n \ll b_n \ll c_n$$
\end{prop}
\subsubsection{Comparaison des suites qui divergent vers $+\infty$}
Soit A > 1, $\alpha > 0$ :
$$ln(n) \ll n^{\alpha} \ll A^n \ll n! \ll n^n$$
\subsubsection{Comparaion des suites qui converge vers 0}
Soit B < 1, $\beta < 0$, alors :
$$0 \ll B^n \ll n^{\beta} \ll \dfrac{1}{ln(n)}$$
\subsubsection{Comparaion des suites convergente de limite non nul}
Soient l,l' deux réels non nuls.\\
Soient $(u_n)$ et ($v_n$) deux suites qui convergent respectivement vers l et l'.\\
Alors :
$$u_n \sim \dfrac{l}{l'}v_n$$
\section{Règles d'utilisation des équivalents et négligabilité}
\begin{prop}
Soient $(u_n)$, $(v_n)$ et $(w_n)$ trois suites.
$$u_n \ll v_n~ et~ v_n \ll w_n \Rightarrow u_n \ll w_n$$
$$u_n \sim v_n~ et~ v_n \sim w_n \Rightarrow u_n \sim w_n$$
$$u_n \sim v_n~ et~ v_n \ll w_n \Rightarrow u_n \ll w_n$$
$$u_n \ll v_n~ et~ v_n \sim w_n \Rightarrow u_n \ll w_n$$
\end{prop}
\begin{prop}
Soient $(u_n)$, $(v_n)$ deux suites et $(a_n),(b_n)$ deux autres suites.\\
Si :
$$(u_n)\sim(v_n)~ et ~ (a_n)\sim(b_n)$$
alors :
$$(u_n)(a_n) \sim (v_n)(b_n)$$
Ceci n'est pas vrai dans le cas de l'addition.
\end{prop}
\begin{prop}
Soient $(u_n)$, $(v_n)$ et $(a_n)$ trois suites et $\lambda,\mu$ deux réels de somme non nuls.\\
Si :
$$(u_n)\sim\lambda(a_n)~ et ~ (v_n)\sim\lambda(a_n)$$
Alors :
$$u_n + v_n \sim (\lambda+\mu)a_n$$
Si la somme des deux réels est nul, nous n'avons aucun résultats.
\end{prop}
\begin{prop}
Si $u_n \sim v_n$, alors $u_n^{\alpha} \sim v_n^{\alpha}$
\end{prop}
\subsubsection{Propriété des équivalents}
\begin{prop}
Soient $u_n$ et $v_n$ deux suites telque $u_n \sim v_n$.
\begin{itemize}
 \item[$\rightarrow$] Si $u_n$ diverge, alors $v_n$ diverge
\item[$\rightarrow$] Si $u_n$ converge, alors $v_n$ converge vers la même limite
 \item[$\rightarrow$] Si la limite de $u_n$ en l'infini est l'infini, alors la limite de $v_n$ en l'infini est aussi l'infini
\end{itemize}

\end{prop}

