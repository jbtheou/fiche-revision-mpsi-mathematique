
\chapter{Espaces vectoriels euclidiens}
\section{Produit scalaire}
\begin{de}
Soit E un $\mathbb{R}-$espace vectoriel et :
$$\varphi : E\times E \rightarrow \mathbb{R}$$
$\varphi$ est un produit scalaire sur E si :
\begin{itemize}
 \item[$\rightarrow$] $\varphi$ est bilinéaire
 \item[$\rightarrow$] $\varphi$ est symétrique
 \item[$\rightarrow$] $\varphi$ est positive
 \item[$\rightarrow$] $\varphi$ est définie
\end{itemize}
\end{de}
\subsection{Notation et Vocabulaire}
\begin{itemize}
\item[$\rightarrow$] Si $\varphi$ est un produit scalaire sur E, on note : $$\forall(u,v) \in E^2~ \varphi(u,v) = <u,v>$$
 \item[$\rightarrow$] On défini la norme de u par : $$\forall u \in E~ ||u|| = \sqrt{<u,u>}$$
 \item[$\rightarrow$] Sachant que $\varphi$ est définie, on obtient : $$\forall u \in E,~ (||u||=0) \Leftrightarrow (u=0)$$
 \item[$\rightarrow$] On dit que u et v sont orthogonaux si : $$<u,v>=0$$
 \item[$\rightarrow$] Un espace vectoriel est dit euclidien si : \begin{enumerate}[1-]
 \item E est de dimension finies 
 \item On a défini un produit scalaire sur E
\end{enumerate}
\end{itemize}
\section{Propriétés}
\begin{prop}
Soit E un $\mathbb{R}$-espace euclidien.\\
Soient u,v deux vecteurs de E : 
$$||u+v||^2 = ||u||^2 + 2<u,v> + ||v||^2$$
\end{prop}
\begin{theo}
Théorème de Pythagore : $$(\mbox{u et v sont orthogonaux}) \Leftrightarrow (||u+v|| = ||u||^2+||v||^2)$$
\end{theo}
\begin{prop}
 Soit $u \in E$, soit $\lambda \in \mathbb{R}$
$$||\lambda u|| = |\lambda|.||u||$$
\end{prop}

\begin{prop}
Inégalité de Cauchy : \\
Soient u,v deux vecteurs :
$$|<u,v>| \leq ||u||\times||v||$$
Si il y a égalité, alors u et v sont colinéaire
\end{prop}

\begin{prop}
Inégalité de Minkouskay :
$$||x+y|| \leq (||x||+||y||)$$
\end{prop}
\begin{prop}
Si $\{u_1,...,u_n\}$ sont des vecteurs non nuls et 2 à 2 orthogonaux, alors la partie est libre.
\end{prop}
\section{Base orthonormée}
\begin{de}
Soit ($e_1,...,e_n$) n vecteur de E, avec E espace de dimension n, deux à deux ortogonaux (famille orthogonale) et unitaire (famille normée) ( $\forall k~ ||e_k|| = 1$).\\
Alors,($e_1,...,e_n$) est une famille dites orthonormée, qui, de plus, est ici une base.
\end{de}
\begin{prop}
Tout $\mathbb{R}$ espace vectoriel euclidien de dimension finie admet au moins une base orthonormée.
\end{prop}
\begin{prop}
Soit B une base orthonormée de E. Soient u,v deux vecteurs de E de coordonnée respectif ($x_1,...,x_n$) et ($y_1,...,y_n$), alors :
$$<u,v> = x_1y_1+...+x_ny_n$$
\end{prop}
\begin{prop}
On défini dans ce cas la norme de u par : 
$$||u|| = \sqrt{x_1^2+...+x_n^2}$$
\end{prop}
\begin{prop}
Pour déterminer les coordonnées dans une base orthonormée, on détermine : 
$$\forall k \in \{1,...,p\}~ <u,e_k> = x_k$$
\end{prop}
\subsection{Matrice orthogonales}
\begin{de}
Une matrice de passage entre deux bases orthonormées est dites orthogonale.
\end{de}
\begin{prop}
 Soit B une base orthonormée. Soit B' une autre base. Soit P = $mat_B(B')$.
$$(\mbox{B' est orthogonale}) \Leftrightarrow (P^{-1} = ^tP)$$
\end{prop}
\begin{prop}
Soit P une matrice orthogonale :
$$det(P) = \pm 1$$
\end{prop}
\begin{prop}
Si P est orthogonale, alors $^tP$ est orthogonale et ($l_1,...,l_n$) forme aussi une base orthonormée de $\mathbb{R}^n$
\end{prop}
\subsection{Orientation de l'espace vectoriel}
\begin{de}
Soit E un espace vectoriel.\\
On oriente une base en définisant une base dites directe.
\end{de}
\begin{prop}
Soit $B_0$ une base directe.\\
Si B est une base de E : 
\[\left\{\begin{array}{l}
   det_{B_0}(B) > 0 \mbox{ B est directe}\\
   det_{B_0}(B) < 0 \mbox{ B est indirecte}\\
  \end{array}\right.\]
\end{prop}
\begin{prop}
Soit B,B' deux bases de E :
$$(det_B(B') > 0) \Leftrightarrow (\mbox{ B et B' ont la même orientation})$$
Dans le cas des bases orthonormée, on as : 
\end{prop}
\subsubsection{Bases orthonormée}
\begin{prop}
Soit $B_1$ une base orthonormée directe
\[\left\{\begin{array}{l}
   det_{B_1}(B) = 1 \mbox{ alors B est orthonormée directe}\\
   det_{B_1}(B) = -1 \mbox{ alors B est orthonormée indirecte}\\
  \end{array}\right.\]
\end{prop}
\begin{prop}
Soit $(u_1,...,u_n) \in \mathbb{R}^n$.\\
Si B et B' sont deux bases orthonormée directe : 
$$det_B(u_1,...,u_n) = det_{B'}(u_1,...,u_n)$$
Ce déterminant commun à toutes les bases orthonormée directe est notée Det($u_1,...,u_n$)
\end{prop}
\subsection{Orthogonalité et sous-espace}
Soit E un espace euclidien
\subsubsection{Sous espace orthogonaux}
\begin{de}
Soient F et G deux sous-espaces.
$$(\mbox{F est orthogonal à G})\Leftrightarrow (\forall x \in F,\forall y \in G,~ <x,y>=0)$$
\end{de}
\begin{prop}
Deux sous espaces orthogonaux sont en somme directe.
\end{prop}
\begin{prop}
On en déduit que : 
$$dim(F)+dim(G) \leq dim(E)$$
\end{prop}
\subsubsection{Orthogonal d'un sous-espace}
\begin{de}
Soit F un sous espace de E.\\
On appele orthogonale de F l'ensemble des vecteurs de E qui sont orthogonaux à tous les vecteurs de F.\\
On note :
$$F^{\bot} = \{x\in E / \forall y \in F~ <x,y>=0\}$$
\end{de}
\begin{prop}
$F^{\bot}$ est un espace vectoriel.
\end{prop}
\begin{prop}
F et $F^{\bot}$ sont deux sous espaces supplémentaire
\end{prop}
\begin{prop}
On en déduit que :
$$dim(F^{\bot})=dim(E) - dim(F)$$
\end{prop}
\begin{prop}
L'orthogonale de l'orthogonale de F :
$$(F^{\bot})^{\bot} = F$$
\end{prop}
\begin{prop}
Soit $\varphi \in L(E,\mathbb{R})$.\\
Soit u un vecteur de E de coordonée $(x_1,...,x_n)$. Il existe donc ($a_1,...,a_n$) telque :
$$\varphi(u) = a_1x_1+...+a_nx_n$$
On obtient donc : 
$$\varphi(u) = <a,u>$$
Par conséquence :
$$Ker(\varphi) = (Vect(a))^{\bot}$$
\end{prop}
\subsection{Projection orthogonale}
\begin{de}
La projection orthogonale sur un sous espace vectoriel F est la projection sur F parallèlement à $F^{\bot}$
\end{de}
\begin{de}
Soit B($e_1,...e_p$) une base orthonormée de F.\\
Soit x un vecteur de E de coordonnées ($x_1,...,x_p$).\\
On obtient donc le projetté orthogonale de x sur F, notée p(x) : 
$$p(x)=<x,e_1>e_1+...+<x,e_p>e_p$$
\end{de}
\begin{prop}
Si p est la projection orthogonale sur F. Si $u \in E$, alors :
$$Inf\{||x-y|| / y \in F \} = ||x - p(x)||$$
Et on note : 
$$d(x,F) = \underset{y \in F}\inf \parallel x - y\parallel$$
Et on appelle ceci distance de x à F.
\end{prop}
\begin{prop}
Si : 
\begin{itemize}
 \item[$\rightarrow$] $y \in F$
 \item[$\rightarrow$] $x - y \in F^{\bot}$
\end{itemize}
Alors y = p(x).
\end{prop}

