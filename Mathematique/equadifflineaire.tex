\chapter{Équations différentielle linéaire}
\section{Généralité}
\begin{de}
On considère (E) l'équation d'inconnue la fonction y n fois dérivable sur une partie A de $\mathbb{R}$ :
$$\forall x \in A,~ a_n(x).y^{(n)}(x)+...+a_0.y(x) = b(x)$$
avec : $a_n,..,a_0,b$ des fonctions définies sur A.\\
On dit que (E) est une équation differentielle linéaire d'ordre n.
\end{de}
\begin{nota}
On note cette équation : 
$$(E) :~ \forall x \in A :~  a_n(x).y^{(n)}+...+a_0.y = b(x)$$
\end{nota}
\begin{prop}
L'ensemble des solutions de (E) est soit:
\begin{itemize}
 \item[$\rightarrow$] Vide
 \item[$\rightarrow$] Un espace affine de direction l'espace vectoriel des solutions de l'équation sans second membre.\\
\end{itemize}
Si les coefficients sont constant.\\
L'ensemble des solutions est un espace de dimension n. 
\end{prop}
\chapter{Équations différentielles linéaire d'ordre 1}
\begin{de}
Soit A $\in \mathbb{R}$.\\
Soit a,b,c trois fonctions définies sur A, et (E):
$$(E) :~ a(x).y' + b(x).y = c(x)$$
$$(E_0) :~ a(x).y' + b(x).y = 0$$
Si :
\begin{itemize}
 \item[$\rightarrow$] Si a et b sont continues sur A
 \item[$\rightarrow$] A est un intervalle, notons le I
 \item[$\rightarrow$] $\forall x \in I~ a(x) \neq 0$
\end{itemize}
Alors : 
$$(E_0) \Leftrightarrow (\exists K \in \mathbb{R}~ tq~ \forall x \in I~ y(x)=K.e^{\int \frac{-b(x)}{a(x)}dx})$$
\end{de}
