\chapter{Intégrale}
\section{Étude de la fonction}
\subsection{Définition}
Soit $$g: x\mapsto \int_{u(x)}^{v(x)} f$$
Pour étudier le domaine de définition d'une fonction :
\begin{itemize}
 \item[{$\rightarrow$}] On travaille à x fixé : Soit $x \in \mathbb{R}$
 \item[{$\rightarrow$}] Puis : (f(x) existe) $\leftarrow$
$\left\{\begin{array}{l}
   \mbox{u(x) existe}\\
   \mbox{v(x) existe} \\
   \mbox{f est continue par morceaux sur [u(x);v(x)]}
  \end{array}\right. $
\end{itemize}
\section{Primitive}
Soit f défini sur un ensemble E : 
\begin{itemize}
 \item[$\rightarrow$] Si E n'est pas un intervalle, alors ????
 \item[$\rightarrow$] Si E est un intervalle : \begin{itemize}
 \item[$\rightarrow$]Si f n'est pas continue en un réel de E, alors ???
 \item[$\rightarrow$]Si f est continue sur E, alors f admet un ensemble de primitive

\end{itemize}


\end{itemize}

\subsection{Dérivabilité}
On défini une fonction primitive de f, qui est une fonction continue par morceaux, sur son domaine de définition. On la note F.\ On obtient :
$$g(x) = F(v(x)) - F(u(x))$$
Ce qui permet de dire que f est dérivable, comme F est par définition dérivable.
