\chapter{Arithmétique}
\section{Théorème de Bezout}
Pour résoudre une équation de la forme : 
$$a.u+b.v = c$$ 
d'inconnus (u,v), avec c le P.G.C.D de u et v, on détermine tout d'abord le P.G.C.D de u et v par le théorème d'Euclide. On procède de la façon suivante : 
\begin{enumerate}[1-]
 \item u = $q_1$.v + $r_1$
 \item v = $q_2$.$r_1$ + $r_2$
 \item $r_1$ = $q_3$.$r_2$ + $r_3$ ......
 \item Puis on arrive à : 
 \item $r_{n-1}$ = $q_{n-1}$.$r_n$ + $r_{n+1}$
 \item $r_n$ = $q_n.r_{n+1} + 0$
\end{enumerate}
Alors le P.G.C.D de u et v est le dernier reste non nul, $r_{n+1}$\\
Puis, par application du théorème de Bezout qui dit que : Il existe deux entier a et b tel que, si le P.C.G.D de u et v est d, alors : 
$$a.u+b.v = d$$
On exprime le premier reste en fonction de u et v, puis on retrouve les autres expressions qu'on injecte dans le première. On obtient au final a et b.