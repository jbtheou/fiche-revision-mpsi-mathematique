\chapter{Suite à valeur complexe}
\section{Convergence}
\begin{de}
 Soit ($z_n$) une suite à valeur complexe. On dit que cette suite converge vers $\lambda$ si et seulement si :
$$\forall \varepsilon > 0, \exists n_0 \in \mathbb{N}~ tq~ \forall n \geq n_0~ |u_n -\lambda|<\varepsilon$$
\end{de}
\section{Partie réeles, partie imaginaire}
\begin{prop}
Si Re($z_n$) converge vers a et Im($z_n$) converge vers b, alors $(z_n)$ converge vers a+ib.
\end{prop}
\section{Suites des modules et suites des arguments}
Soit ($z_n$) la suite défini par :
$$\forall n \in \mathbb{N}~ z_n = \rho_ne^{i\theta_n}$$
\begin{prop}
Si :
$$\left\{\begin{array}{l}
    (\rho_n) \mbox{ converge vers a}\\
    (\theta_n) \mbox{ converge vers b}
  \end{array}\right.$$
alors $(z_n)$ converge vers $ae^{ib}$
\end{prop}
Mais si la suite des arguments ne converge pas, la suite $(z_n)$ peut quand meme converger.
\section{Opération}
\subsection{Somme de deux suites convergente}
\begin{prop}
 Soient $(z_n)$ et $(z'_n)$ deux suites convergentes de limite $\lambda$ et $\lambda'$.\\
On obtient que $(z_n+z'_n)$ converge vers $\lambda+\lambda'$
\end{prop}

