\chapter{Fonctions de $\mathbb{R}^2$ dans $\mathbb{R}$}
\section{Caractérisation de l'existence d'une limite}
Pour montrer qu'il existe une limite, on peut procéder de trois façon differentes : 
\begin{itemize}
 \item[$\rightarrow$] Par utilisation de la définition : $$\forall \varepsilon > 0~ \exists \alpha > 0~ tq~ \forall (x,y)\in B((x_0,y_0),\alpha)~ :~ |f(x,y) -l| < \varepsilon$$
 \item[$\rightarrow$] Par utilisation des propriétés sur les sommes, produits et compositions
 \item[$\rightarrow$] À l'aide du théorème d'encadrement
\end{itemize}
\section{Caractérisation de la non-existence de limite}
Pour caractériser le fait qu'une limite n'existe pas, on utilise l'idée des "chemins".\\
On recherche tout d'abord deux couples différents, ($x_1,y_1$) et ($x_2,y_2$), qui converge vers (a,b) par exemple :  
\begin{itemize}
 \item[$\rightarrow$] $\underset{t_0}\lim x_1 = a$ 
 \item[$\rightarrow$] $\underset{t_0}\lim y_1 = b$ 
 \item[$\rightarrow$] $\underset{\alpha}\lim x_2 = a$ 
 \item[$\rightarrow$] $\underset{\alpha}\lim y_2 = b$ 
\end{itemize}
Puis si on obtient, avec l différents de l' :
\begin{itemize}
 \item[$\rightarrow$] $\underset{t \rightarrow t_0}\lim f(x_1(t),y_1(t)) = l$
 \item[$\rightarrow$] $\underset{u \rightarrow \alpha}\lim f(x_2(y),y_2(u)) = l'$
\end{itemize}
Alors on en déduit que la limite en (a,b) de f n'existe pas