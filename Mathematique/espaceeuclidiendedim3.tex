
\chapter{Espace euclidien de dimension 3}
\section{Définitions}
\section{Angle de deux vecteurs non nuls}
Soient $\overrightarrow{u},\overrightarrow{v}$ deux vecteurs non nuls, non colinéaire.\\
Soit $\overrightarrow{n}$ vecteur orthogonale à $\overrightarrow{u}$ et $\overrightarrow{v}$.
On obtient : 
$$\left\{\begin{array}{l}
   cos(\theta) = \dfrac{<\overrightarrow{u};\overrightarrow{v}>}{||\overrightarrow{u}||.||\overrightarrow{v}||}\\
   et~ : \\
   sin(\theta) = \dfrac{Det(\overrightarrow{u};\overrightarrow{v};\overrightarrow{n})}{||\overrightarrow{u}||.||\overrightarrow{v}||.||\overrightarrow{n}||}
  \end{array}\right.$$
\subsection{Produit vectoriel}
\begin{de}
 Soit ($\overrightarrow{i},\overrightarrow{j},\overrightarrow{k}$) une base orthonormée directe de E.\\
Le produit vectoriel est l'unique application de E$\times$E dans E :\\
\begin{itemize}
 \item Alternée : $\overrightarrow{u}\wedge\overrightarrow{v} = -\overrightarrow{v}\wedge\overrightarrow{u}$
 \item Bilinéaire
 \item Vérifiant : 
$\left\{\begin{array}{l}
\overrightarrow{i}\wedge\overrightarrow{j} = \overrightarrow{k}\\
\overrightarrow{j}\wedge\overrightarrow{k} = \overrightarrow{i}\\
\overrightarrow{k}\wedge\overrightarrow{i} = \overrightarrow{j}\\  
\end{array}\right.$
\end{itemize}
La définition du produit vectoriel est indépendante du choix de la base orthonormée.
\end{de}
\begin{prop}
Soit $\overrightarrow{u},\overrightarrow{v}$ deux vecteurs :
$$(\overrightarrow{u}\wedge\overrightarrow{v}=\overrightarrow{0}) \Leftrightarrow ( \overrightarrow{u}~ et~ \overrightarrow{v} \mbox{ sont colinéaire})$$
\end{prop}
\begin{prop}
Soient $\overrightarrow{u},\overrightarrow{v}$ deux vecteurs non colinéaire : 
$$\overrightarrow{u}\wedge\overrightarrow{v}\bot\overrightarrow{u}$$
\end{prop}
\begin{prop}
On obtient la propriété suivante, pour la norme du produit vectoriel : 
$$||\overrightarrow{u}\wedge\overrightarrow{v}|| = ||\overrightarrow{u}||.||\overrightarrow{v}||.|sin(\theta)|$$
\end{prop}
\subsection{Double produit vectoriel}
\begin{prop}
Soient $\overrightarrow{a},\overrightarrow{b},\overrightarrow{c}$ trois vecteurs :
$$\overrightarrow{a}\wedge(\overrightarrow{b}\wedge\overrightarrow{c}) = \overrightarrow{b}.(\overrightarrow{a}.\overrightarrow{c}) - \overrightarrow{c}.(\overrightarrow{a}.\overrightarrow{b})$$ 
\end{prop}

\subsection{Produit mixte}
\begin{de}
 Soient $\overrightarrow{u},\overrightarrow{v},\overrightarrow{w}$ trois vecteurs de E :
$$\overrightarrow{u}\wedge\overrightarrow{v}.\overrightarrow{w} = Det(\overrightarrow{u},\overrightarrow{v},\overrightarrow{w})$$
\end{de}

