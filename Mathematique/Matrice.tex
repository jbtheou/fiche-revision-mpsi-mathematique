
\chapter{Matrice et espaces vectoriel de dimension finies}
\section{Matrice}
\subsection{Définition}
\begin{de}
La matrice à n lignes et p colonnes est défini par :
$$M = [a_{i,j}]$$
avec i variant de 1 à n, et j variant de 1 à p
\end{de}
\subsection{Matrice carrée}
\begin{de}
Une matrice M est carrée si n=p. On défini la diagonale de A comme le n-uplet : $(a_{11},a_{22},...,a_{nn})$ 
\end{de}
\subsection{Vecteur ligne}
\begin{de}
On défini le vecteur ligne comme le p-uplet :
$$l_i = (x_{i,1},...,x_{i,p})$$
\end{de}
\subsection{Vecteur colonne}
\begin{de}
On défini le vecteur colonne comme le n-uplet :
$$l_j = (x_{1,j},...,x_{n,j})$$
\end{de}
\subsection{Matrice carrée particulière}
Soit T = $[x_{i,j}] \in M_n(K)$
\begin{de}
 On dit que T est triangulaire supérieur si :
$$T=\begin{bmatrix}
a_{1,1} & a_{1,2} & a_{1,3}  \\
0 & a_{2,2} & a_{2,3} \\
0 & 0 & a_{3,3} \\
\end{bmatrix}$$
\end{de}
\begin{de}
 On dit que T est triangulaire inférieur si :
$$T=\begin{bmatrix}
a_{1,1} & 0 & 0  \\
a_{2,1} & a_{2,2} & 0 \\
a_{3,1} & a_{3,2} & a_{3,3} \\
\end{bmatrix}$$
\end{de}
\begin{de}
 On dit que T est une matrice scalaire si :
$$T=\begin{bmatrix}
\lambda & 0 & 0  \\
0 & \lambda & 0 \\
0 & 0 &\lambda \\
\end{bmatrix}$$
\end{de}
Si $\lambda$=1, alors la matrice est la matrice unité, noté $I_{n}$
\subsection{Matrice carrée symétrique et antisymétrique}
\begin{de}
Soit S = $[x_{i,j}] \in M_n(K)$ \\
S est symétrique si $x_{i,j}=x_{j,i}$\\
S est antisymétrique si $x_{i,j}=-x_{j,i}$. Ceci implique que la diagonale de S est forcément nul dans ce cas
\end{de}
\subsection{Transposition et trace}
\begin{de}
La transposée de M noté $^{t}M$ est défini par :
$$M=\begin{bmatrix}
a & b & c & d\\
e & f & g & h  \\
i & j & k & l \\
\end{bmatrix}$$
alors
$$^{t}M=\begin{bmatrix}
a & e & i \\
d & f & j  \\
c & g & k  \\
d & h & l 
\end{bmatrix}$$
On a : 
$$^{t}(^{t}M) = M $$
\end{de}
\begin{de}
On défini la trace d'un matrice carrée comme la somme des termes de sa diagonale :
$$Trace(A) = \sum_{i=1}^nx_{k,k}$$
\end{de}
\subsection{Espace vectoriel des matrices}
On note $M_{n,p}(K)$ l'ensemble des matrices à n lignes et p colonne. \\
L'addition des matrices est une addition termes à termes \\
$(M_{n,p}(K),+)$ est un groupe commutatif d'élément neutre [O] \\
$(M_{n,p}(K),+,o)$ est un K espace vectoriel \\
Soit $$A_{1,1} = \begin{bmatrix}
1 & 0 & 0 & 0\\
0 & 0 & 0 & 0\\
0 & 0 & 0 & 0\\
\end{bmatrix}$$
L'ensemble {$A_{1,1},....,A_{n,p}$} est une base de $M_{n,p}(K)$ et dim($M_{n,p}(K)$)=n.p\\
$(M_n(K),+,x,o)$ est un K-algèbre de dimension $n^2$. Si AB=BA, alors les identités remarquables sont utilisables.
\subsection{Transposition}
\begin{de}
 Soit :
$$\varphi : M_{n,p}(K) \rightarrow M_{p,n}(K)$$
$$M \mapsto ^tM$$
$\varphi$ est un isomorphisme, donc c'est une application linéaire.
De plus, si la matrice est une matrice carrée :
$$\varphi \mbox{ est symétrie de }M_n(K)$$
\[\left.\begin{array}{l}
   ^tM=M \Leftrightarrow \mbox{M est symétrique}\\
    ^tM=-M \Leftrightarrow \mbox{M est antisymétrique} \\
  \end{array}\right\}
\mbox{Ce sont deux espaces supplémentaire}\]
\end{de}
\subsection{Produit de matrice}
\begin{de}
On défini le produit de matrice par : \\
Soit $l_1$ la première ligne de la matrice A\\
Soit $c_1$ la première colonne de la matrice B\\
Soit $a_1$ le première terme de la matrice produit\\
$$AB = \begin{bmatrix}
a & b & c & d
\end{bmatrix}\times\begin{bmatrix}
e \\
f \\
g \\
h \end{bmatrix} = [ae+bf+cg+dh]$$$$$$
avec [ae+bf+cg+dh] = $a_1$\
Le produit $l_2.c_1$ donne le terme $a_{2,1}$\
Le produit est non commutatif
\end{de}
\subsubsection{Cas particuliers}
Si $M \in M_{n,p}(K) et [0] \in M_{p,q}(K) $ alors :
$$M \times [0] = 0$$
Soit T une matrice scalaire $\in M_{p,q}(K)$, alors :
$$M \times T = \lambda M$$
Soit $I_n$ matrice unité d'ordre n, et $M \in M_{n}(K)$ alors :
$$MI_n=I_nM=M$$ 
\subsection{Transposition et trace du produit}
Soit A et B deux matrice $\in M_{n,p}(K)$. Alors :
$$^t(AB) = ^tB\times^tA$$
et 
$$\mbox{Trace(AB) = Trace (BA), mais généralement AB }\neq BA$$
\subsection{Matrice Carrée inversible}
\begin{de}
 Soit M $\in M_n(K)$. On dit que M est inversible si 
$$\exists N \in M_n(K) \mbox{ telque } MN = NM = I_n $$
On pose $N = M^{-1}$. On note $GL_n(K)$ l'ensemble des matrice carrée inversible d'ordre n. Cette ensemble est un groupe linéaire. Et :
$$(AB)^{-1} = B^{-1}A^{-1}$$
\end{de}
\begin{prop}
Soit (A,B)$\in (M_n(k))^2$ telque : $$AB = I_n$$
On obtient que:
\[\left\{\begin{array}{l}
   \mbox{ A est inversible et B = }A^{-1}\\
   \mbox{ B est inversible et A = }B^{-1} \\
  \end{array}\right.\]
\end{prop}

\subsubsection{Matrice carrée et inverse}
Voir Méthodologie.
\subsection{Rang d'une matrice}
\begin{de}
Le rang d'une matrice est le rang de ses vecteurs colonnes.
Si A $\in M_{n,p}(K)$ et qu'on note $c_i$ son $i^{eme}$ vecteurs colonnes, alors :
$$rang(A)=rang(\left\{c_1,...,c_p\right\})=dim(Vect \left\{c_1,...,c_p\right\})$$
et de plus : 
$$0 \leq rang(A) \leq Min\left\{n,p\right\} $$
\end{de}
\subsubsection{Rang de matrice particulière}
Le rang d'une matrice diagonale est r, si : 
$$\forall i \in \left\{1,....,r\right\} \lambda_{ii} \neq 0$$
Si dans une matrice carrée d'ordre n, les termes diagonaux sont non tous nul, alors $rang(A)=n$
\subsection{Opération élémentaire}
\begin{de}
 Il existe trois opérations élémentaire :
\begin{enumerate}[I) ]
 \item L'échange de deux colonnes : $c_i \leftrightarrow c_j$
 \item Le produit d'une colonne par un scalaire non nuls
 \item L'addition à une colonne d'une combinaison linéaire des autres
\end{enumerate}
Ces opérations élémentaire ne modifie par le rang de A
\end{de}
\section{Matrice et espaces vectoriel de dimension finies}
\subsection{Matrice de coordonnée d'un vecteur dans une base}
\begin{de}
Soit $B=\left\{e_1,...,e_n\right\}$ base d'un espace vectoriel de E\\
Soit $u \in E$, $\exists !(x_1,...,x_n) \in K^n$ telque : 
$$ u = x_1e_1+...+x_ne_n$$
On note $mat_{B}(u)=\begin{bmatrix}
x_1\\
.\\
.\\
x_n\\
\end{bmatrix} \in M_{1,n}(K)$ la matrice de de u dans B 
\end{de}
\subsection{Matrice d'une famille de vecteurs}
\begin{de}
 Si $ \forall j \in \left\{1,..p\right\}$, $u_j \in E$ et $mat_B(u)= \begin{bmatrix}
x_{1,j}\\
.\\
.\\
x_{n,j}\\
\end{bmatrix}$, alors :
$$mat_B(u_1,...,u_p) = \begin{bmatrix}
x_{1,1} & . & . & x_{1,p}\\
. & . & . & .\\
. & . & . & . \\
x_{n,1} & . & . & x_{n,p}\\
\end{bmatrix}$$
De plus, le rang de la matrice est le rang de la famille de vecteurs.
\end{de}
\subsection{Matrice de passage entre deux bases}
\begin{de}
 On note $mat_B(B')$ la matrice de passage de B à B'. On la note P
\end{de}
\subsection{Coordonnée d'un vecteur dans deux bases}
\begin{de}
On note B et B' deux bases de E.\\
On note $P=mat_B(B') \in M_n(K)$ la matrice de passage de B à B'\\
Soit $u \in E$\\
On pose $X=mat_B(u)$ et $X'=mat_{B'}(u)$ \\
On obtient la relation :
$$X = PX'$$
De plus, P est inversible, et son inverse est : 
$$P^{-1}=mat_{B'}(B)$$
\end{de}
\subsection{Matrice d'une application linéaire}
\begin{de}
Soit $f \in L(E,F)$. \\
Soit $B_E = \left\{e_1,...,e_p\right\}$ base de E et  $B_F = \left\{f_1,...,f_n\right\}$ base de F.\\
Soit M la matrice de f dans $B_E,B_F$ :
$$M = mat_{B_E,B_F}(f) = mat_{B_F}(\left\{f(e_1),...,f(e_p)\right\})$$
Le nombre de colonne de la matrice est défini par la dimension de l'espace de départ, celui des ligne par la dimension de l'espace d'arrivé
\end{de}
\subsubsection{Cas Particuliers}
\begin{enumerate}[I) ]
 \item La matrice de l'application nul $\in L(E,F)$ est la matrice nul de $M_{dim(F),dim(E)}(K)$\\
 \item La matrice d'un endomorphisme de E est une matrice carrée d'ordre dim(E)
 \item La matrice de l'identité est $I_n$
 \item La matrice de l'homothétie est de rapport k par rapport à $I_n$
\end{enumerate}
\subsection{Coordonnée de l'image d'un vecteur}
\begin{de}
Soit $B_E$ base de E, $B_F$ base de F. \\
Soit u $\in$ E \\
Posons $M = mat_{B_E,B_F}(f)$, $X = mat_{B_E}(u)$, $Y = mat_{B_F}(u)$. Alors :
$$Y = M.X$$
\end{de}
De plus : 
\begin{theo}
 Si f est une application de E dans F telque $\exists M \in M_{n,p}$ telque $\forall u \in E$, l'égalité ci-dessus est vérifié, alors f est une application linéaire
\end{theo}
\subsection{Unicité de la matrice, pour les bases fixes}
\begin{de}
Soient $B_E,B_F$ bases de E et de F, avec dim(E) = p, dim(F)=n
Soit : 
$$\varphi : L(E,F) \rightarrow M_{n,p}(K)$$
$$f \rightarrow mat_{B_E,B_F}(f)$$
$\varphi$ est une application linéaire. On en déduit donc que :
$$(f=g)\Leftrightarrow(mat_{B_E,B_F}(f)=mat_{B_E,B_F}(g))$$
\end{de}
\begin{prop}
Soit $f \in L(E,F)$, $B_E,B_F$ bases de E et F\\
Soit $A \in M_{n,p}(K)$ \\
Soit $x \in E$. Supposons que $X = mat_{B_E}(x)$ et $Y=mat_{B_F}(f(x))$, et qu'on obtient :
$$Y = AX $$
Alors $A = mat_{B_E,B_F}(f)$
\end{prop}
\subsection{Matrice et opérations}
\begin{de}
 Soient $B_E,B_F$ bases fixées de E et de F. \\
$\varphi$ est une application linéaire, c'est donc un isomorphisme. Nous avons en effet montré que $\varphi$ est bijective. On en déduit que $M_{n,p}(K)$ est de dimension finies, donc L(E,F) l'est aussi.
$$dim(L(E,F)=dim(E)\times dim(F)=dim(M_{n,p})$$
\end{de}
\subsection{Composée d'application linéaire}
\begin{de}
Soient E,F,G espaces vectoriel de dimension finie, et de bases respective $B_E,B_F,B_G$.\\
Soit $f \in L(E,F)$, $g \in L(F,G)$. Alors : 
$$mat_{B_E,B_G}(gof)=mat_{B_F,B_G}(g)\times mat_{B_E,B_F}(f)$$
\end{de}
\subsection{Matrice inversible et isomorphisme - Endomorphisme}
\begin{de}
Si f est un isomorphisme de E dans F, alors $mat_{B_E,B_F}(f)$ est inversible est : 
$$(mat_{B_E,B_F}(f))^{-1}=mat_{B_F,B_E}(f^{-1})$$
Si f est un endomorphisme, on a : 
$$(mat_{B_E}(f))^n = mat_{B_E}(f^n)$$
\end{de}
\subsection{Changement de bases}
\begin{de}
Soit $f \in L(E,F)$. Soient $B_E,B_{E'}$ bases de E. Soient $B_F,B_{F'}$ bases de F. \\
On pose : \[\left\{
  \begin{array}{ll}
    M = mat_{B_E,B_F}(f)\\
    M' = mat_{B_{E'},B_{F'}}(f) \\
    P = mat_{B_E}(B_{E'}) \\
    Q = mat_{B_{F'}}(B_{F})
  \end{array}\right.\]
Alors : 
$$M' = Q^{-1}MP$$
ou, si f est un endomorphisme :
$$M' = P^{-1}MP$$
\end{de}
\subsection{Trace d'un endomorphisme}
Soit f un endomorphisme, A,B deux matrices d'ordre n. On sait déjà que Trace(AB)=Trace(BA) et si :
\[\left\{
  \begin{array}{ll}
    M = mat_{B_E}(f)\\
    M' = mat_{B_{E'}}(f) \\
  \end{array}\right.\]
alors \[\left\{
  \begin{array}{ll}
   rang(M)=rang(M')=rang(f)\\
   Trace(M)=Trace(M')=rang(f) \\
   det(M)=det(M')=det(f)
  \end{array}\right.\]
\subsection{Matrice semblable}
\begin{de}
 Soient A,B deux matrice carrée d'ordre n. On dit que A et B sont semblable si $\exists$E espace vectoriel, $\exists B_E,B_{E'}$ bases de E, $\exists$f endomorphisme de E telque : 
$$B = P^{-1}MP$$
Ce qui revient à : 
$$(\mbox{A et B sont semblables}) \Leftrightarrow (\exists P  \in GL_n(K) \mbox{ telque } B = P^{-1}MP) $$
\end{de}
\subsection{Rang d'une application linéaire}
On peut toujours ramener la matrice dans des bases $B_{E'},B_{F'}$ de f à $J_r$ : 
$$J_r = \begin{bmatrix}
1 & . & . & . & 0\\
0 & 1 & . & . & .\\
. & . & 1 & . & . \\
. & . & . & . & . \\
0 & . & . & . & 0
\end{bmatrix}$$
Donc si : \\
f $\in$ L(E,F) tel qu'il $\exists B_{E'},B_{F'}$ bases de E et F telque $mat_{B_{E'},B_{F'}}(f)=J_r$, alors rang(f)=r\\
De plus, on a :
$$(^tP)^{-1}=^t(P^{-1})$$
On montre que $^tA$ et $^tJ_r$ sont semblable, donc le rang d'une matrice est le rang de ses vecteurs colonnes comme celui de ses vecteurs lignes.\\
Et on obtient : 
$$rang(^tA) = rang(A)$$
