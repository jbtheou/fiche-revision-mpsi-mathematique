\chapter{Vrac - Analyse}
\section{Équation de droite}
Si on a un point A et un vecteur directeur $\overrightarrow{u}$, alors on pose un point M et on détermine :
$$det(\overrightarrow{AM},\overrightarrow{u}) = 0$$
Si on a un point A et un vecteur directeur $\overrightarrow{n}$, alors on pose un point M et on détermine :
$$\overrightarrow{n}.\overrightarrow{AM}=0$$ 
\section{Etude d'une limite}
Comment étudier la limite en a de f ?\\
Soit f une fonction définie sur I, sauf peut etre en a.\\
Soit b$\in \bar{\mathbb{R}}$.\\
Pour déterminer si la limite en a de f est b : 
\begin{itemize}
 \item[$\rightarrow$] Regarder si f est une "composée" au voisinage de a de fonction continue.
 \item[$\rightarrow$] Utiliser le théorème d'encadrement
 \item[$\rightarrow$] Décomposer en limite à gauche et limite à droite
\end{itemize}
