\chapter{Limite d'une fonction}
\section{Définitions}
\begin{de}
Soit f une fonction, I un intervalle. \\
f est majorée sur I si :$$\exists m~ tq~ \forall x \in I~ f(x) < m$$
\end{de}
\begin{de}
On dit que f est croissante sur I si :
$$\forall (x,x') \in I^2~ si~ x<x',~ f(x) \leq f(x')$$
\end{de}
\subsection{Fonction k-lipschitzienne}
\begin{de}
Soit f : I $\rightarrow \mathbb{R}$.\\
f est k-lipschitzienne si : 
$$\forall (x,y) \in I^2~ |f(x)-f(y)|\leq k|x-y|$$
\end{de}
\begin{prop}
Soient f et g deux fonctions k-lipschitziennes sur I, alors f+g est aussi k-lipschitzienne sur I
\end{prop}
\subsection{Limite et continuité}
\subsubsection{Au voisinage d'un réel a}
Soit $a \in R$.
\begin{de}
La propriété P est vraie au voisinage de a si elle est vraie sur l'intersection de I et d'un intervalle ouvert de centre a :
$$(\exists \alpha > 0~ tq~ (P)~ soit~ vraie~ \forall x \in ]a-\alpha;a+\alpha[\cap I)$$
\end{de}
\subsubsection{Au voisinage de $+\infty$}
\begin{de}
(P) est vraie au voisinage de $+\infty$ si elle est vrai sur I$\cap]A;+\infty[$, avec A fixé.
\end{de}
\subsection{Limite}
\begin{de}
Soit (a,b) $\in \mathbb{R}^2$ :
$$(\lim_a f = b) \Leftrightarrow (\forall \varepsilon > 0~ \exists \alpha > 0~ \forall x \in D_f~ |x-a|\leq \alpha~ \Rightarrow~ |f(x)-b|<\varepsilon)$$ 
\end{de}
\begin{prop}
Soit a un réel, a = $+\infty$ ou a = $-\infty$.\\
On suppose que f et g coincident au voisinage de a, alors : 
$$\lim_a f = \lim_a g$$
\end{prop}
\subsection{Continuité}
\begin{de}
Si f est définie en a, la limite éventuelle en a est nécessairement b=f(a)
\end{de}
\begin{prop}
Soit $a \in \mathbb{R}$ :\\
\begin{itemize}
 \item Si a $\in D_f$
\begin{itemize}
 \item Si $\underset{a}\lim f = f(a)$, alors f est continue en a.
 \item Si $\underset{a}\lim f \neq f(a)$, alors c'est impossible.
 \item Si la limite n'existe pas, alors f n'est pas continue en a
\end{itemize}
\item Si a $\notin D_f$
\begin{itemize}
 \item Si $\underset{a}\lim f$ existe (dans $\mathbb{R}$), alors f est prolongable par continuité
 \item Si la limite n'existe pas, rien à dire, sauf que f n'est pas prolongable.
\end{itemize}
\end{itemize}
\end{prop}
\subsubsection{Caractérisation à l'aide de suite}
\begin{prop}
$$(\lim_a f = b) \Leftrightarrow (\forall(x_n) \mbox{ suite convergente de limite a, } f(x_n) \mbox{ converge vers b})$$
\end{prop}
On en déduit que : 
\begin{prop}
Si il existe $(u_n),(v_n)$ deux suite telque : 
\end{prop}
$$\left\{\begin{array}{l}
   \underset{n \mapsto +\infty}\lim (u_n) = a\\
   \underset{n \mapsto +\infty}\lim (v_n) = a\\
  \end{array}\right.$$
et avec b $\neq$ b': 
$$\left\{\begin{array}{l}
   \underset{n \mapsto +\infty}\lim (f(u_n)) = b\\
   \underset{n \mapsto +\infty}\lim (f(v_n)) = b'\\
   \end{array}\right.$$
alors $\underset{a}\lim f$ n'existe pas
\section{Limité ou continuité à gauche et à droite}
\subsection{Segment}
\begin{de}
Soit I C $\mathbb{R}$. I est un intervalle si $\forall (a,b) \in I^2~ \left[a,b \right] C I $\\
\end{de}
\begin{de}
Soit a $\in \mathbb{R}$, et I un intervalle.\\
a est interieur à I si : 
$$\exists \alpha > 0~ tq~ \left]a-\alpha,a+\alpha \right[ C I$$
L'interieur de I, notée $\overset{o}I$, est l'ensemble des points interieurs à I.
\end{de}
\subsection{Limite à droite, limite à gauche}
\subsubsection{Limite à droite}
\begin{de}
Soit f, fonction définie sur un intervalle I, sauf peut etre en a, avec a interieur à I.\\
La limite à droite de f en a est, si elle existe, la limite en a de la restriction de f à I$\cap \left]a,+\infty \right[$
On la note : 
$$\lim_{a^+} f$$
\end{de}
\subsubsection{Limite à gauche}
\begin{de}
Soit f, fonction définie sur un intervalle I, sauf peut etre en a, avec a interieur à I.\\
La limite à gauche, de f en a est, si elle existe, la limite en a de la restriction de f à I$\cap \left]-\infty,a \right[$
On la note : 
$$\lim_{a^-} f$$
\end{de}
\begin{prop}
Si f est défini au voisinage de a : \\
Si $a \in D_f$, la limite en a de f est b si et seulement si :
$$\left\{\begin{array}{l}
   \underset{a^+}\lim f = b\\
   \underset{a^-}\lim f = b\\
   f(a)=b\\
\end{array}\right.$$
Si $a \notin D_f$, la limite en a de f est b si et seulement si :
$$\left\{\begin{array}{l}
   \underset{a^+}\lim f = b\\
   \underset{a^-}\lim f = b\\
\end{array}\right.$$
\end{prop}
\begin{prop}
Si :
$$\lim_a f = b$$
et 
$$\lim_b g = c$$
Alors :
$$\lim_a gof = c$$
\end{prop}
\subsection{Continuité d'un intervalle}
\begin{prop}
Soit a $\in I$ :
$$(\lim_a f \mbox{ existe }) \Leftrightarrow (\mbox{ f est continue en a})$$
\end{prop}
\begin{prop}
Soit I un intervalle : 
$$(\mbox{ f est continue sur I }) \Leftrightarrow ( \forall a \in I, \mbox{ f est continue en a })$$
\end{prop}
\section{Image continue}
\subsection{D'un intervalle}
\subsubsection{Théorème des valeurs intermédiaires}
\begin{prop}
Soit I un intervalle, (a,b)$\in I^2$. Si f est continue sur I, et $y_0 \in \left[f(a),f(b)\right]$, alors :
$$\exists c \in \left[a,b\right]~ tq~ y_0 = f(c) $$
\end{prop}
\begin{prop}
Si f est continue sur I, un intervalle, si $a \in I$, et $b \in I$, et f(a) et f(b) sont de signes contraire, alors : 
$$\exists c \in \left[a,b\right]~ tq~ f(c) = 0$$
\end{prop}
\begin{prop}
Si I est un intervalle, et f continue sur I, alors f(I) est un intervalle.
\end{prop}
\subsection{D'un segment}
\begin{prop}
Soit f fonction continue sur [a,b], avec a et b réel.\\
Alors f est bornée sur [a,b]
\end{prop}
\begin{prop}
Soit f fonction continue sur [a,b], avec a et b réel.\\
Alors le Sup et l'Inf de la fonction sur [a,b] existent.
\end{prop}
\begin{prop}
Soit f continue sur [a,b]. Alors : 
$$\exists (m,M) \in \mathbb{R}^2, ~tq~ f(\left[a,b\right]) = \left[m,M\right] $$
\end{prop}
\section{Continuité uniforme sur un intervalle}
\begin{de}
Soit f fonction définie sur un intervalle I.\\
On dit que f est uniformement continue sur I si: 
$$\forall \varepsilon > 0,~ \exists \alpha >0~ tq~ \forall (x,y) \in I^2~ |x-y| < \alpha~ \Rightarrow |f(x)-f(y)|<\varepsilon$$
\end{de}
\begin{prop}
Si f est uniformement continue sur I, alors f est continue sur I.
\end{prop}
\begin{prop}
Une fonction k-lipschitzienne sur I est uniformement continue sur I.
\end{prop}
\begin{theo}
Théorème de Heine :\\
Toutes fonctions continue sur un segments [a,b] est uniformement continue sur le segment.
\end{theo}
\section{Fonction monotone}
\subsection{Théorème de la "limite monotone"}
\begin{prop}
Si f est croissante sur I, et a $\in \overset{o}I$, alors :
$$\lim_{a^-} f~ et~ \lim_{a^+} f \mbox{ existent, mais peuvent etre différentes}$$
\end{prop}
\subsection{Monotonie et continuité}
\begin{prop}
Soit f fonction définie et croissante sur I\\
$$( \mbox{ f est continue sur I })\Leftrightarrow(\mbox{ f(I) est un intervalle })$$
\end{prop}
\subsection{Théorème de la bijection}
\begin{prop}
Soit I un intervalle.\\
Si f est continue, strictement monotone sur I, alors f est une bijection de I sur l'intervalle f(I), et $f^{-1}$ est continue sur f(I).
\end{prop}

