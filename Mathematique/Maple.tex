\documentclass[a4paper,10pt,twocolumn]{report}

\usepackage[utf8x]{inputenc}			  % Utilisation du UTF8
\usepackage{textcomp}				  % Accents dans les titres
\usepackage [ french ] {babel}                    % Titres en français
\usepackage [T1] {fontenc} 			  % Correspondance clavier -> document
\usepackage[Lenny]{fncychap}                      % Beau Chapitre
\usepackage{dsfont}                    	  % Pour afficher N,Z,D,Q,R,C
\usepackage{fancyhdr}                             % Entete et pied de pages
\usepackage [outerbars] {changebar}               % Positionnement barre en marge externe
\usepackage{amsmath}				  % Utilisation de la librairie de Maths
%\usepackage{amsfont}				  % Utilisation des polices de Maths
\usepackage{cite}                                 % Citations de la bibliographie
\usepackage{openbib}                              % Gestion avancée de Bibtex
\usepackage{enumerate}				  % Permet d'utiliser la fonction énumerate
\usepackage{dsfont}				  % Utilisation des polices Dsfont
\usepackage{ae}					  % Rend le PDF plus lisible

\newtheorem{de}{Définition}
\newtheorem{theo}{Théorème}
\newtheorem{prop}{Propriété}

\title{Fiche Maple}
\author{MPSI}
\begin{document}
\begin{center}
\begin{LARGE}\textit{\textbf{ \begin{Large}Maple      \end{Large}
}}\end{LARGE}
\end{center}

\section{Introduction}
\begin{itemize}
\item[$\rightarrow$] Evalf : Evalutation numérique d'une expression
\item[$\rightarrow$] Digits : Nombre de décimales
\item[$\rightarrow$] Evalc : Evalution d'un complexe
\item[$\rightarrow$] Re ; Im ; Conjugate : Partie réelle, imaginaire et conjugée
\end{itemize}
\section{Fonction}
\begin{itemize}
 \item[$\rightarrow$] ?inifens : Liste des fonctions disponibles
\item[$\rightarrow$] simplify : Simplifie la fonction
\item[$\rightarrow$] expand : Développe la fonction
\item[$\rightarrow$] sort : Range les termes
\item[$\rightarrow$] subs : Substituer une variable par une valeur
\item[$\rightarrow$] diff : Dérive par rapport à une varible
\item[$\rightarrow$] plot : Trace une courbe, un ensemble de courbe et tout type d'arc
\item[$\rightarrow$] unapply : Transforme une équation en fonction
\item[$\rightarrow$] D : Opérateur de dérivation
\item[$\rightarrow$] pointplot : Trace un ensemble de point
\end{itemize}
\section{Equation}
\begin{itemize}
 \item[$\rightarrow$] solve : Résoud une équation ou d'un système d'équation
\item[$\rightarrow$] allvalues : Liste les racines d'un polynome
\item[$\rightarrow$] floor : Partie entier
\end{itemize}
\section{Equation differentielles}
\begin{itemize}
 \item[$\rightarrow$] rhs;lhs : Membre de droite d'une equation, membre de gauche
\item[$\rightarrow$] dsolve : Resoud une équation differentielle
\end{itemize}
\section{Arc paramétré}
\begin{itemize}
 \item[$\rightarrow$] seq : Création d'une séquance
\item[$\rightarrow$] with : Charge un paquet de commande
\end{itemize}
\section{Boucle et instruction conditionnelle}
\begin{itemize}
 \item[$\rightarrow$] while .... do ... od; 
\item[$\rightarrow$] for ... from .... to ... do ... od;
\end{itemize}
\subsection{Variante pour l'instruction for}
\begin{itemize}
\item[$\rightarrow$] for ... to ... do ... od; Ommet la valeur initial et la défini égale à 1
\item[$\rightarrow$] for ... from ... to ... by pas ... do ... od; Modifie le pas utilisé
\item[$\rightarrow$] for ... in ens do .. od; Fait varier les valeurs selon la liste ens
\end{itemize}
\section{Procédure}
On défini une procédure de la façon suivante :
\begin{itemize}
 \item[$\rightarrow$] Title := proc(parametre);local vars; ...... Resultat end;
\end{itemize}
On l'appele grâce à la commande :
$$Title(parametre);$$
\section{Calcul Matriciel}
On utilise le paquet linalg.
\begin{itemize}
 \item[$\rightarrow$] matrix : Crée une matrice 
\item[$\rightarrow$] transpose : Crée la transposé
\item[$\rightarrow$] diag : Crée un matrice diagonale
\item[$\rightarrow$] evalm : Évalue une matrice
\item[$\rightarrow$] $\mbox{A\&*B : Éffectue le produit AxB}$
\end{itemize}
\section{Développement limité}
\begin{itemize}
 \item[$\rightarrow$] series(f(h),h=0) : Développement limité de f en 0
 \item[$\rightarrow$] op(X) : Crée une liste contenant les coefficiants relatifs à chaque degres
\end{itemize}
\end{document}
