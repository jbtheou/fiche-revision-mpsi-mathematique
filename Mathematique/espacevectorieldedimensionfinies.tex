\chapter{Espace vectoriel de dimensions finies}
\section{Partie libre - Partie liée - Partie génératrice}
\subsection{Partie finie liée}
\begin{de}
Soient $u_1,...,u_p$ p vecteurs d'un K-espace vectoriels de E.\\
On dit que $\{u_1,...,u_n\}$ est liée ou que les vecteurs $u_1,...,u_n$ sont linéairement dépendants si $\exists(\lambda_1,...,\lambda_p)\in K^p$ non tous nuls telque : $$\lambda_1u_1+...+\lambda_pu_p = O_E$$
\end{de}
\begin{prop}
Toutes parties qui contient le vecteur nul est liée
\end{prop}
\begin{prop}
Si L est liée, et L c L', alors L' est liée.
\end{prop}
\subsubsection{Vecteurs colinéaires}
Soit u,v deux vecteurs de E.
$$(\mbox{ u et v sont colinéaire} \Leftrightarrow (\exists \lambda \in K~ tq~ u=\lambda v~ ou~ v=0))$$
\subsection{Partie fini libre}
\begin{de}
 Soit L partie finie de E.
$$(\mbox{L est libre}) \Leftrightarrow (\mbox{L n'est pas libre})$$
On la caractérise par : Si $\exists(\lambda_1,...,\lambda_p)\in K^p$ tq $\lambda_1u_1+...+\lambda_pu_p = O_E$ alors $$\lambda_1=...=\lambda_p=0$$
On dit que la partie est linéairemement indépendante.
\end{de}
\begin{prop}
Si L est libre, et L' c L, alors L' est aussi libre.
\end{prop}
\subsection{Partie génératrice}
Soit E un K espace vectoriel.
Nous avons l'ensemble des propriétés suivantes :
\begin{enumerate}[1-]
 \item G est une partie génératrice de E si : $$Vect(G)~ c~ E$$
 \item Si A et B sont deux parties de E, si A c B, alors : $$Vect(A)~ c~ Vect(B)$$
 \item Si G est une partie génératrice de E, G' une partie telque G c G', alors G' est une partie génératrice de E
\end{enumerate}
\subsection{Base}
\begin{de}
Une base d'un espace E est une partie génératrice de E et libre.
\end{de}
\begin{prop}
Si B = {$e_1,...,e_n$} est une base fini de E, alors, $\forall u \in E$, $\exists !$ n uplet de scalaire ($x_1,...,x_n$) telque : $$u = x_1e_1+...+x_ne_n$$
Le n-uplet ($x_1,...,x_n$) est appelé le n-uplet de coordonées de u dans la base B. On le note aussi : 
$$mat_B(u) = \begin{bmatrix}
 x_1 \\
  . \\
  . \\
 x_n \\
\end{bmatrix}$$
\end{prop}
\subsubsection{Base de référence}
\begin{itemize}
 \item[$\rightarrow$] $\mathbb{R}_n[X]$ : $\{$1,X,$X^2$,...,$X^n$$\}$\\
 \item[$\rightarrow$] $\mathbb{R}^n$ : $\{$(1,...,0),...,(0,...,1)$\}$
\end{itemize}
\section{Dimension d'un espace de dimension finie}
\begin{de}
Soit E un K-espace vectoriel.\\
Si E possède une partie génératrice finie, on dit que E est un espace de dimension finies.
\end{de}
\begin{prop}
Soit E un espace de dimension finies et G = $\{ u_1,...,u_p \}$ une partie génératrice de E.\\
Si G est libre, alors G est une base de E.
\end{prop}
\begin{prop}
De toute partie génératrice finie, on peut extraire une base.
\end{prop}
\begin{theo}
Théorème de la base incomplète :\\
Soit L = $\{ u_1,...,u_n \}$.\\
Si L est une partie libre, on peut la completer en une base.
\end{theo}
\begin{prop}
Lemme de Steiniz :\\
Si E possède une partie génératrice de n vecteurs, alors toute partie de n+1 vecteurs est liée
\end{prop}
\begin{theo}
 Si E est de dimension finie, toutes les bases de E ont le meme nombre d'élements et ce nombre commun est la dimension de l'espace.\\
Si B est une base de E :
$$dim(E) = card(B)$$
\end{theo}
\begin{prop}
Soit E un espace de dimension n, soit A une partie de E 
\begin{itemize}
 \item[$\rightarrow$] Si A est une partie génératrice de E, alors card(A) $\geq$ n
 \item[$\rightarrow$] Si card(A) < n, alors A n'est pas génératrice de E
 \item[$\rightarrow$] Si A est libre, alors card(A) $\leq$ n
 \item[$\rightarrow$] Si card(A) > n, alors A est liée.
\end{itemize}
\end{prop}
\subsection{Caractérisation des bases}
Soit E un espace vectoriel de dimension n et A une partie de E.\\
Si A est une base, alors A est libre, A est génératrice de E et card(A) = dim(E).\\
\begin{itemize}
 \item[$\rightarrow$] A est une base $\Leftrightarrow$ A est libre et card(A) = dim(E)
 \item[$\rightarrow$] A est une base $\Leftrightarrow$ A est génératrice de E et card(A) = dim(E)
\end{itemize}
\section{Sous-espace d'un espace de dimension finie}
Soit E un K-espace vectoriel de dimension finie :
\begin{prop}
Soit F sous espace de E. \\
F est de dimension finie et dim(F) $\leq$ dim(E).
\end{prop}
\begin{prop}
Soient F et G deux sous-espace de E :
$$(F = G) \Leftrightarrow \left\{\begin{array}{l}
   F~ c~ G\\
   dim(F) = dim(G)
  \end{array}\right.$$
\end{prop}

\begin{prop}
Formule de Grassman : \\
Soit F et G deux sous-espace de E :
$$dim(F+G) = dim(F) + dim(G) - dim(F\cap G)$$
\end{prop}
\begin{prop}
 Soit E espace de dimension finie.\\
Soient F et G deux sous espaces de E. F et G sont supplémentaire si :
$$\left\{\begin{array}{l}
   F+G = E \\
   dim(E) = dim(F) + dim(G)
  \end{array}\right.$$
\end{prop}
\begin{prop}
Tous sous-espace possède au moins un supplémentaire
\end{prop}
\begin{prop}
\begin{itemize}
 \item[$\rightarrow$]dim($\emptyset$) = 0
 \item[$\rightarrow$]Si D est un sous espace de dimension 1, c'est une droite vectorielle
 \item[$\rightarrow$]Si P est un sous espace de dimension 2, c'est un plan vectoriel
 \item[$\rightarrow$]Si H est un sous espace de dimension n-1, c'est un hyperplan
 \item[$\rightarrow$]Tout supplémentaires d'un hyperplan est une droite vectoriel
\end{itemize}
\end{prop}
\subsection{Rang d'une partie}
\begin{de}
Soit A une partie d'un espace E.\\
Le rang de A est la dimension du sous espace engendré par A :
$$rang(A) = dim(Vect{A})$$
\end{de}
\begin{prop}
Soit A c E : 
\begin{itemize}
 \item[$\rightarrow$] (rang(A) = dim(E)) $\Leftrightarrow$ (A est une partie génératrice de E)
 \item[$\rightarrow$] (A est libre) $\Leftrightarrow$ ($rang(A) = card(A)$)
\end{itemize}
\end{prop}
\subsection{Sous espace supplémentaire et base}
Soient F,G deux sous espaces de E, de base $B_F,B_G$ :
$$( \mbox{ F et G sont supplémentaire }) \Leftrightarrow \left\{\begin{array}{l}
  B_F \cup B_G = B_E \\
   B_F \cap B_G = \emptyset
  \end{array}\right.$$
\section{Application linéaire entre deux espaces de dimension finies}
Soient E et F deux K-espaces vectoriel de dimension finies
\subsection{Caractérisation par l'image d'une base de E}
Soit $B_E$ = {$e_1,...,e_p$} une base de E.\\
Soit u un vecteur de E telque :$$mat_B(u) = \begin{bmatrix}
 x_1 \\
  . \\
  . \\
 x_p \\
\end{bmatrix}$$
Si f $\in$ L(E,F), alors :
$$f(u) = x_1f(e_1)+...+x_pf(e_p)$$
\subsection{Image d'une partie libre, liée ou génératrice de E}
Soit f $\in$ L(E,F).\\
\begin{itemize}
 \item[$\rightarrow$]Si $L_1$ = {$u_1,...,u_p$} est une partie liée, alors {$f(u_1),...,f(u_p)$}\\
 \item[$\rightarrow$]Si $L_1$ = {$u_1,...,u_p$} est une partie libre, alors ??????\\
 \item[$\rightarrow$]L'image d'une partie génératrice de E est une partie génératrice de Im(f)
 \item[$\rightarrow$]Si B est une base de E, alors f(B) est génératrice de Im(f)
\end{itemize}
\subsection{Rang d'une application linéaire}
\begin{de}
Soit f $\in$ L(E,F).\\
Le rang de f est la dimension de l'image de f :
$$rang(f) = dim(Im(f))$$
\end{de}
\begin{prop}
(f est surjective) $\Leftrightarrow$ (rang(f) = dim(F))
\end{prop}
\subsection{Théorème du rang}
Soit f $\in$ L(E,F) :
$$dim(Ker(f)) + rang(f) = dim(E)$$
\begin{prop}
(f est injective) $\Leftrightarrow$ (rang(f) = dim(E))
\end{prop}
\subsection{Forme linéaire}
\begin{de}
Une forme linéaire d'un espace est une application linéaire de E dans son corps K.
\end{de}
\begin{prop}
Si $\varphi$ est une forme linéaire : $rang(\varphi) \leq 1$
\end{prop}

\begin{prop}
Le noyau d'une forme linéaire non nuls est un hyperplan
\end{prop}
\section{Isomorphisme}
\begin{de}
On considère E,F deux espaces de dimension finie.\\
On dit que E et F sont isomorphe si il existe un isomorphisme de E dans F.\\
Dans cette situation, on obtient :
\begin{itemize}
 \item[$\rightarrow$] Ker(f) = {$O_E$}
 \item[$\rightarrow$] Im(f) = F
 \item[$\rightarrow$] rang(f) = dim(F) = dim(E)
 \item[$\rightarrow$] Deux espaces isomorphe ont meme dimension
 \item[$\rightarrow$] Soit $B_E$ une base de E, f un isomorphisme, alors f($B_E$) est une base.
\end{itemize}
\subsection{Caractérisation des isomorphismes}
Soit $\varphi$ une application linéaire de E dans F :
\begin{prop}
 $$( \varphi~ est~ bijective) \Leftrightarrow \left\{\begin{array}{l}
  Ker(\varphi) = \{O_E\} \\
   dim(E) = dim(F)
  \end{array}\right.$$
 $$( \varphi~ est~ bijective) \Leftrightarrow \left\{\begin{array}{l}
  rang(\varphi) = dim(F) \\
   dim(E) = dim(F)
  \end{array}\right.$$
 $$( \varphi~ est~ bijective) \Leftrightarrow (\varphi(B_E) \mbox{ est une base de F})$$
\end{prop}
\end{de}
\subsection{Espace isomorphe}
\begin{theo}
Si F est un espace de dimension finie, si E et F sont isomorphe, alors E est aussi de dimension finie, et dim(E) = dim(F)
\end{theo}
