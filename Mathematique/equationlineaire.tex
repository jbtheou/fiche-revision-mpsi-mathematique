\documentclass[a4paper,12 pt,oneside]{report}     % Type de document
\usepackage[utf8x]{inputenc}			  % Utilisation du UTF8
\usepackage{textcomp}				  % Accents dans les titres
\usepackage [ french ] {babel}                    % Titres en français
\usepackage [T1] {fontenc} 			  % Correspondance clavier -> document
\usepackage[Lenny]{fncychap}                      % Beau Chapitre
\usepackage{dsfont}                    	  	  % Pour afficher N,Z,D,Q,R,C
\usepackage{fancyhdr}                             % Entete et pied de pages
\usepackage [outerbars] {changebar}               % Positionnement barre en marge externe
\usepackage{amsmath}				  % Utilisation de la librairie de Maths
%\usepackage{amsfont}				  % Utilisation des polices de Maths
\usepackage{cite}                                 % Citations de la bibliographie
\usepackage{openbib}                              % Gestion avancée de Bibtex
\usepackage{enumerate}				  % Permet d'utiliser la fonction énumerate
\usepackage{dsfont}				  % Utilisation des polices Dsfont
\usepackage{ae}					  % Rend le PDF plus lisible

\newtheorem{de}{Définition}
\newtheorem{theo}{Théorème}
\newtheorem{prop}{Propriété}

%$\begin{bmatrix}
%  \cos(x)-X & -\sin x \\
%  \sin x & \cos(x)-X \\
%\end{bmatrix}$

%opening
\title{Equations linéaires}
\author{MPSI}
\begin{document}
\maketitle
\tableofcontents
\chapter{Espace affine}
\begin{de}
Soit E un K-espace vectoriel.\\
Soit F un sous-espace vectoriel de E et $x_0 \in E$
$$\{ x_0 \} + F = \{x_0 +y / y \in F \}$$
est appelé espace affine de direction F.
\end{de}
\begin{prop}
Les espaces vectorielles sont des espaces affines particuliers.
\end{prop}
\begin{prop}
Si F est de dimension finies, on dit que $\{x_0\} + F$ est un espace affine de dimension finies.\\
Si F est une droite vectorielle, alors $\{x_0\} + F$ est une droite affine.
\end{prop}
\chapter{Equations linéaires}
\begin{de}
Soient E et F deux K-espace vectoriel.\\
Soit $f \in L(E,F)$. Soit $b \in F$.\\
L'équation d'inconnue $x$, vecteur de E :
$$f(x) = b$$
est appelé équation linéaire.
\end{de}
\section{Structure de l'ensemble des solutions}
\begin{itemize}
 \item[$\rightarrow$] Si $b \in Im(f)$, l'espace des solutions est un espace affine de dimenseion Ker(f)
 \item[$\rightarrow$] Si $b \notin Im(f)$, l'espaces des solutions est l'ensemble vide.
\end{itemize}
\begin{prop}
Si l'équation f(x) = b à une unique solution, alors :
\begin{itemize}
 \item[$\rightarrow$]$b \in Im(f)$
 \item[$\rightarrow$]Ker(f) = $\{ O_E \}$
\end{itemize}
Donc : 
\begin{itemize}
 \item[$\rightarrow$]$b \in Im(f)$
 \item[$\rightarrow$]f est injective
\end{itemize}
\end{prop}
\chapter{Système linéaire}
\begin{de}
Soit (S) un système d'inconnu ($x_1,...,x_p$).\\
Posons :
$$K^p \rightarrow K^n$$
$$(x_1,...,x_n) \rightarrow (a_{1,1}x_1+...+a_{1,p}x_p,...,a_{n,1}x_1+...+a_{n,p}x_p)$$
Notons : 
\begin{itemize}
 \item[$\rightarrow$] x = $(x_1,...,x_n)$
 \item[$\rightarrow$] b = $(b_1,...,b_n)$
\end{itemize}
On dit que :
\begin{itemize}
 \item[$\rightarrow$] f est l'application associée à (S)
 \item[$\rightarrow$] Le rang du système est le rang de A ou rang de f
 \item[$\rightarrow$] Si b $\in$ Im(f), alors S est un espace affine de dimension : $$ dim(S) = p-rang(A) = p-rang(S)$$
\end{itemize}
\end{de}
\section{Système de Cramer}
\begin{de}
Un système de Cramer est un système linéaire de n équations, à n inconnues, de rang n.
\end{de}
On obtient la formule :
$$x_j = \dfrac{1}{det(A)}det_{\varphi}(c_1,...,c_{j-1},b,c_{j+1},...,c_n)$$
\end{document}
