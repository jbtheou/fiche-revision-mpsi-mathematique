\documentclass[a4paper,12 pt,oneside]{report}     % Type de document
\usepackage[utf8x]{inputenc}			  % Utilisation du UTF8
\usepackage{textcomp}				  % Accents dans les titres
\usepackage [ french ] {babel}                    % Titres en français
\usepackage [T1] {fontenc} 			  % Correspondance clavier -> document
\usepackage[Lenny]{fncychap}                      % Beau Chapitre
\usepackage{dsfont}                    	  	  % Pour afficher N,Z,D,Q,R,C
\usepackage{fancyhdr}                             % Entete et pied de pages
\usepackage [outerbars] {changebar}               % Positionnement barre en marge externe
\usepackage{amsmath}				  % Utilisation de la librairie de 
\usepackage{amssymb}				  % Utilisation de la librairie de Maths
%\usepackage{amsfont}				  % Utilisation des polices de Maths
\usepackage{cite}                                 % Citations de la bibliographie
\usepackage{openbib}                              % Gestion avancée de Bibtex
\usepackage{enumerate}				  % Permet d'utiliser la fonction énumerate
\usepackage{dsfont}				  % Utilisation des polices Dsfont
\usepackage{ae}					  % Rend le PDF plus lisible

\newtheorem{de}{Définition}
\newtheorem{theo}{Théorème}
\newtheorem{prop}{Propriété}

%$\begin{bmatrix}
%  \cos(x)-X & -\sin x \\
%  \sin x & \cos(x)-X \\
%\end{bmatrix}$

%opening
\title{Généralités - Algèbre}
\author{MPSI}
\begin{document}
\maketitle
\tableofcontents
\chapter{Théorie des ensembles}
\begin{de}
 Un ensemble est un composé d'élements
\end{de}
Soit E un ensemble, soit a un élement de E \\
On note l'appartenance de a à l'ensemble par : $$a \in E$$
Soit F un ensemble inclue dans E.\\ On note cette inclusion par : $$F~ c~ E$$
On dit aussi de F est une partie de E. On le note :
$$F = P(E)$$
\section{Opération sur les parties d'un ensemble}
Il existe trois opération principales sur les ensembles :
\begin{itemize}
 \item[$\rightarrow$] $A \cup B = \left\lbrace x \in E / x\in A \mbox{ ou } x \in B \right\rbrace$ 
 \item[$\rightarrow$] $A \cap B = \left\lbrace x \in E / x\in A \mbox{ et } x \in B \right\rbrace$ 
 \item[$\rightarrow$] $A \diagup B = \left\lbrace x \in E / x\in A \mbox{ et } x \notin B \right\rbrace$ 
\end{itemize}
\section{Quantificateurs}
Soit P une propriété
\begin{itemize}
 \item[$\rightarrow$] P(x) est vrai pour tout élement x : $$\forall x \in E~ P(x)$$ 
 \item[$\rightarrow$] Il existe un x pour lequel P(x) est vraie : $$\exists x \in E ~ P(x)$$
 \item[$\rightarrow$] Il existe un unique x pour lequel P(x) est vraie : $$\exists ! x \in E ~ P(x)$$
\end{itemize}
\end{document}
