\chapter{$\mathbb{R}$}
\section{D\'efinitions}
\section{Structure}
\begin{de}
($\mathbb{R}$,+,$\times$) est un corps totalement ordonnée. On dit qu'il est archimédien.\\
\end{de}
\begin{de}
La relation "$\leq$" est une relation d'ordre. Elle est :
\begin{itemize}
 \item[$\rightarrow$] Reflexive : $$\forall x \in \mathbb{R}~ x \leq x$$
 \item[$\rightarrow$] Anti-symétrique : $$\forall(x,y)\in \mathbb{R}^2 \mbox{ si } : (x \leq y,y\leq x),  \mbox{ alors } x=y$$
 \item[$\rightarrow$] Transitive : $$\forall(x,y,z)\in \mathbb{R}^3 \mbox{ si } : (x \leq y,y\leq z), \mbox{ alors } x\leq z$$
\end{itemize}
\end{de}
\subsection{Majorant - Minorant}
Soit A un ensemble
\subsubsection{Majorant}
\begin{de}
Si M est un majorant de A, avec $M \in A$, alors :
$$M = Max(A)$$
\end{de}
\begin{de}
Si M est le plus petit des majorants de A, alors M est la borne supérieure de A : 
$$M = Sup(A)$$
\end{de}
\begin{prop}
Si $A~ c~ \mathbb{R}$, si Max(A) existe, alors Sup(A) existe et :
$$Sup(A) = Max(A)$$
\end{prop}
\subsubsection{Minorant}
\begin{de}
Si M est un minorant de A, avec $M \in A$, alors :
$$M = Min(A)$$
\end{de}
\begin{de}
Si M est le plus grand des minorant de A, alors M est la borne inférieure de A : 
$$M = Inf(A)$$
\end{de}
\begin{prop}
Si $A~ c~ \mathbb{R}$, si Min(A) existe, alors Inf(A) existe et :
$$Min(A) = Inf(A)$$
\end{prop}
\subsection{Borne supérieure - Borne inférieure}
\begin{prop}
Toute partie de $\mathbb{R}$ non vide et minorée possède une borne inférieure.
\end{prop}
\begin{prop}
Toute partie de $\mathbb{R}$ non vide et majorée possède une borne supérieure.
\end{prop}
\begin{prop}
Toute partie de Z non vide et majorée possède un plus grand éléments. (Max)
\end{prop}
\subsection{Partie bornée de $\mathbb{R}$}
Soit $A$ une partie de E. On note ceci : $A \in P(E)$.\\
A est bornée si et seulement si :$$\exists M \in \mathbb{R}~ tq~ \forall a\in A,~ |a|\leq M$$
\begin{prop}
Propriété d'Archimède : Soient (x,y)$\in \mathbb{R}$ et x>0, alors :
$$\exists p \in Z~ tq~ y < px$$
\end{prop}
\subsection{Partie entière}
\begin{de}
Soit $x \in \mathbb{R}$.\\
Il existe un unique entier p telque p$\leq x$<p+1\\
Cette entier p en la partie entière de $x$. On le note $E(x)$.
\end{de}
\begin{de}
En complément, on défini la partie décimale de $x$, notée $D(x)$ :
$$D(x) = x-E(x)$$
\end{de}
\subsection{Densité}
\begin{de}
Soit A une partie de $\mathbb{R}$\\
A est dense dans $\mathbb{R}$ si, avec $x \neq y$ :
$$\forall(x,y) \in \mathbb{R}^2~ \exists a \in A~ tq~ a\in]x;y[$$
\end{de}
\begin{prop}
Puisque l'espace des fractions rationnels, notée Q, est dense dans $\mathbb{R}$, si $x \in \mathbb{R}$, alors il existe une suite de rationnelle qui converge vers x. 
\end{prop}
\section{Partie de $\mathbb{R}$}
\begin{de}
 Soient (a,b)$\in \mathbb{R}^2$. On appelle segment d'extrémité a,b :
$$[a,b] = \left\lbrace x\in \mathbb{R} / a\leq x \leq b\right\rbrace $$
\end{de}
\begin{de}
 Soit I une partie de $\mathbb{R}$. I est un intervalle si :
$$\forall x\in I, \forall y \in I,~ [x;y]~c~I$$
\end{de}
\subsection{Sous-groupes de ($\mathbb{R}$;+)}
\subsubsection{Critère de reconnaissance des sous-groupes}
\begin{de}
Soit H une partie de $\mathbb{R}$.\\
On dit de H est un sous-groupe de ($\mathbb{R}$;+) si (H;+) est un groupe. 
\end{de}
\begin{prop}
H est un sous-groupe si et seulement si :
\begin{enumerate}[1-]
 \item $H~ c~ \mathbb{R}$ et H non vide
 \item $\forall(x;y)\in H^2, x-y\in H$
\end{enumerate}

\end{prop}

