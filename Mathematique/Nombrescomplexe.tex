\chapter{Nombres complexes}
\section{Formules}
\subsection{Généralités}
\begin{itemize}
\item[{$\rightarrow$}] (C,+,x) est un corps
 \item[{$\rightarrow$}] (a+ib)+(a'+ib') = (a+a')+i(b+b')\\
 \item[{$\rightarrow$}] (a+ib).(a'+ib') = (aa'-bb')+i(ab'+a'b)\\
 \item[{$\rightarrow$}] Il y a unicité de la partie réelle et de la partie imaginaire pour un complexe.\\
 \item[{$\rightarrow$}]$\overline{z+z'}=\overline{z}+\overline{z'}$\\
 \item[{$\rightarrow$}]$\overline{z.z'}=\overline{z}.\overline{z'}$\\
 \item[{$\rightarrow$}]Re(z) = $\dfrac{z+\overline{z}}{2}$\\
\item[{$\rightarrow$}]Im(z) = $\dfrac{z-\overline{z}}{2}$\\
\item[{$\rightarrow$}]$z_{\overrightarrow{AB}} = z_b-z_a$ \\
\item[{$\rightarrow$}]On défini le barycentre de la façon suivante : $$\alpha + \beta \neq 0, z_g = \dfrac{\alpha z_A + \beta z_B}{\alpha + \beta}$$
\end{itemize}
\subsection{Forme Trigonométrique et exponentielle}
Soit z = x+iy un complexe.
\begin{itemize}
 \item[{$\rightarrow$}] |z| = $\sqrt{x^2+y^2}$\\
 \item[{$\rightarrow$}] $|z|^2 = z.\overline{z}$\\
 \item[{$\rightarrow$}] |-z| = |z| = $|\overline{z}|$\\
\item[{$\rightarrow$}] $(z \in \mathbb{R}) \Leftrightarrow (\overline{z} = z)$\\
\item[{$\rightarrow$}] $(z \in i\mathbb{R}) \Leftrightarrow (\overline{z} = -z)$\\
 \item[{$\rightarrow$}] |Im(z)| $\leq$ |z|\\
 \item[{$\rightarrow$}] |Re(z)| $\leq$ |z|\\
 \item[{$\rightarrow$}] $|z.z'| = |z|.|z'|$ \\
 \item[{$\rightarrow$}] $|z+z'| \leq |z|+|z'|$\\
\item[{$\rightarrow$}] $e^{i(\theta + \theta'}) = cos(\theta+\theta')+isin(\theta+\theta')$\\
\item[{$\rightarrow$}] $\dfrac{e^{i\theta}+e^{-i\theta}}{2} = cos(\theta)$\\
\item[{$\rightarrow$}] $\dfrac{e^{i\theta}-e^{-i\theta}}{2} = sin(\theta)$\\
\item[{$\rightarrow$}] cos(iy) = ch(y)\\
\item[{$\rightarrow$}] sin(iy) = ish(y)\\
\item[{$\rightarrow$}] Formule de Moivre : $$(e^{i\theta})^n =e^{in\theta} \Leftrightarrow (cos(\theta)+isin(\theta))^n = cos(n\theta)+isin(in\theta)$$
\item[{$\rightarrow$}] z = x+iy = $\rho e^{i\theta}$\\
\item[{$\rightarrow$}] Racine $n^{eme}$ de l'unite : $$z^n = 1 \Leftrightarrow \exists k \in \left\lbrace 0,1,...,n-1\right\rbrace~ z = e^{i\dfrac{k2\pi}{n}}$$
\item[{$\rightarrow$}] Racine $n^{eme}$ d'un complexe non nul, avec $z = \rho e^{i\theta}$, $z_0 = \rho_0 e^{i\theta_0}$ : $$z^n = z_0 \Leftrightarrow \exists k \in \left\lbrace 0,1,...,n-1\right\rbrace~ z =A.e^{i\dfrac{k2\pi}{n}}$$ Avec A la solution évidente (Passage à la racine $n^{eme}$)
\end{itemize}